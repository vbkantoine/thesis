\documentclass[a4paper]{book}

\usepackage[utf8]{inputenc}

\usepackage{geometry}
\usepackage{hyperref}
\usepackage{xcolor}
\usepackage[english]{babel}
\usepackage{graphicx}


\usepackage{amsmath}
\usepackage{amsfonts}
\usepackage{amsthm}

\usepackage{mathrsfs}
\usepackage{bbm}
\usepackage{enumitem}
\usepackage{cleveref}
\usepackage{titlesec}
\usepackage{minitoc}
\usepackage{fancyhdr}
\usepackage{ragged2e}

\usepackage{tikz}
\usetikzlibrary{arrows.meta}

\hypersetup{colorlinks, linkcolor=red, citecolor=blue, urlcolor=blue}



% mathbb
\newcommand{\CC}{\mathbb{C}}
\newcommand{\EE}{\mathbb{E}}
\newcommand{\NN}{\mathbb{N}}
\newcommand{\PP}{\mathbb{P}}
\newcommand{\QQ}{\mathbb{Q}}
\newcommand{\RR}{\mathbb{R}}
\newcommand{\VV}{\mathbb{V}}

% mathcal
\newcommand{\cA}{\mathcal{A}}
\newcommand{\cB}{\mathcal{B}}
\newcommand{\cC}{\mathcal{C}}
\newcommand{\cD}{\mathcal{D}}
\newcommand{\cE}{\mathcal{E}}
\newcommand{\cF}{\mathcal{F}}
\newcommand{\cG}{\mathcal{G}}
\newcommand{\cH}{\mathcal{H}}
\newcommand{\cI}{\mathcal{I}}
\newcommand{\cJ}{\mathcal{J}}
\newcommand{\cL}{\mathcal{L}}
\newcommand{\cM}{\mathcal{M}}
\newcommand{\cN}{\mathcal{N}}
\newcommand{\cO}{\mathcal{O}}
\newcommand{\cP}{\mathcal{P}}
\newcommand{\cQ}{\mathcal{Q}}
\newcommand{\cR}{\mathcal{R}}
\newcommand{\cT}{\mathcal{T}}
\newcommand{\cW}{\mathcal{W}}
\newcommand{\cX}{\mathcal{X}}
\newcommand{\cY}{\mathcal{Y}}
\newcommand{\cZ}{\mathcal{Z}}

% mathscr
\newcommand{\sA}{\mathscr{A}}
\newcommand{\sB}{\mathscr{B}}
\newcommand{\sI}{\mathscr{I}}
\newcommand{\sM}{\mathscr{M}}
\newcommand{\sO}{\mathscr{O}}
\newcommand{\sP}{\mathscr{P}}
\newcommand{\sR}{\mathscr{R}}
\newcommand{\sT}{\mathscr{T}}
\newcommand{\sX}{\mathscr{X}}
\newcommand{\sY}{\mathscr{Y}}
\newcommand{\sZ}{\mathscr{Z}}

% \theoremstyle{plain}
% \newtheorem{thm}{Theorem}[chapter]
% \crefname{thm}{theorem}{theorems}
% \newtheorem{prop}{Proposition}[chapter]
% \crefname{prop}{proposition}{propositions}
% \newtheorem{lem}{Lemma}[chapter]
% \crefname{lem}{lemma}{lemmas}
% \newtheorem{cor}{Corollary}[chapter]
% \crefname{cor}{corollary}{corollaries}

% \theoremstyle{definition}
% \newtheorem{defi}{Definition}[chapter]
% \crefname{defi}{definition}{definitions}
% \newtheorem{assu}{Assumption}[chapter]
% \crefname{assu}{assumption}{assumptions}
% \newtheorem{rem}{Remark}[chapter]
% \crefname{rem}{remark}{remarks}
% \newtheorem{ex}{Example}[chapter]
% \crefname{ex}{example}{examples}

% operateurs
\DeclareMathOperator{\Tr}{Tr}
\DeclareMathOperator{\Supp}{Supp}
\DeclareMathOperator{\Vect}{Vect}
\DeclareMathOperator*{\argmin}{arg\,min}
\DeclareMathOperator*{\argmax}{arg\,max}
\DeclareMathOperator{\Cov}{Cov}
\DeclareMathOperator{\Var}{Var}
\DeclareMathOperator{\erf}{erf}
\DeclareMathOperator*{\tend}{\to}
\newcommand{\rD}{\mathrm{D}}
\def\bX{\mathbf{X}}

\newcommand{\conv}[2][\ ]{\overset{#1}{\underset{#2}{\to}}}
\newcommand{\aseq}[2][]{\overset{#1}{\underset{#2}{=}}}
\newcommand{\equi}[1]{\underset{#1}{\sim}}
\newcommand{\mbf}[1]{\mathbf{#1}}
\newcommand{\indic}{\mathbbm{1}}
\let\eps\varepsilon
\let\to\longrightarrow


\newenvironment{abstract}[1][Abstract]{\paragraph{#1}}{}






% \defbibheading{none}{%
% }
% \DeclareSourcemap{
%   \maps[datatype=bibtex]{
%     \map[overwrite]{
%       \perdatasource{biblio.bib}
%       \step[fieldset=keywords, fieldvalue={,conf}, append]
%     }
%   }
% }





\usepackage[osf]{mathpazo} %sc
\linespread{1.05} 

\setlength{\parskip}{0.25em}
\geometry{
	left=20mm,
	top=30mm,
	right=20mm,
	bottom=30mm
}


\usepackage[backend=biber, style=authoryear, natbib=true, maxnames=4, maxcitenames=3, maxbibnames=5]{biblatex}
\addbibresource{biblio.bib}


% full arabic numbering of the pages

% \makeatletter
% \renewcommand{\frontmatter}{\cleardoublepage\@mainmatterfalse}
% \renewcommand{\mainmatter}{\cleardoublepage\@mainmattertrue}
% \makeatother

\theoremstyle{plain}
\newtheorem{thm}{Theorem}[chapter]
\crefname{thm}{theorem}{theorems}
\newtheorem{prop}{Proposition}[chapter]
\crefname{prop}{proposition}{propositions}
\newtheorem{lem}{Lemma}[chapter]
\crefname{lem}{lemma}{lemmas}
\newtheorem{cor}{Corollary}[chapter]
\crefname{cor}{corollary}{corollaries}

\theoremstyle{definition}
\newtheorem{defi}{Definition}[chapter]
\crefname{defi}{definition}{definitions}
\newtheorem{assu}{Assumption}[chapter]
\crefname{assu}{assumption}{assumptions}
\newtheorem{rem}{Remark}[chapter]
\crefname{rem}{remark}{remarks}
\newtheorem{ex}{Example}[chapter]
\crefname{ex}{example}{examples}




\DeclareSymbolFont{CMletters}{OML}{cmm}{m}{it}
\DeclareMathSymbol{\xi}{\mathord}{CMletters}{"18}



\definecolor{niceBlue}{HTML}{023E8A}
\definecolor{niceBluelight}{HTML}{0077B6}
\definecolor{ayellow}{HTML}{D62828}
\hypersetup{linkcolor=niceBlue, citecolor=niceBlue, urlcolor=niceBlue}

% \renewcommand{\beforeminitoc}{\hypersetup{linkcolor=niceBluelight}}
% \renewcommand{\afterminitoc}{\hypersetup{linkcolor=niceBlue}}

\mtcsetrules{minitoc}{off}

\renewcommand{\mtctitle}{}

\usepackage{anyfontsize}
\titleformat{\chapter}[display]{\Huge\raggedleft}{\fontsize{40}{45}\selectfont\color{niceBlue}\thechapter}{0pt}{\vspace*{0.5em}}

% \titleformat*{\section}{\Large\color{niceBlue}}

\titleformat{\part}[block]{\color{niceBlue}\Huge\MakeUppercase\centering}{\begin{center}\fontsize{40}{45}\selectfont\Roman{part}\end{center}}{0pt}{\Huge\scshape}

% \titlelabel{\thepar}
% \titleformat*{\part}{\fontsize{30}{35}\selectfont\scshape\centering}


\renewenvironment{abstract}[1][Abstract]{\begin{center}\begin{minipage}{0.87\textwidth}\paragraph{\hspace*{2.5em}#1}}{\end{minipage}\end{center}}


\renewcommand{\chaptermark}[1]{%
\markboth{#1}{}}
\renewcommand{\sectionmark}[1]{\markright{#1}}


\fancyhead[EL]{\color{niceBlue}\chaptertitlename\ \thechapter.\ \emph{\leftmark}}
\fancyhead[ER]{}
\fancyhead[OR]{\color{niceBlue}\thesection.\ \emph{\rightmark}}
\fancyhead[OL]{}

\renewcommand{\headrulewidth}{0pt}


\crefformat{chapter}{\textcolor{niceBlue}{#2\MakeLowercase\chaptertitlename~#1#3}}
\Crefformat{chapter}{\textcolor{niceBlue}{#2\chaptertitlename~#1#3}}

\crefformat{part}{\textcolor{niceBlue}{#2\MakeLowercase\partname~#1#3}}
\Crefformat{part}{\textcolor{niceBlue}{#2\partname~#1#3}}



\title{Brouillon de manuscrit de thèse}
\author{Antoine Van Biesbroeck}

\begin{document}

\pagestyle{empty}

\maketitle

\dominitoc
\setcounter{tocdepth}{1}
\tableofcontents

\pagestyle{fancy}
%\section{}


\chapter{Introduction}




\section{Motivation and positioning of the thesis}


\section{Outline of the manuscript and contributions}



\chapter{Introduction en français}

\section{Motivation et positionnement de la thèse}

\section{Esquisse du manuscrit et constributions}




\part{Contribution to the reference prior theory}\label{part:ref-theory}
% \renewcommand{\chaptername}{Chapter}
% \renewcommand{\partname}{Part}

\chapter{Review of the reference prior theory}\label{chap:intro-ref}




\begin{abstract}
    The reference prior theory is a perfect tool 
    We provide a comprehensive state-of-the-art. 
    this chapter is also the occasion to introduce 
    %We review the reference prior theory
\end{abstract}

\minitoc


\section{Introduction}

Bayesian analysis offers a coherent framework for integrating prior information and updating beliefs with data. The incorporation of prior knowledge can be motivated by various needs, such as enhancing interpretability, introducing uncertainty, or embedding existing knowledge into the analysis. Actually, the selection of this a priori centralizes a lot of attention in Bayesian inference and can be a critical aspect of the method. As a matter of fact, it can significantly influence the posterior distribution and, consequently, the conclusions drawn from the analysis. Thus, an inconsistent or insufficiently justified choice of prior can jeopardize the validity of the entire study, a concern now well identified in the modern Bayesian workflow described in \cite{gelman_bayesian_2020}, and echoed across applied domains.

A plethora of approaches exist in the literature to address the prior elicitation problem. They offer different strategies for different purposes. 
Comprehensive reviews of such methods are proposed in \cite{mikkola_prior_2023}
and in \cite{consonni_prior_2018}, for instance.
Among the different challenges that are commonly addressed by these methods, we can emphasize (i) the incorporation of information, (ii) the ease of posterior inference, or (iii) the research for interpretability.\\
The former refers to the so-called ``informative priors'', because their distributions tend to concentrate the information. %in some area of the doamin.  significantly inform the posterior distribution with strong distribution tails.
They aim to incorporate genuine prior knowledge about parameters, often arising from past data, domain expertise, physical constraints, or theoretical understanding. %They are particularly common in fields like clinical trials, risk analysis, ecology, and engineering, where historical information is abundant or regulatory frameworks encourage explicit prior modeling.
The difficulty lies in doing so transparently, reproducibly, and justifiably, especially when seeking to avoid subjective bias.\\
Regarding the second, we can recall conjugate priors (see e.g. \cite{robert_bayesian_2007}) that are designed to provide a simple formulation of the posterior. Penalized Complexity (PC) priors \citep{simpson_penalising_2017} and sparse priors \citep{castillo_bayesian_2015} are made to be suitable for high-dimensional problems; and g-priors \citep{liang_mixtures_2008} are used for variable selection in normal regression models.\\
Thirdly, in order to provide a better interpretability of the information transmission process, several studies favor a hierarchical design, in which the prior is the result of a latent prior modeling. This approach allows the use of historical data, as in power priors \citep{chen_relationship_2006} and Information Matrix priors \citep{gupta_information_2009}; or even imaginary data \citep{perez_expectedposterior_2002,spitzner_neutral-data_2011}. Posterior priors, developed for sensitivity analysis studies \citep{bousquet_discussion_2024}, are also part of this approach. %with imaginary data. 
These hierarchical constructions are appreciated for their ability to tackle the improper aspect (they integrate to infinity) of most low-informative priors
(priors whose distribution is more diffuse through the parameter's space, in opposition with informative priors)



Many of these  methods involve subjective choices that remain open to criticism. Often, they are based on the selection of a class and hyperparameters that remain to be tuned.


The research for so-called ``objective priors'' arose in contexts where the incorporation of uncritical beliefs is impossible (either by lack of available beliefs, or by lack of confidence within the existing ones).
In those contexts, there is consensus that an appropriate prior belongs within the non-informative ones. In this respect, \citet{lindley_measure_1956} already suggested the maximization of the Shannon's entropy, as a way to choose a prior that would provide the least possible information.
The same philosophy has been the source of several rules \citep{kass_selection_1996, datta_invariance_1996,berger_objective_2008}, which lead to different definitions for objective priors.

In this chapter and in this thesis, we focus on the reference prior theory, first introduced by \citet{bernardo_expected_1979}. 
Based on an ``expected utility maximization'',
the idea was to shift from minimizing the information in the prior to maximizing the knowledge brought by the observations over prior distribution itself.
This notion of expected information, is defined with the support of the mutual information which allows for a formal construction of priors that are designed to let the data dominate the inference.

The reference prior theory has been extensively studied since its deepen formalization by \citet{berger_formal_2009}. For instance, \citet{mure_objective_2018} provides a comprehensive review of the theory.
This chapter proposes a novel brief review of the reference prior theory, fixing the framework (in section ??) and defining its main tools, such as the mutual information (in section ??). The reference prior are formally defined in section ??, and we review its different properties that are proven in the literature.
Eventually, this section is the occasion to discuss the limitations of the standard framework of the reference prior, and to suggest a solution based on novel formalization of the Bayesian inference tools.  The latter fixes the mathematical elements and notations that we will manipulate through the whole manuscript.






%The different approaches are generally categorized within two classes: informative and non-informative priors.


%Despite their variety, many of them involve subjective choices that remain open to criticism. Indeed, prior elicitation methods generally con- sist of choosing a way to inform a prior. They are often based on the selection of a class and hyperparameters that remain to be tuned.

% Informative priors aim to incorporate genuine prior knowledge about parameters, often arising from past data, domain expertise, physical constraints, or theoretical understanding. They are particularly common in fields like clinical trials, risk analysis, ecology, and engineering, where historical information is abundant or regulatory frameworks encourage explicit prior modeling.


\section{The standard Bayesian framework}

\subsection{A minimal framework}

The Bayesian framework is commonly defined from the given of statistical model.
Let $(\Omega,\sP,\PP)$ be a probability space, and define $Y$, a random variable defined on $\Omega$ and taking values on a measurable space $(\cY,\sY)$.
A statistical model is characterized by a collection of parameterize probability measure $(\PP_{Y|\theta})_{\theta\in\Theta}$ on $(\cY,\sY)$.
The Bayesian viewpoint considers a random variable $T:\Omega\to\Theta$ taking values in a measurable space $(\Theta,\sT)$, following a prior distribution denoted $\Pi$, and which is such that the conditional distribution of $Y$ to $T=\theta$ is $\PP_{Y|\theta}$. In other terms:
    \begin{equation}
        \forall A\in\sY,\,B\in\sT,\, \PP(Y\in A,\, T\in B) = \int_B \PP_{Y|\theta}(A) d\Pi(\theta).
    \end{equation}
Note that the above quantities are well-defined and are almost surely unique, at least while $(\cY,\sY)$, $(\Theta,\sT)$ and $(\PP_{Y|\theta})$ align with the statements of the disintegration theorem (see e.g. \cite{chang_conditioning_1997}).

In practice, $k$ realizations $\mbf y= (y_1,\dots,y_k)$ of the r.v. $Y$ are observed, they are supposed to be identically distributed and independent conditionally to $T$, meaning they are the realization of a random vector $\mbf Y = (Y_1,\dots,Y_k)$, whose distribution $\PP_{\mbf Y}$ is defined on $(\cY^k,\,\sY^{\otimes k})$ by:
\begin{equation}
    \forall A\in\sY^{\otimes k},\, \PP_{\mbf Y}(A) = \int_\Theta \PP_{\mbf Y|\theta}(A)d\Pi(\theta),
\end{equation}
where $\PP_{\mbf Y|\theta}:=\PP_{Y|\theta}^{\otimes k}$. The distribution $\PP_{\mbf Y}$ is called the marginal distribution.

Given the observations $\mbf y$, the posterior distribution, denoted $P_{T|\mbf y}$, is then defined as the a.s. unique one verifying
    \begin{equation}
        \forall A\in\sY^{\otimes k},\,\forall B\in\sT,\, \int_{A}\PP_{T|\mbf y}(B)d\PP_{\mbf Y}(\mbf y) = \int_B \PP_{\mbf Y|\theta}(A)d\Pi(\theta).
    \end{equation}
We can notice that the posterior distribution is always absolutely continuous w.r.t. the prior distribution.


\subsection{Likelihood and densities}


It is common to assume that the statistical model admits a likelihood: there exists a collection of density functions $(\ell(\cdot|\theta))_{\theta\in\Theta}$  w.r.t. a common  $\sigma$-finite measure $\mu$ on $\cY$ such that for $\Pi$-a.e. $\theta\in\Theta$:
    \begin{equation}
        \forall A\in\sY,\,\PP_{Y|\theta} = \int_A\ell(y|\theta)d\mu(y).
    \end{equation}
This way, $\PP_{\mbf Y}$ admits a \textbf{marginal density} $p_{\mbf Y}$ w.r.t. $\mu^{\otimes k}$ defined as
    \begin{equation}
        \forall\mbf y\in\cY^k,\, p_{\mbf Y}(\mbf y) = \int_\Theta\prod_{i=1}^k\ell(y_i|\theta) d\pi(\theta) = \int_\Theta \ell_k(\mbf y|\theta)d\pi(\theta),
    \end{equation}
where $\ell_k(\mbf y|\theta)$ denotes $\prod_{i=1}^k\ell(y_i|\theta)$ and is called the likelihood.

Also, we can suppose that $\Pi$ admits a \textbf{prior density} $\pi$ w.r.t. a $\sigma$-finite measure $\nu$ on $\Theta$ (usually, it is simply assumed that $\Theta\subset\RR^d$ and $\nu$ is the Lebesgue measure). 
%We acknowledge that it is usual to make the confusion between the prior distribution and its density, while there is no ambiguity doing so. 
%Thus, we denote by $\pi$ the density of $\i$ w.r.t. $\nu$. 
The posterior distribution admits a \textbf{posterior density} $p(\cdot|\mbf y)$ w.r.t. $\nu$ defined by
    \begin{equation}
        \forall\theta\in\Theta,\, p(\theta|\mbf y) = \frac{\ell_k(\mbf y|\theta)\pi(\theta)}{p_{\mbf Y}(\mbf y)}\text{\ if\ } p_{\mbf Y}(\mbf y)\ne0,\ p(\theta|\mbf y)=1 \text{\ otherwise}.
    \end{equation}
%The same way, 
To conclude this section, we acknowledge the usual confusion between the notation for the random variable $T$ and notation for the values it takes $\theta$. Thus, we might refer as ``the distribution of $\theta$'' or the ``distribution conditionally to $\theta$'' to respectively mention the prior distribution or the conditional distributions to $T=\theta$.


\subsection{Improper priors}

It is common when dealing with non-informative priors in Bayesian analysis to define improper ones.
Improper priors are defined as distributions whose density $\pi$ integrates to infinity: $\int_\Theta\pi(\theta)d\nu(\theta)=\infty$. 

Of course, such a prior is not a probability distribution anymore, so that the framework elucidated in Section ?? does not stand. 
A way-around to be able to use improper priors is to focus on restrictions of the model. In precise terms,  as $\nu$ is a $\sigma$-finite measure, there exist an increasing sequence $(B_n)$ of elements in $\Theta$ such that 
$\Theta=\bigcup{n\in\NN}B_n$ and $\nu(B_n) <\infty$.
An improper prior is said to be admissible if for any $n$, $\int_{B_n}\pi(\theta)d\nu(\theta)<\infty$, and if there exists $n_0$ such that $\int_{B_n}\pi(\theta)d\nu(\theta)>0$.

In this case, on any $\tilde\Theta\in\sT$ such that there exists $N\in\NN$ verifying $\tilde\Theta\subset\bigcup_{n\geq n_0}^NB_n$ and such that $B_{n_0}\subset\tilde\Theta$, one can define the probability space $(\tilde\Theta,\tilde\sT,\tilde\Pi)$ with:
    \begin{equation}
        \tilde\sT = \{B\cap\tilde\Theta,\,B\in\sT\},\quad\forall \tilde B\in\tilde\sT,\,\tilde\Pi(\tilde B) = \frac{\int_{\tilde B}\pi(\theta)d\nu(\theta)}{\int_{\tilde\Theta}\pi(\theta)d\nu(\theta)}.
    \end{equation}
Thus, restricting the study to $\tilde\Theta$ let the modeling of previous sections to stand, that modeling is the restricted modeling driven by $\tilde\Theta$. 
The probability $\Tilde\Pi$ (respectively its density) is referred as the restriction of the prior $\Pi$ (resp. of the prior density $\pi$) to $\tilde\Theta$.

On any restricted model, the posterior and the marginal distributions exist. 
Given two restrictions driven by $\tilde\Theta_1$ and $\tilde\Theta_2$, the restricted prior densities $\tilde\pi_1$ and $\tilde\pi_2$ are equal on $\tilde\Theta_1\cap\tilde\Theta_2$ up to a constant:
    \begin{equation}
        \tilde\pi_1(\theta) = K\tilde\pi_2(\theta),\quad\text{so that}\quad \forall \mbf y\in\cY^k,\,\tilde p_{1,\mbf Y}(\mbf y) = K\tilde p_{2,\mbf Y}(\mbf y),
    \end{equation}
where $\tilde p_{1,\mbf Y}$ (reps. $\tilde p_{2,\mbf Y}$) denotes the marginal density in the restricted model driven by $\tilde\Theta_1$ (resp. $\tilde\Theta_2$). Thus, calling $\tilde p_1(\cdot | \mbf y)$ (resp. $\tilde p_2(\cdot | \mbf y)$) the posterior density given the observations $\mbf y$ and under the restricted model driven by $\tilde\Theta_1$ (resp. $\tilde\Theta_2$), we have
    \begin{equation}
        \forall\theta\in\tilde\Theta_1\cap\tilde\Theta_2,\, \tilde p_1(\theta | \mbf y) =\tilde p_2(\theta | \mbf y).
    \end{equation}
We conclude that the posterior density is always well defined on $\Theta$ when the improper prior is admissible.
If the posterior density integrates to $\infty$, the posterior is said to be improper.




%We denote that density by $\pi$ as well, this slight abuse 

\section{Mutual information and reference priors}

\subsection{Mutual information}


The principle of the theory 
is to minimize the role of the prior within the posterior, in order to maximize the information gain from the data.
Thus, it amounts to maximize how ``away'' is the prior to the posterior.

When it resorts to compare information between probability distributions, we generally consider dissimilarity measures. The most common one is the Kullback-Leibler divergence, it is the one that is considered in the historical settings of the reference prior theory.
We recall below the expression of the Kullback-Leibler divergence between two distribution $P$ and $Q$ on $\cX$, which admit densities w.r.t. a common measure $\omega$:
    \begin{equation}
        \text{KL}(Q||P)=\int_\cX\log\left(\frac{q(x)}{p(x)}\right)q(x)d\omega(x).
    \end{equation}

Thus, the idea is to maximize the divergence $\text{KL}(\PP_{T|\mbf y}||\Pi)$. As observed sample $\mbf y$ is unknown, we consider the average value of the divergence:
\begin{defi}[Mutual information]\label{def:intro-ref-MI}
    Given a Bayesian framework defined as in Section ??, and given a number of observations $k$, the mutual information is defined as the quantity
    \begin{equation}
        \sI^k(\Pi) =  \EE_{\mbf Y\sim \PP_{\mbf Y}}[\text{KL}(\PP_{T|\mbf Y}||\Pi)] =  \int_{\cY^k}\text{KL}(\PP_{T|\mbf y}||\Pi) p_{\mbf Y}(\mbf y)d\mbf y.
    \end{equation}
\end{defi}


%This quantity is called the mutual information.

We recall that the definition of the mutual information is not limited to the scope of Bayesian analysis. In information theory it is common to refer to the mutual information $\sI(X,Z)$ between a random variable $X$ and a random variable $Z$. The quantity in \cref{def:intro-ref-MI} aligns with their definition of the mutual information $\sI(T,\mbf Y)$ between the random parameter $T$ and the random data $\mbf Y$.

One can apply Fubini-Lebesgue's theorem to write the mutual information as
\begin{equation}
    \sI^k(\Pi) =  \EE_{\mbf Y\sim \PP_{\mbf Y}}[\text{KL}(\PP_{T|\mbf Y}||\Pi)] =  \EE_{T\sim\Pi}[\text{KL}(\PP_{\mbf Y|\theta}||\PP_{\mbf Y}) ]
\end{equation}
This second expression allows an interpretation of the mutual information from another viewpoint: it measures how the parameter $T$ impacts the distribution of $\mbf Y$.


This construction is called a way-around because the framework remains incomplete: $T$ is not appropriately defined anymore, nor the marginal distribution $\PP_{\mbf Y}$. Later in this chapter, we will introduce a novel construction that authorize the definition of improper prior in a more suitable way.
Before that, the current construction is enough to define the reference priors among the admissible priors: priors that are either proper, either admissible improper priors.
%In the next section, a classe of 







\subsection{Reference priors}

Reference priors are defined as the priors that are expected to maximize the role of the observed data within the posterior distribution.
Consequently, they are historically sought as maximizer of the mutual information.

However, formally maximizing the mutual information $\sI^k(\Pi)$ w.r.t. $\Pi$ presents three main issues. %, as its expression depends on the value of $k$.
First, its expression depends on the value of the number of  observations $k$, and $k$ alters the form of its maximal argument. %The prior expression comes, by nature, prior to the observations, and should depend neither on $k$, neither on $\mbf y$. A 
%In normal settings, there are no knowledge on the v
This problem is tackled arguing that $\sI^k$ only measures a limited quantity of information brought by $T$ onto the distribution of the data $\mbf Y$. The Bayesian framework does considerate the data $(Y_1,\dots,Y_k)$ to be independent, so that the distribution of $\mbf Y$ is enriched as $k$ rises.
According to \citet{bernardo_bayesian_1994}, the limit $\sI^\infty$ (if it exists) measures the knowledge missing form the prior.
Thus, the formal definition tends to maximize $\sI^k$ as $k$ diverges to $\infty$.\\
The second issue comes with the fact that the quantity $\sI^k(\Pi)$ does not often admit a finite limit as $k\to\infty$.\\
The third issue is that the mutual information is only correctly defined when the prior is proper, while many non-informative priors used in Bayesian analysis are improper.

The commonly admitted definition for the reference prior, which takes into account the three issues aforementioned is the following, adapted from \cite{dey_reference_2005}.
\begin{defi}[Reference priors]\label{def:intro-ref:ref-priors}
    Let $\cP$ be class of priors on $\Theta$, a prior $\Pi\in\cP$ %whose prior density is $\pi$ 
    is a reference prior over $\cP$ if there exist an increasing sequence $(\Theta_i)_{i\in\NN}$ such that $\bigcup_{i\in\NN}\Theta_i=\Theta$ with for any $i$: $0<\Pi(\Theta_i)<\infty  $ and
        \begin{equation}
            \lim_{k\rightarrow\infty} (\sI^k_i(\Pi_i) - \sI^k_i(P_i) )\geq 0 \text{\ for all\ } P\in\cP_i
        \end{equation}
    with equality if and only if $\Pi_i=P$; with $\Pi_i$ (resp. $P_i$) being the restriction of $\Pi$ (resp. $P$) on $\Theta_i$, $\sI_i^k$ denoting the mutual information on the restricted modeling driven by $\Theta_i$, and $\cP_i=\{P\in\cP,\,0<P(\Theta_i)<\infty\}$.
\end{defi}


This definition will serve as a reference. It considers a set $\cP$ called a class of priors. 
We provide a thorough definition of such a class in Section ??.
Until then, it refers to a set of priors that are admissible as described in the section ??.


We must note that other definition of the reference priors exist.
In the geneses of the reference prior theory, \citet{bernardo_reference_1979} suggested maximizing the mutual information $\sI^k$, and derive the limit of the maximizers when $k\to\infty$ to define reference priors.
In \cite{berger_formal_2009}, the authors prove that in the real and one parameter case (i.e. $\Theta\subset\RR$), this strategy is consistent with \cref{def:intro-ref:ref-priors}. Their result is the following, sometimes taken as a defintion.

\begin{thm}[Explicit for of the reference prior]\label{thm:intro-ref:explicitRP}
    Assume $\Theta\subset\RR$ with $\nu$ being the Lebesgue measure.\\
    Call $\cP_s$ the class of admissible priors that are positive and continuous, and that issue proper posterior distributions.\\
    Let $\Pi^\ast\in\cP_s$ we call $p^\ast(\cdot|\cdot)$ its posterior and define for any interior point $\theta_0\in\Theta$,
        \begin{equation}
            % \begin{aligned}
                f_k(\theta) = \exp\left(\int_{\cY^k}\ell_k(\mbf y|\theta)\log(p^\ast(\theta|\mbf y)) d\mbf y \right) \quad\text{and}\quad  %\\
                f(\theta) = \lim_{k\rightarrow\infty}\frac{f_k(\theta)}{f_k(\theta_0)}.
            % \end{aligned}
        \end{equation}
    If: (i) $ \forall \theta,\eps>0,\,\int_{|\tau-\theta|} p^\ast(\tau|\mbf y)d\tau \conv[\PP_{\mbf Y|\theta}]{k\rightarrow\infty}1$,\\ (ii) each $f_k$ is continuous with $\forall k,\theta,\, \frac{f_k(\theta)}{f_k(\theta_0)} <h(\theta)$ such that $h$ is integrable on any compact set, and\\
    (iii) $f$ is the density of $F\in\cP_s$  and is such that for any increasing sequence of compacts $(\Theta_i)_{i\in\NN}$ that covers $\Theta$, calling $g_i$ the restricted posterior density of $F$ on $\Theta_i$, we have $\forall \mbf y,\, \int_{\Theta_i}g_i(\theta|\mbf y)\log\frac{g_i(\theta|\mbf y)}{g(\theta|\mbf y)}d\theta \conv{i\rightarrow\infty}0$ where $g$ is $F$'s posterior,\\
    then $F$ is a reference prior over $\cP_s$.
 \end{thm}


%


\Cref{def:intro-ref:ref-priors} and \cref{thm:intro-ref:explicitRP}
both define reference prior as maximizer of the mutual information.
While there is consensus that such an approach leads to appropriate objective priors in low-dimensional cases, it is not always the case in high-dimensional settings.
Indeed, the prior maximizing the mutual information has often non-desirable properties in such scenarios, according to \citet{berger_overall_2015}. %They say that the actual maximizer of the mutual information is either too diffuse either too con
In \cite{berger_development_1992}, the authors details their suggestion to define the reference priors hierarchically, by considering an ordering of the parameters:
 \begin{equation}\label{eq:intro-ref:hierartheta}
     \theta = (\theta_1,\dots,\theta_r) \in \Theta=\Theta_1\times\dots\times\Theta_r,
 \end{equation}
Typically, it is recommended to assume $\Theta_j\subset\RR^{d_j}$ with small dimensions $d_j$ (e.g., lower or equal than $2$) for any $j\in\{1,\dots,r\}$, and to sequentially build a reference prior on the $\Theta_j$,  $j\in\{1,\dots,r\}$:
 \begin{enumerate}
     \item initially fix $\ell_k^1=\ell_k$;
     \item for any values of $\theta_{j+1},\dots,\theta_r\in\Theta_{j+1}\times\dots\times\Theta_r$, compute a reference prior (in the sense of \cref{def:intro-ref:ref-priors}) $\pi_j(\cdot|\theta_{j+1},\dots,\theta_r)$ under the model with likelihood $\theta_j\mapsto\ell_k^j(\mbf y|\theta_j,\dots,\theta_r)$;
     \item derive $\ell_k^{j+1}$ such as 
         \begin{equation}\label{eq:hier:condlikeint}
            \ell_k^{j+1}(\mbf y|\theta_{j+1},\dots,\theta_r) =  \int_{\Theta_j}\ell_k^j(\mbf y|\theta_j,\dots,\theta_r)d\pi_j(\theta_j|\theta_{j+1},\dots,\theta_r).
         \end{equation}
 \end{enumerate}


This construction is often considered in the literature when it resorts to define reference prior in multidimensional context. 
However, it relies on the hierarchization of the parameter fixed in \cref{eq:intro-ref:hierartheta}, and the recommended  one depend on the model of interest.
There are a few models, such as the multinomial one for which the different reference priors resulting from this construction have been extensively studied. In the case of the multinomial model, it is claimed that this construction with an ordering where $r=d$ when $\Theta\subset\RR^d$ must be favored  \citep{berger_ordered_1992,berger_overall_2015}.




\subsection{Properties}


travaux de clarke

%\section{The concept of objective priors and the role of mutual information}

%\section{Reference prior definition and properties}

\section{A framework for Bayesian inference with improper priors}

Framework et discussion sur ce qu'on a dit avant, comment ça tient toujours

\section{Open paths and conclusion}


This section should conclude the state-of-the-art and delimit the limitations of the current theory on different aspects. 














\chapter{Generalized mutual information and their reference priors}\label{chap:ref-generalized}


\begin{abstract}[\hspace*{-10pt}]
    This chapter draws mainly on the published work: \fullcite{van_biesbroeck_generalized_2024}  % Ce chapitre reprend principalement les travaux publiés dans: 
\end{abstract}

\begin{abstract}
    abstract
\end{abstract}

\minitoc

\section{Introduction and motivations}

The reference prior theory aims at defining priors that are the most objective possible




\section{Generalized mutual information}

\subsection{Definitions and developments}

\subsection{Motivating $f$-divergences}


\section{Generalized reference priors}

\subsection{Definitions}

\subsection{Results when $\delta<0$}

\subsection{Results when $\delta>0$}


\section{Discussions}

    \subsection{About the assumptions and their limitations}


    \subsection{About the robustness of Jeffreys prior with different divergences}


        \subsubsection{When $\delta=0$}

        \subsubsection{A heuristic for other $f$-divergences}

        \subsubsection{A simple development with a Maximum Mean Discrepancy divergence}

\section{Conclusion and prospects}
















\chapter{Properly constrained generalized reference~priors}\label{chap:constrained-prior}



\begin{abstract}[\hspace*{-10pt}]
    This chapter draws mainly on the submitted work: \fullcite{van_biesbroeck_properly_2024}  % Ce chapitre reprend principalement les travaux publiés dans: 
\end{abstract}

\begin{abstract}
    Reference priors are widely recognized for their objective nature. Yet, they often lead to intractable and improper priors, which complicates their application.
Besides, informed prior elicitation methods are penalized by the subjectivity of the choices they require. %to be made.
In this chapter, we aim at proposing a reconciliation of the aforementioned aspects. Leveraging the objective aspect of the reference prior theory, we introduce two strategies of constraint incorporation to build tractable reference priors.
One provides a simple and easy-to-compute solution when the improper aspect is not questioned, and the other introduces constraints to ensure the reference prior is proper, or it provides proper posterior.
Our methodology emphasizes the central role of Jeffreys prior decay rates in this process, and the practical applicability of our results is demonstrated using an example taken from the literature.
\end{abstract}

\minitoc



\section{Introduction}\label{sec:BA:intro}


The reference prior theory represents a widely elected theory for constructing priors that are qualified as ``objective''.
The theory has been thoroughly introduced in \cref{chap:intro-ref}. %It defines priors that maximize the impact of the observed data over themselves in the posterior definition.
The theory provides a formal mechanism to incorporate prior information in a way that maximizes the information gained from the data within the issued \emph{a posteriori} quantities.
In opposition with a plethora of existing  methods for building a prior (see e.g. \cite{mikkola_prior_2023}), 
this process is tuned to prevent the incorporation of subjective beliefs in the workflow.


However, despite their  objective nature,  their implementation is often cumbersome and not always recommended in high dimensions \citep{berger_overall_2015}. %Moreover, 
Moreover, the low-informative nature of these priors is associated with their common improper aspect, necessitating careful handling to ensure valid statistical inference.
Thus, the construction of priors is expected to strike a balance between several criteria. While many works restrict their sets of priors to ones that are tractable, proper, or suitable for high dimensions, others seek to minimize any source of subjectivity.

This chapter aims to reconcile all these viewpoints to improve the prior elicitation.
Our contribution takes the form of an enrichment of the reference prior theory to leverage the objective aspect that it provides to reference priors. Building on the developments presented in \cref{chap:ref-generalized},
we restrict priors to the ones that belong to  well-chosen ---and not too restrictive--- sets, and we introduce two strategies to define  convenient reference priors.
% First 
Our first strategy provides a simple,  tractable solution for constraining reference priors when the improper aspect is not questioned. The second, by contrast, introduces constraints that lead to  reference priors that are proper, or lead to proper posteriors. 
% 
For both strategies, we try to examine the potential loss of objectivity induced by the constraints, and we discuss their limits.
Our results emphasize the central role of Jeffreys prior decay rates when they are improper.
Additionally, we draw attention to the fact that our methodology opens a way to define various reference priors on the basis of constraints that could result from any other motivation.



This chapter is organized as follows.
In \cref{sec:BA:defsnotsmots}, after reviewing briefly the notations for the generalized reference priors framework, we develop our motivation and the objective of this work. This section is also the occasion to introduce a novel definition that is useful for the rest of the study: the quasi $D$-reference priors.
Our main results on constrained reference priors are presented in \cref{sec:BA:ress}, and discussed in \cref{sec:BA:discu}.
Then, the  practical aspect of our work is studied by the application of our method to an example taken from  the literature in \cref{sec:BA:exa}. Detailed mathematical proofs are compiled in \cref{sec:BA:proofs}. \Cref{sec:BA:conclusion} terminates the chapter with a conclusion.


\section{Definitions, notations and motivation}\label{sec:BA:defsnotsmots}

\subsection{Notations}\label{sec:BA:nots}

% \subsection{}

% notations IM, results of preceeding chapters, quasi-reference priors

%motivations %maybe its whole section

    %\subsection{The limitations of Jeffreys prior}

    %\subsection{}

In this work we consider a statistic model characterized by a collection of probability distributions $(\PP_{Y|\theta})_{\theta\in\Theta}$ on  a measurable set $(\cY,\sY)$. 
We consider the same construction of the Bayesian framework as in \cref{chap:intro-ref} (\cref{sec:intro-refs:limits}): considering any prior $\varPi$ (that is a $\sigma$-finite measure on $\Theta$) we denote $\mbf Y_k$ a random vector of $k$ observations whose distribution conditionally to $T=\theta$ is $\PP_{\mbf Y^k|\theta}=\PP_{Y|\theta}^{\otimes k}$, where $T$ is a r.v. whose distribution is the prior $\varPi$ (see \cref{sec:intro-ref:novelframework}).

The modeling is supposed to be regular: we assume $\Theta\subset\RR^d$ with $\nu$ being the Lebesgue measure on $\RR^d$ and every prior $\varPi$ is supposed to admit a density $\pi$ w.r.t. $\nu$ (i.e. $\varPi\in\sM^\nu$).
We also assume that the model admits a likelihood, denoted by $\ell$ with for any $\theta\in\Theta$ and $\mbf y\in\cY^k$, $\ell_k(\mbf y|\theta):= \prod_{i=1}^k\ell(\mbf y|\theta)$. We suppose that it verifies \cref{assu:intro-ref:jeffreysexist} in \cref{chap:intro-ref}, making the Fisher information matrix (denoted $\cI$) and the Jeffreys prior (whose density is denoted $J$) being well-defined.
The marginal distribution (resp. density)  is denoted by $\PP_{\mbf Y_k}$ (resp. $p_{\mbf Y_k}$) and the posterior distribution (resp. density) given the observations $\mbf y\in\cY^k$ is denoted by $\PP_{T|\mbf y}$ (resp. $p(\cdot|\mbf y)$).



Given these notations, we recall the expression of the generalized mutual information, defined in \cref{chap:ref-generalized} when $\varPi$ is proper:
\begin{equation}
    \sI_D^k(\varPi) := \EE_{T\sim\varPi}[D(\PP_{\mbf Y_k}||\PP_{\mbf Y_k|T} )],
\end{equation}
with $D$ being a dissimilarity measure.
In this chapter, we mostly focus on such generalized mutual information when $D$ is a $\delta$ divergence with $\delta\in(0,1)$. For some $\delta\in(0,1)$, the notation $D_\delta$ will refer to the $\delta$-divergence whose expression is reminded below:
    \begin{equation}
        D_\delta(P||Q) = \int_\cX f_\delta\left( \frac{p(x)}{q(x)}  \right) q(x) d\omega(x)\quad \text{with}\quad f_\delta(x) = \frac{x^\delta-\delta x-(1-\delta)}{\delta(1-\delta)},
    \end{equation}
where $p,q$ respectively are densities of $P$ and $Q$ w.r.t. a common measure $\omega$ on $\cX$.
When evoking reference priors in a general way, we will refer to generalized reference priors, as proposed in \cref{chap:ref-generalized}.
We remind below their definition:
\begin{defi}[Generalized reference prior]\label{def:BA:genref}
    Let $D$ be a dissimilarity measure and $\cP$ a set of priors on $\Theta$. A prior $\varPi\in\cP$ is called a $D$-reference prior over $\cP$ with rate $\varphi(k)$ if there exists an openly increasing  sequence of compact subsets $(\Theta_i)_{i\in\NN}$
    such that $\bigcup_{i\in\NN}\Theta_i=\Theta$ and for any $i$: $0<\varPi(\Theta_i)<\infty $ and
    % with $\pi^\ast(\Theta_i)>0$, $\Theta_i\subset\Theta$, $\bigcup_{i\in I}\Theta_i=\Theta$ such that
        \begin{equation} %\label{eq:defrefpriorsi}
            \lim_{k\rightarrow\infty}\varphi(k)[\sI^k_D(\varPi(\cdot|\Theta_i))-\sI^k_D(P(\cdot|\Theta_i))] \geq0 \text{\ for all\ } P\in\cP\text{\ verifying\ }0<P(\Theta_i)<\infty;
        \end{equation}
    where  $\varphi(k)$ is a {positive and}  monotonous function of $k$. It is said to be unique if for any other $D$-refernce prior $\varPi'$, $\varPi\simeq\varPi'$.
\end{defi}

We also remind the following result on the $D_\delta$-mutual information and their reference priors (see \cref{chap:ref-generalized}).
%  that are defined considering a generalized ,
\begin{thm}\label{thm:BA:l(pi)}
    Suppose $\Theta$ to be compact and $\varPi\in\sM^\nu_\cC$ be a prior with $\varPi(\Theta)=1$. The $D_\delta$-mutual information admits a limit:
        \begin{equation}
            \lim_{k\rightarrow\infty} k^{d\delta/2} \sI_{D_\delta}(\varPi) = l(\pi) - (\delta(1-\delta))^{-1}, \quad l(\pi) = C_\delta \int_{\Theta}\pi(\theta)^{1+\delta} |\cI(\theta)|^{-\delta/2}  d\theta ,
        \end{equation}
        where $\pi$ is the density of $\varPi$, and with $C_\delta= (2\pi)^{d\delta/2} (1-\delta)^{-d/2}/(\delta(\delta-1))$.\\
    Call $\cR\subset\sR_{\cC^b}$ a set of densities such that $M(\cR)=\cP$, with $M$ mapping a density to its associated prior. Then
        $\varPi$ is a $D_\delta$-reference prior over $\cP$ iff $\pi$ maximizes $l$ over $\cR$. 
\end{thm}




\subsection{Objective and motivation}\label{sec:BA:mots}



We already know that the definition of the reference prior and the $D_\delta$-reference prior is satisfied by the Jeffreys prior over the large set of priors $\sM^\nu_\cC$ in most cases (see \cref{chap:intro-ref,chap:ref-generalized}).
% in most case (see \cref{chap:intro-ref} for ). 

%the large set of priors admitting locally bounded and a.e. continuous densities w.r.t. the Lebesgue measure \citep{VanBiesbroeckBA2023}. 

This result is, however, limiting and disappointing in some cases. The reasons are the following ones: (i) the Jeffreys prior is not recommended in high-dimensional problems as it is known to be ``either too diffuse or too concentrated'' \citep{berger_overall_2015}; moreover (ii) when the expression of the likelihood is itself complex, the computation of the Jeffreys prior can become  intractable; %which is why (iii) in practice a restriction to the set of priors to ones which are easier to compute is often favored; 
also (iii) the Jeffreys prior is known to often lead to an improper prior, which does not necessarily issue a proper posterior distribution, essential for practical \emph{a posteriori} inference and sampling.

To tackle these limitations, we propose in this work to restrict the set of priors over which we derive the reference priors.
Indeed, the reference prior definition is usually considered with very large sets of priors, which are constrained only by some regularity assumptions imposed to the priors (such as continuity, positivity). These regularity assumptions do not generally discriminate the Jeffreys prior from the studied set of priors.
In this chapter, different restricted sets of priors will be suggested, they are sets that are though 
to counter the limitations (ii) and (iii) aforementioned. In most cases, they will not include the Jeffreys prior.
%to not include the Jeffreys prior when 



The tackling of limitation (i) mentioned above is not a purpose of this work. We recall that it is actually frequently tackled by a sequential construction of the reference prior as %presented in \cref{chap:intro-ref} (\cref{sec:intro-ref:refpriors}).
suggested by \citet{bernardo_reference_1979}.
On the condition that an ordering of the parameters is set:
 \begin{equation}
     \theta = (\theta_1,\dots,\theta_r) \in \Theta=\Theta_1\times\dots\times\Theta_r,
 \end{equation}
this construction considers a hierarchical construction of the reference prior.
It is already described in \cref{chap:intro-ref} (\cref{sec:intro-ref:refpriors}). We remind below the steps of the sequential construction:
%
%Typically, it is recommended to assume $\Theta_j\subset\RR^{d_j}$ with small dimensions $d_j$ (e.g., lower or equal than $2$) for any $j\in\{1,\dots,r\}$, and to sequentially build a reference prior on the $\Theta_j$,  $j\in\{1,\dots,r\}$:
 \begin{enumerate}
     \item initially fix $\ell_k^1=\ell_k$;
     \item for any values of $\theta_{j+1},\dots,\theta_r\in\Theta_{j+1}\times\dots\times\Theta_r$, compute a reference prior (in the sense of \cref{def:intro-ref:ref-priors})  under the model with likelihood $\theta_j\mapsto\ell_k^j(\mbf y|\theta_j,\dots,\theta_r)$, denote $\pi_j(\cdot|\theta_{j+1},\dots,\theta_r)$ its normalized density;
     \item derive $\ell_k^{j+1}$ such as 
         \begin{equation} %\label{eq:hier:condlikeint}
            \ell_k^{j+1}(\mbf y|\theta_{j+1},\dots,\theta_r) =  \int_{\Theta_j}\ell_k^j(\mbf y|\theta_j,\dots,\theta_r)d\pi_j(\theta_j|\theta_{j+1},\dots,\theta_r).
         \end{equation}
 \end{enumerate}

In this work, the reference priors will be derived only given their formal definition (\cref{def:BA:genref}). Yet, our results can be incorporated in this sequential construction. Indeed,
step 2 of the method depicted above consists of the derivation of a reference prior w.r.t. the variable $\theta_j$.
%in the sense of 
% Thus, our reference priors over constrained setes of priors can be  plainly incorporated into this method.
Additionally, we invite to note that this construction does not solve the limitations (ii) and (iii) previously evoked. Actually, it makes them essential. Indeed, step 2 requires, firstly, a derivation of a reference prior, so that it would lead to a low-dimensional Jeffreys prior if the set of priors is not constrained. Also step 3 necessitates, secondly, that the latter leads to a proper posterior so that the integral involved does not diverge.

We note that this last issue is taken into account by \citet{berger_development_1992} with the suggestion of such construction on an increasing sequence of compact subsets of $\Theta$: $\bigcup_{i\in\NN}\Theta_i=\Theta$. The hierarchical reference prior can then be chosen as a limit of the ones obtained under $\Theta_i$ when $i\to\infty$. However, this limit can be cumbersome to derive in practice. Another solution suggested by \citet{mure_objective_2018} is to restrict the $\sigma$-algebra $\sY$ until the reference prior derived in step 2 leads to a proper posterior. It is still imperfect, as there is no guarantee that such a restricted $\sigma$-algebra exists outside the trivial one.


%In the following subsections, we propose a range of solutions to some of the issues aforementioned, based on the derivation of reference priors over constrained setes of priors. In  Section  {sec:constrainedasympt}, we derive a quasi $D_\delta$-reference prior over setes of priors that are easy to compute, in order to tackle the limitations (ii) and (iii) previously evoked. Then, Section  {sec:constrainedproper} explores another kind of constrained setes of priors, which leads to $D_\delta$-reference priors that can solve the item (iv).












\subsection{A useful definition: quasi-reference priors}\label{sec:BA:defsquas}

%As explained in the introduction, the objective of this work is to study reference priors over restricted sets of priors. Indeed, the reference prior definition is usually considered with very large sets of priors, which are constrained only by some regularity assumptions imposed to the priors (such as continuity, positivity). %This regularity does not generally discriminate the Jeffreys prior
%Actually, 
%The choice of the set of priors $\cP$ in \cref{def:BA:genref} remains open and can be restrained from the large one of priors in $\cM^\varrho_\cC/\!\simeq$. %admitting continuous densities w.r.t. the Lebesgue measure.

While we aim at restricting the set of priors $\cP$ in \cref{def:BA:genref}, we must notice
that such a restriction leaves really unsure the existence of a reference prior. 
Indeed, the definition is itself restrictive, as to admit a reference prior, the set $\cP$ must contain a prior whose restrictions are optimal on any compact subsets of $\Theta$.
In this section, we suggest an extension of the definition of reference priors in the case where in the set $\cP$, the optimal priors on compact subsets of $\Theta$ are not renormalization of each other, but converge to a  prior in $\cP$. 
Such convergence is considered in the sense of the Q-vague convergence \citep{bioche_approximation_2016} on $\sM^\nu_\cC$. % on $\cM^\nu\cC/\!\simeq$. This convergence defines a topology 
The Q-vague convergence of a sequence $(\varPi_n)_n$ to a limit $\varPi$ is equivalent to the convergence of $([\varPi_n])_n$ to $[\varPi]$ in $\sM^\nu_\cC/\!\simeq$ for the quotient topology of the vague convergence on $\sM^\nu_\cC$.




\begin{defi}[Quasi reference prior]\label{defi:quasiRefprior}
    Let $\cP$ be a set of priors. We call $\varPi\in\cP$ a quasi $D$-reference prior if it exists an openly increasing sequence $(\Theta_i)_{i\in \NN}$  of compact sets with $\bigcup_{i\in\NN}\Theta_i=\Theta$ such that
    \begin{itemize}
        \item[(i)] for any $i\in \NN$, there exists a $D$-reference prior $\varPi_i$ over $\cP_i=\{P(\cdot|\Theta_i),\, P\in\cP,\,P(\Theta_i)\in(0,\infty) \}$, %, the set of renormalized restrictions to $\Theta_i$ of priors in $\cP$,
        \item[(ii)] $\varPi$ is the Q-vague limit of the sequence $(\varPi_i)_{i\in \NN}$.
    \end{itemize}
    It is said to be unique  if for any other quasi $D$-reference prior $\varPi'$, $\varPi\simeq\varPi'$.
\end{defi}

Proposition below ensures that this definition properly extends \cref{def:BA:genref} in the case of $\delta$-divergences.
\begin{prop}\label{prop:quasi}
     \begin{itemize}
        \item If $\varPi$ is a $D_\delta$-reference prior over a set $\cP$, then it is a quasi $D_\delta$-reference prior.
        \item If $\cP$ is a set of priors convex and stable by multiplication by indicator functions over measurable sets, then the quasi $D_\delta$-reference prior over $\cP$ is the unique $D_\delta$-reference prior over $\cP$.
        \item If $\cP$ is a convex set of priors and if the sequence of subsets $(\Theta_i)_i$ in \cref{def:BA:genref} is fixed, then the quasi-reference prior over $\cP$ is unique.
    \end{itemize}
\end{prop}


\begin{proof}
    The first statement of the proposition is clear given the definition of a $D_\delta$-reference prior.\\
    For the second, let us adopt the notations of \cref{thm:BA:l(pi)} and  notice that if $\Theta$ is compact and if $\pi^\ast$ is the maximal argument of $l$ over $\cR$, then its renormalized restriction $\pi_1^\ast$ on a compact subset $U$ maximizes $l$ over the set of all renormalized densities $\cR_U$.
    Indeed, if we suppose that $\pi_1\in\cR_U$ maximizes $l$ then, denoting $\pi_0^\ast$ the renormalized restriction of $\pi^\ast$ to $\Theta\setminus U$, $t=\int_U\pi^\ast$, and $\pi = t\pi_1+(1-t)\pi_0^\ast$, $\pi\in\cR$ and
        \begin{equation}
            l(\pi) = t^{\delta+1}l(\pi_1) + (1-t)^{\delta+1}l(\pi_0^\ast) > t^{\delta+1}l(\pi_1^\ast) + (1-t)^{\delta+1}l(\pi_0^\ast) = l(\pi^\ast).
        \end{equation}
    Hence $\pi^\ast$ does not maximize $l$ over $\cR$, which is absurd.\\
    Therefore, in our problem, considering two sequences $(\pi^{(1)}_i)_i$ and $(\pi_i^{(2)})_i$ respectively defined on $(\Theta_i^{(1)})_i$ and $(\Theta^{(2)}_i)_i$, we will get that for any $i$, $\pi^{(1)}_i(\theta)=\pi_i^{(2)}(\theta)$ for all $\theta\in\Theta_i^{(1)}\cap\Theta_i^{(2)}$. Eventually, they are identical on every compact subsets of $\Theta$, and equal to their Q-vague limits which are the same.\\
    Finally, the third statement of the proposition results from the strict concavity of $l$. Indeed, for any $i$, the set $\cR_i$ of renormalized restricted densities on $\Theta_i$ is convex so that the maximal argument of $l$ over $\cR_i$ is unique. Hence the uniqueness of the quasi-reference prior over $\cP$.
\end{proof}
    



\section{Constrained $D_\delta$-reference priors}\label{sec:BA:ress}


In this section, we present results of two different constraint incorporations on the reference priors.
In \cref{sec:BA:res1}, the constraint limits the set of priors to exponentiation of coordinates of $\theta$. The result suggests a simpler reference priors in a case where the improper Jeffreys prior's decay rates are not an issue.
%When the Jeffreys prior has an improper asymptotic behavior that matches with a prior in this set, we show that
In \cref{sec:BA:res2}, we introduce a well-chosen linear constraint on the set of priors that ensures the constraint reference prior is proper, or leads to proper posteriors.

%Our first strategy provides a simple,  tractable solution for constraining reference priors when the improper aspect is not questioned. The second, by contrast, introduces constraints that lead to  reference priors that are proper, or lead to proper posteriors. 


\subsection{Constrained $D_\delta$-reference priors based on Jeffreys' asymptotic decay rates}\label{sec:BA:res1}



    % \subsection{Results}

    


    In this section, we tackle the computational cost of the reference prior. As mentioned in \cref{sec:BA:mots}, the Jeffreys prior expression is often complex to derive even in low dimensional models. This is even more a problem in practical studies where the prior must be evaluated a numerous number of times, when it resorts to MCMC simulations to provide posterior samples of $\theta$ for instance.

    In some works of the literature, the Jeffreys prior is replaced by its decay rates at the boundary of the domain. For instance, the reference prior for Gaussian processes suggested by \citet{gu_jointly_2019} is built on the basis of the decay rates of a Jeffreys prior sequentially computed on the different variables that compose $\theta$ (following the construction presented in \cref{sec:BA:mots}).
    Their idea is that, in particular when it is improper, the prior provides the most information from its asymptotic rates, and variations of them are noticed to have a strong influence on the posterior distribution.
    The result that follows provides a formalization of this intuition, focusing on the case where the Jeffreys prior asymptotically behaves like exponentiation of coordinates of $\theta$.
    
    
    
    
    
    \begin{thm}\label{thm:Jthetaa}
        Suppose $\Theta\subset\RR$ is an interval of the form $[c,b)$ (or $(b,c]$). Call $M:\pi\in\sR_{\cC^b}\mapsto(B\mapsto\int_B \pi d\nu)\in\sM^\nu_\cC$. %, with $J$ integrable and non-null in the neighborhood of $c$.
        \begin{itemize}
            \item If $b\in\RR$ and $J(\theta)\equi{\theta\rightarrow b}C|\theta-b|^a$ for constants $C\in\RR$ and $a\leq-1$, then $M(\pi^\ast)$ where $\pi^\ast(\theta)\propto|\theta-b|^a$ is the unique quasi $D_\delta$-reference prior over $M(\hat\cR)$ where $\hat\cR = \{\pi(\theta)\propto|\theta-b|^u,\,u\in\RR\}$.
            \item If $|b|=\infty$ and $J(\theta)\equi{\theta\rightarrow b}C\theta^a$ for constants $C\in\RR$ and $a\geq-1$, then $M(\pi^\ast)$ where $\pi^\ast(\theta)\propto\theta^a$ is the unique quasi $D_\delta$-reference prior over $M(\hat\cR)$ where $\hat\cR = \{\pi(\theta)\propto\theta^u,\,u\in\RR\}$.
        \end{itemize}
    \end{thm}
    
    \begin{proof}
        The proof is technical and detailed in \cref{sec:BA:proofs}.
        The idea is that $l(\pi)$ can be seen as a negative divergence between $\pi$ and $J$. However, when $J$ is improper at the boundary of the domain, the maximization of $l(\pi)$ gets closer to the minimization of a divergence between $\pi$ and the improper decay rate of $J$.
    \end{proof}
    
    
    \begin{rem}
        \Cref{thm:Jthetaa} still stands when $\Theta=(b,c)$ (or $(c,b)$) if $c\ne\infty$ and if $J(\theta)$ admits a non-null and finite limit when $\theta\to b$.
    \end{rem}
    
    
    
    
    This theorem serves the statement of two conclusions: (i) it emphasizes that when Jeffreys prior is improper, its improper decay rates contain the most relevant information, and (ii) it proposes to choose this asymptotic expansion of Jeffreys as a quasi $D_\delta$-reference prior when we look for an easy prior to compute.
    
    
    






\subsection{Properly constrained $D_\delta$-reference priors}\label{sec:BA:res2}



In \cref{sec:BA:res1}, we have provided some elements to construct a tractable reference prior on the coordinates over which Jeffreys prior is improper.
The reference prior proposed by our theorem
%that our theorem proposes 
keeps the improper characteristic of Jeffreys prior on the same coordinates.
This improper aspect can, however, remain an issue in some cases, especially when the resulting posterior is improper as well.

For this reason, it might happen that some asymptotic rates in some directions still have to be tackled. The work in this section is concluded by results that allow defining a $D_\delta$-reference prior (or quasi $D_\delta$-reference prior), which benefits from adjusted decay rates from Jeffreys prior. 
The proposition below constitutes a preliminary result that gives the form of a $D_\delta$-reference prior over a set of priors with linear constraints.

\begin{assu}\label{assu:glibre}
    A family of functions from $\Theta$ to $\RR$ $(g_j)_{j=1}^p$ is said to satisfy \cref{assu:glibre} if $g_0,\dots,g_p$ are linearly independent in the space of a.e. continuous functions from $\Theta$ to $\RR$, where $g_0:=\theta\mapsto 1$.
\end{assu}

\begin{prop}\label{prop:constraints}
    Suppose $\Theta$ to be a compact subset of $\RR^d$. Let $g_1,\dots,g_p$ be %a.e. continuous 
    functions in $\sR_{\cC^b}$ %from $\Theta$ to $\RR$ 
    that satisfy \cref{assu:glibre}. Define $\tilde\cP$ the set of priors $\varPi$ on $\Theta$ such that $\forall 1,\dots,p$, $\int_\Theta g_jd\varPi=c_j$, for some $c_j\in\RR$.
    If %$\tilde\cP$ is not empty, then 
    there exists a $D_\delta$-reference prior over $\tilde\cP$, it is unique. If it is positive, its density $\pi$ verifies
    \begin{equation}
        \pi(\theta) = J(\theta)\left(\lambda_0+\sum_{j=1}^p\lambda_jg_j(\theta) \right)^{1/\delta},
    \end{equation}
    for some $\lambda_j\in\RR$. Reciprocally, if there exists a prior %$\pi^\ast\in\tilde\cP$ 
    whose density verifies the above equation for some $\lambda_j\in\RR$, it is the $D_\delta$-reference prior over $\tilde\cP$.
\end{prop}

\begin{proof}
    This proposition results from a Lagrange multipliers theorem. A detailed proof is proposed in \cref{sec:BA:proofs}.
\end{proof}



\begin{rem}
    While it is not the subject of this work, we let the reader notice that this proposition opens the way to the introduction of constraints based on expert judgments in prior elicitation. They can take the form of moment constraints or predictive constraints. %\citep{bousquet_contributions_2024}.    
\end{rem}

\begin{rem}\label{rem:klconst}
    The expression of the reference prior given by \cref{prop:constraints} depends on the chosen $\delta$-divergence.
    While this work considers only the framework of reference priors under $\delta$-divergences as a dissimilarity measure, a version of this theorem could be written in the original framework of the reference prior theory that uses the Kullback-Leibler divergence. The expression of the resulting reference prior would be impacted. 
    In the appendix we prove that the expression using the Kullback-Leibler divergence would take the form:
    %In \cite{bernardo_bayesian_1994}, the authors suggest that the expression should take the form of
        \begin{equation}
            \pi^\ast\propto J\cdot\exp\left(\sum_{j=1}^p\lambda_j g_j\right),
        \end{equation}
        for some $\lambda_j$ that remain to be determined.
    This expression was already intuited by \citet{bernardo_bayesian_1994}.
        %Their suggestion is supported by the derivations made in \cite[\S C.3, Theorem 2]{GauchyPhD}.
\end{rem}
 



Below, given a function $g$ that is selected to adjust the asymptotics of Jeffreys prior, is stated the expression of a proper $D_\delta$-reference prior.

\begin{thm}\label{thm:lintoproper}
    Let $g:\Theta\to(0,\infty)$ be a function in $\sR_{\cC^b}$ such that
        \begin{equation}\label{eq:intgalphafinite}
            \int_\Theta J(\theta)g^{1/\delta}(\theta)d\theta<\infty \quad\text{and}
            \quad\int_\Theta J(\theta)g^{1/\delta+1}(\theta)d\theta<\infty,
        \end{equation}
    and suppose that $g$ is bounded in the neighborhood of $b$ for an element $b\in\partial\Theta$. %such that $J$ is improper in the neighborhood of $b$.
    We denote by $\overline\cP$ the set of positive priors $\varPi$ on $\Theta$ such that $\int_\Theta gd\varPi<\infty$, and we define 
    $\varPi\in\overline\cP$ as the prior whose density $\pi$ verifies % follows 
        \begin{equation}
            \pi(\theta)\propto J(\theta)g(\theta)^{1/\delta}.
        \end{equation}
    If $Jg$ is non-integrable in the neighborhood of $b$, then $\varPi$ is a $D_\delta$-reference prior over $\overline\cP$. Otherwise, and if $J$ is improper in the neighborhood of $b$, $\varPi$ is a $D_\delta$-reference prior over the set of proper priors in $\overline\cP$.
\end{thm}

\begin{proof}
    The statement of this theorem results from the sequential use of \cref{prop:constraints} on an increasing sequence of compact subsets of $\Theta$. A detailed proof is written in \cref{sec:BA:proofs}.     
\end{proof}


%\begin{rem}
   To improve the above theorem, one would like 
     to relax the first assumption in \cref{eq:intgalphafinite}, i.e., to let $\int_\Theta J g^{1/\delta}$  be infinite. Indeed, in this way, the result would provide a reference prior $\varPi$ ---non-necessarily proper--- but such that $\pi g\in L^1$, where $\pi$ is a density of $\varPi$. With a good choice of $g$, $\pi^\ast$ could be built as a prior that provides a proper posterior. It is the purpose of the next theorem. 
    The cost of this relaxation is the provision of a quasi-reference prior instead of a reference prior.
    %However, our proof needs this assumption, but one can feel that such (quasi)-reference prior is not far to exist as expressed in the proposition  below.
%\end{rem}

\begin{thm}\label{thm:quasipostpropre}
    Let $g:\Theta\to(0,\infty)$ be in $\sR_{\cC^b}$ such that
    \begin{equation}
        \int_\Theta J(\theta)g(\theta)d\theta=\infty\quad\text{and}\quad \int_\Theta J(\theta)g^{1/\delta+1}(\theta)d\theta<\infty,
    \end{equation}
and suppose that $g(\theta)\conv{\theta\rightarrow b}0$ for an element $b\in\partial\Theta$ such that $J$ is non-integrable in the neighborhood of $b$.\\
Let $(\Theta_i)_{i\in\NN}$ be an openly increasing sequence of compact sets that covers $\Theta$ and $(c_i)_i$ be a bounded sequence in $(0,\infty)$. Define the set of priors  $\overline\cP'=\{\varPi,\,\forall i,\,\int_{\Theta_i} gd\varPi=c_i\int_{\Theta_i}d\varPi\}$.\\ %, where $(c_i)_i$ is a bounded sequence in $(0,\infty)$.\\
Denote for any $i$ $\overline{\cP}'_i$ the set of renormalized restrictions to $\Theta_i$ of priors in $\overline{\cP}'$. If for any $i$ there exists a  positive maximum of $l$ over $\overline\cP'_i$, then  $\varPi$ whose density is denoted $\pi$ is a quasi $D_\delta$-reference prior over $\overline\cP'$ with
    \begin{equation}
        \pi(\theta)\propto J(\theta)g(\theta)^{1/\delta}.
    \end{equation}
    This prior is such that $\int_\Theta\pi gd\nu<\infty$.
\end{thm}




\section{Discussion}\label{sec:BA:discu}

The knowledge of Jeffreys prior's decay rates is central in the results presented in this work. These results indicate that in common scenarios where Jeffreys prior is improper, these rates must be explicitly considered in order to construct a reference prior. 

We let the reader note that \cref{thm:lintoproper,thm:quasipostpropre} introduce results that also depend on the chosen dissimilarity measure. Therefore, a balance must be found between the subjective influence of the constraint and the quest for an informed prior to facilitate possible sampling from the posterior. This is illustrated in the example we address in the following section.
However, it is important to observe that using the KL-divergence instead of a $\delta$-divergence %considered in our work 
would result in a stronger influence of the constraint on the final prior. As noted in \cref{rem:klconst}, the exponentialization of the function $g$ could lead to a prior with distribution tails that are significantly negligible beyond those of Jeffreys, thereby jeopardizing its objective nature.

Generally, the results we propose in \cref{sec:BA:res1,sec:BA:res2} address different problems and are thus  fundamentally different in nature. In one case, the improper aspect of Jeffreys prior is not necessarily challenged, and an efficient construction of the latter is proposed. In the other case, the goal is to significantly attenuate its improper aspect while maintaining as much objectivity as possible. In this latter case, however, the expressions of the proposed  reference priors still depend on the expression of Jeffreys prior. Nevertheless, when Jeffreys prior is proper, there is no guarantee that a straightforward construction inspired by its convergence rates at the domain boundaries will be relevant. %ensure its reference nature in a simple and interpretable way. 
Indeed, although improper tails concentrate an infinite mass that constitutes all the information at the boundaries, when they are proper, the information of interest may need to be sought elsewhere. In this case, a calculation or approximation of the `proper' Jeffreys prior remains to be considered.

Finally, regarding \cref{thm:Jthetaa}, although the result is limited to parameter power distribution tails, it is observed that, in practice, these include a wide range of improper Jeffreys priors.
For example, this includes Jeffreys priors derived from various Gaussian models, such as those introduced by \citet{neyman_consistent_1948}; Jeffreys priors related to specific parameters within Gaussian process models \citep{gu_parallel_2016}; and those arising in more specialized contexts, like the one %
that we develop in the \cref{part:spra} of this manuscript.
%in \cite{VanBiesbroeck2023}.
%
%
%are concerned the Jeffreys prior in the normal 
%\textcolor{red}{Il faudrait en lister plusieurs.}
Moreover, the invariance of Jeffreys priors under re-parameterization can sometimes allow us to return to this case. Specifically, if $J$ can be asymptotically written as a power of a function $f$, where $f$ is differentiable, monotone and with bounded derivative (from above and from below), then the re-parameterization $\vartheta=f(\theta)$ %with a well  
should allow us to recover the reference prior among those expressible as powers of $f$.

In the following section, we illustrate an application of our work with an example taken from the literature.


\section{An example}\label{sec:BA:exa}


In their work, \citet{rubio_inference_2014} prove that the two piece location-scale model they proposed has an improper Jeffreys prior, which issues an improper posterior.
The model is parameterized by $\theta=(\mu,\sigma_1,\sigma_2)\in\RR\times(0,\infty)^2$, inferred over observations in $\cY=\RR$. It has the following likelihood:
    \begin{equation}\label{eq:examplelikilihood}
        \ell(y|\theta) =  \frac{2}{\sigma_1+\sigma_2}\left[f\left(\frac{y-\mu}{\sigma_1}\right)\indic_{(-\infty,\mu)}(y) + f\left(\frac{y-\mu}{\sigma_2}\right)\indic_{(\mu,\infty)}(y)  \right],
    \end{equation}
where $f$ is a density function with support on $\RR$, assumed to be symmetric with a single mode at zero, and with a few integrability assumptions that are detailed in \cite{rubio_inference_2014}.
    The choice of $f$ is open and we can take, % let the assumptions of Theorem  {thm:l(pi)} to be verified.
    %\textcolor{red}{
    %We may take, 
    for instance, the standard Gaussian density function.
    Under this construction, the Fisher information matrix of this model takes the form:
    \begin{equation}\label{eq:fishermatrix}
        \cI(\theta) = \left(\begin{array}{ccc}
             \frac{\alpha_1}{\sigma_1\sigma_2}& -\frac{2\alpha_3}{\sigma_1(\sigma_1+\sigma_2)} & \frac{2\alpha_3}{\sigma_2(\sigma_1+\sigma_2)}  \\
             \ast &\frac{\alpha_2}{\sigma_1(\sigma_1+\sigma_2)} + \frac{\sigma_2}{\sigma_1(\sigma_1+\sigma_2)^2} & - \frac{1}{(\sigma_1+\sigma_2)^2}  \\
             \ast&\ast& \frac{\alpha_2}{\sigma_2(\sigma_1+\sigma_2)} + \frac{\sigma_1}{\sigma_2(\sigma_1+\sigma_2)^2}
        \end{array}\right)
    \end{equation}
    for some positive constants $\alpha_1$, $\alpha_2$ and $\alpha_3$.

    The full Jeffreys prior can be computed as 
    \begin{equation}
        J(\theta) \propto \frac{1}{\sigma_1\sigma_2(\sigma_1+\sigma_2)},
    \end{equation}
    it is improper and leads to an improper posterior. In the following, we construct different priors based on the suggestions developed in this chapter. %a method which, on the basis of our suggestions developed in what precedes, to construct a more suitable prior.

    \paragraph{Proper priors based on a moment constraint}
    Considering results in \cref{sec:BA:res2} and the decay rates of the Jeffreys prior above, a simple correction can be done to issue a proper reference prior w.r.t. $\sigma_1,\sigma_2$, which results in a proper posterior.
    We consider $\delta\in(0,1)$; given $\eps\in(0,\frac{1}{1+1/\delta})$ we have 
        \begin{equation}
            %\int J(\theta)(\sigma_1\sigma_2)^{\eps/\delta} d\sigma_1d\sigma_2<\infty\quad\text{and}\quad 
            \int J(\theta)(\sigma_1\sigma_2)^{\eps/\delta+1} d\sigma_1d\sigma_2<\infty,
        \end{equation}
    so that the associate proper $D_\delta$-reference prior density $\pi$, which is such that $\pi(\mu,\sigma_1,\sigma_2)\sigma_1^\eps\sigma_2^\eps$ is integrable w.r.t. $\sigma_1,\sigma_2$, is
        \begin{equation}
            \pi(\mu,\sigma_1,\sigma_2) \propto \frac{(\sigma_1\sigma_2)^{\eps/\delta -1}}{\sigma_1+\sigma_2}.        
        \end{equation}


\paragraph{Overview and sensitivity on the parameters}
Our prior densities are compared with the Jeffreys prior one in \cref{fig:priorpost}.(a), w.r.t. the parameter $\sigma_2$ the others being fixed to $1$. For this comparison, a multiplicative constant had to be chosen on $J$, we have chosen the one such that $J(1,1,1)=2$.
Our priors differ as a function of $\gamma=\eps/\delta\in(0,\frac{1}{1+\delta})\subset(0,1)$. On the one hand, when $\gamma$ becomes close to $0$, the prior ---which we denote by $\pi_\gamma$ from now on--- becomes close to the Jeffreys prior, i.e., the most objective prior w.r.t. the mutual information criterion. However, in this case, $\pi_\gamma$ becomes close to an improper prior, and its posterior becomes close to an improper posterior. On the other hand, setting $\gamma$ away from $0$ 
rearranges the quantity of information in the prior. Its referential nature decreases in favor of an increase in its entropy.
%make the prior more informative, and probably more subjective. 
Therefore, a trade-off has to be made between suitability for inference and objectivity.
Finally, note that in \cref{fig:priorpost}.(a) is also drawn the \emph{independent Jeffreys prior} ($\text{IJ}$) proposed by the authors in \cite{rubio_inference_2014} for this model as an alternative to Jeffreys:
        \begin{equation}
            \text{IJ}(\mu,\sigma_1,\sigma_2) \propto \frac{\sqrt{\sigma_1+\alpha_2(\sigma_1+\sigma_1)}\sqrt{\sigma_2+\alpha_2(\sigma_1+\sigma_2)}}{\sqrt{\sigma_1\sigma_2}(\sigma_1+\sigma_2)^2}.
        \end{equation}
Its decay rates are actually the same as the ones of $\pi_\gamma$ when $\gamma=1/2$, and the two priors are hard to distinguish. Note that this latter prior $\pi_{1/2}$ equals the hierarchical reference prior (as described in  \cref{sec:BA:mots}) constructed from the ordering $\pi(\theta)=\pi_1(\sigma_1,\sigma_2|\mu)\pi_2(\mu)$.




To evaluate a bit further our method, we propose a visualization of the posterior sensitivity to the priors, i.e., to $\gamma$. 
Such influence quantification constitutes a critical step of the \emph{Bayesian workflow}, as expressed by   \citet{gelman_bayesian_2020}. Several methods exist  in the literature for this purpose (e.g., \cite{berger_robust_1990,nott_checking_2020}). 
An approach is to compare the variations of an \emph{a posteriori} quantity as a function of the parameter \citep{kallioinen_detecting_2023}. In this example, this methodology is yet limited by the improper aspect of Jeffreys posterior. It cannot be considered for any comparison.
In \cref{fig:priorpost}.(b), (c) and (d) are plotted, for numerous $\gamma$ and several data set sizes $k$, the posterior densities that result from a sample of data and from the prior densities $\pi_\gamma$.
%The influence of the priors can be recognized, especially for tiny values of $k$. 
As expected, the influence of the prior appears for small values of $k$.
Indeed, we can notice that for high values of $\gamma$, the posterior is slightly more shifted to the right and seems to be a little flatter. 
It is remarkable that this observation becomes limited when $k$ increases. When $k=50$, the difference between the posterior densities 
is hard to distinguish. %limited and it is 
%The remarkable result is that all the posterior median densities are very close in this example. 
This indicates that the little losses of objectivity should induce small variations in the resulting inference in this example.

For practical details, the chosen $f$ is the standard Gaussian density, and the different data sets have been generated according to the likelihood in \cref{eq:examplelikilihood} conditionally to $\theta^\ast=(2,2,2)$. %They were of size $k=50$, and a number $N=1000$ of them have been generated for the computation of the median densities.



%%However, they are difficult to implement in this example as 
%An approach is to compare the posterior-prior divergences \citep{Nott2020}. %If evaluated and averaged for different data-set, this method amount to compare mutual information evaluations. We propose it in Figure  to illustrate the loss of `objectivity' as a function of $\gamma$.


% In Figure  another approach is considered: we plot the divergences of posteriors issued by $\pi^\ast_\gamma$ to the improper posterior issued by Jeffreys \citep{Kallioinen2023}. Those divergences are computed using $\delta$-divergence, and considering the improper density $p^J(\dot|\mbf y)$ issued by Jeffreys:
%     \begin{equation}
%         p^J(\theta|\mbf y)=J(\theta)\prod_{i=1}^k\ell(y_i|\theta),
%     \end{equation}
% with $J(\theta)$ being such that $J(1,1,1)=1$.



    
    
% \paragraph{Hierachical prior}

%    In accordance with the criticize of \citet{Berger2009} about the derivation of the Jeffreys prior on the full parameter space aforementioned in Sectio
%        \begin{equation}
%            J(\sigma_1,\sigma_2|\mu) \propto \frac{1}{\sqrt{\sigma_1\sigma_2}(\sigma_1+\sigma_2)}
%        \end{equation}
%    which is also improper as
%        \begin{equation}
%            \int_0^\infty\int_0^\infty\frac{1}{\sqrt{\sigma_1\sigma_2}(\sigma_1+\sigma_2)}d\sigma_1d\sigma_2 = \int_0^\infty \frac{2}{\sigma_2}\int_0^\infty \frac{1}{1+\gamma^2}d\gamma d\sigma_2 = +\infty
%        \end{equation}
%     (using the substitution $\gamma=\sqrt{\sigma_1/\sigma_2}$). 
    %More over, this prior also leads to an improper posterior given that
%        \begin{align}
%            \int_0^\infty\frac{2}{\sqrt{\sigma_1\sigma_2}(\sigma_1+\sigma_2)^2}f\left(\frac{y-\mu}{\sigma_2}\right)d\sigma_1 &= \int_0^\infty\frac{4}{\sigma_2^2(\gamma^2+1)^2} f\left(\frac{y-\mu}{\sigma_2}\right) d\gamma \\
%            &= \frac{\pi}{\sigma_2^2}f\left(\frac{y-\mu}{\sigma_2}\right)
%        \end{align}
%    with the quantity above being non-integrable w.r.t. $\sigma_2$ in the neighborhood of $0$.

%    Therefore, the process of the derivation of a hierarchical reference prior cannot be pursued at this point.


% $f(\sigma_1,\sigma_2)=(\gamma,\eta)=(\sigma_1/\sigma_2,\sigma_1+\sigma_2)$, $f^{-1}(\gamma,\eta)=(\gamma\eta)$, $\mathrm{Jac}\,f(\sigma_1,\sigma_2) = 1/\sigma_2,-\sigma_1/\sigma_2^2,1,1$, $\det=\frac{\sigma_1+\sigma_2}{\sigma_2^2}$

% $\int\int \frac{\sigma_1+\sigma_2}{\sigma_2^2(\sigma_1/\sigma_2)^{1/2}(\sigma_1+\sigma_2)^2\sigma_2^{-1}}=\int\int\frac{(\eta/\gamma-1)^{1/2}}{\gamma^{1/2}\eta^2}$


    


\begin{figure}[h]%
    \centering%
    \includegraphics[width=5.7cm]{figures/constrained-priors/priors_.pdf}\hspace*{1cm}%
    \includegraphics[width=5.7cm]{figures/constrained-priors/post5.pdf}\\
    \makebox[13cm][c]{%
    {~\hspace{\stretch{1}}(a)\hspace{\stretch{2}}(b)\hspace{\stretch{1}}~}}\\[5pt]
    \includegraphics[width=5.7cm]{figures/constrained-priors/post15.pdf}\hspace*{1cm}%
    \includegraphics[width=5.7cm]{figures/constrained-priors/post50.pdf}\\
    \makebox[13cm][c]{%
    {~\hspace{\stretch{1}}(c)\hspace{\stretch{2}}(d)\hspace{\stretch{1}}~}}%
    \caption{In (a), different prior densities $\pi_\gamma$ (in dashed line) w.r.t. $\sigma_2$ along with a Jeffreys prior density (that delimits the blue area) and the \emph{Independent Jeffreys} of \cite{rubio_inference_2014} (that delimits the orange area). In (b), (c) and (d), posterior densities for different values of $\gamma$ are plotted. %One observed sample served the calculation of all the posteriors on one figure. 
    One observed sample was used to calculate all the posteriors in a figure.
    The observed sample has a size of $k=5$ in (b), $k=15$ in (c), and $k=50$ in (d).\label{fig:priorpost}}
\end{figure}








\section{Detailed proofs}\label{sec:BA:proofs}


\subsection{Proof of \cref{thm:Jthetaa}}

To prove this theorem, we consider to simplify the derivations that $b=0$ with $\Theta=(0,1]$. We will show later how to extend the result to the other cases. As the Jeffreys prior can be defined up to a positive multiplicative constant with no incidence on the definition of a $D_\delta$-reference prior, 
we will simplify its decay rate assuming $J(\theta)\equi{\theta\rightarrow 0}\theta^a$.

Let us define the increasing sequence of compact subsets $\Theta$: $\Theta_i=[\theta_i,1]$, $i\geq0$, with $\theta_i\conv{i\rightarrow\infty}0$. We denote by $\psi_i$ and $\tilde\psi_i$ the functions defined as follow:
    \begin{equation}
        \psi_i(u) = -\frac{\int_{\Theta_i}J(\theta)^{-\delta}\theta^{u(1+\delta)}d\theta }{\left(\int_{\Theta_i}\theta^ud\theta \right)^{1+\delta}},\qquad \tilde\psi_i(u) = -\frac{\int_{\Theta_i}\theta^{-a\delta }\theta^{u(1+\delta)}d\theta }{\left(\int_{\Theta_i}\theta^ud\theta \right)^{1+\delta}}.
    \end{equation}
The quantity $\psi_i(u)$ corresponds ---up to a positive constant--- to the parametrization w.r.t. $u\in\RR$ of $l(\pi_i)$; where $\pi(\theta)\propto\theta^u$, $\pi_i$ is re-normalized restriction of $\pi$ to $\Theta_i$  and where $l$ is the function defined in \cref{thm:BA:l(pi)} that we seek to maximize.
Therefore, if we call $u_i\in\argmax_{u\in\RR}\psi_i(u)$ for any $i$, and if the associated sequence of priors $(M(\pi_i))_i$ converge Q-vaguely to a prior $\varPi^\ast$ over $\Theta$, that would prove that $\varPi^\ast$ is a quasi $D_\delta$-reference prior.

\paragraph{The case $\mathbf{a<-1}$}
    Firstly, we assume that $a<-1$. Denote $U=(-\infty, u_m:=\frac{\delta a-1}{1+\delta})$, we want to derive an asymptotic equivalent when $i\to\infty$ of $\psi_i(u)$, uniformly w.r.t. $u\in U$.
    More precisely, we are about to show that for any $\eps>0$ there exists a $i_1$ such that for any $u\in U$ and $i>i_i$, $|\psi_i(u)-\tilde\psi_i(u)|<\eps|\tilde\psi_i(u)|$. %\theta_i^{-\delta(a+1)}$.
    

    Let $\eps>0$, using that $J(\theta)^{-\delta}\equi{\theta\rightarrow0}\theta^{-\delta a}$, there exists $i_0$ such that for any $\theta<\theta_{i_0}$, $|J(\theta)^{-\delta}-\theta^{-\delta a}|<\eps\theta^{-\delta a}$. %for a $\tilde\eps$ that we determine later.
    Now, we consider an $i_1$ such that: % for any $u\in U$:
        \begin{equation}
            \frac{\int_{\Theta_{i_0}}J(\theta)^{-\delta}d\theta }{\int_{\theta_{i_1}/\theta_{i_0}}^1J(\theta_{i_0}\theta)^{-\delta}\theta^{u_m(1+\delta)} d\theta}<\theta_{i_0}\eps/2\quad \text{and}\quad
            \frac{\int_{\Theta_{i_0}}\theta^{-\delta a}d\theta }{\int_{\theta_{i_1}/\theta_{i_0}}^1(\theta_{i_0}\theta)^{-\delta a}\theta^{u_m(1+\delta)} d\theta}<\theta_{i_0}\eps/2.
        %
            % \frac{\int_{\Theta_{i_0}} J(\theta)^{-\delta}\theta^{u(1+\delta)}  d\theta}{\int_{\Theta_{i_1}\setminus\Theta_{i_0} } J(\theta)^{-\delta }\theta^{u(1+\delta)}d\theta }
            % = 
            % \frac{\theta_{i_0}^{-1}\int_{\theta_{i_0}}^1 J(\theta)^{-\delta}(\frac{\theta}{\theta_{i_0}})^{u(1+\delta)}  d\theta}{\int_{\theta_{i_1}/\theta_{i_0}}^1 J(\theta_{i_0}\tilde\theta)^{-\delta }\tilde\theta^{u(1+\delta)} d\tilde\theta }
            % %
            % < \frac{\theta_{i_0}^{-1}\int_{\Theta_{i_0} }J(\theta)^{-\delta} d\theta}{\int_{\theta_{i_1}/\theta_{i_0} }^1 J(\theta_{i_0}\theta)^{-\delta }\theta^{u^m(1+\delta)}d\theta}
        \end{equation}
    Thus, for any $i>i_1$  and any $u\in U$:
        \begin{equation}
            \frac{\int_{\Theta_{i_0}} J(\theta)^{-\delta}\theta^{u(1+\delta)}  d\theta}{\int_{\Theta_{i}\setminus\Theta_{i_0} } J(\theta)^{-\delta }\theta^{u(1+\delta)}d\theta }
            = 
            \frac{\theta_{i_0}^{-1}\int_{\theta_{i_0}}^1 J(\theta)^{-\delta}(\frac{\theta}{\theta_{i_0}})^{u(1+\delta)}  d\theta}{\int_{\theta_{i}/\theta_{i_0}}^1 J(\theta_{i_0}\theta)^{-\delta }\theta^{u(1+\delta)} d\theta } %\\
            %
            < \frac{\theta_{i_0}^{-1}\int_{\Theta_{i_0} }J(\theta)^{-\delta} d\theta}{\int_{\theta_{i_1}/\theta_{i_0} }^1 J(\theta_{i_0}\theta)^{-\delta }\theta^{u_m(1+\delta)}d\theta} < \eps/2
        \end{equation}
    and
        \begin{equation}
            \frac{\int_{\Theta_{i_0}} \theta^{-\delta}\theta^{u(1+\delta)}  d\theta}{\int_{\Theta_{i}\setminus\Theta_{i_0} } \theta^{-\delta }\theta^{u(1+\delta)}d\theta }
            = 
            \frac{\theta_{i_0}^{-1}\int_{\theta_{i_0}}^1 \theta^{-\delta}(\frac{\theta}{\theta_{i_0}})^{u(1+\delta)}  d\theta}{\int_{\theta_{i}/\theta_{i_0}}^1 (\theta_{i_0}\theta)^{-\delta }\theta^{u(1+\delta)} d\theta } %\\
            %
            < \frac{\theta_{i_0}^{-1}\int_{\Theta_{i_0} }\theta^{-\delta} d\theta}{\int_{\theta_{i_1}/\theta_{i_0} }^1 (\theta_{i_0}\theta)^{-\delta }\theta^{u_m(1+\delta)}d\theta} < \eps/2,
        \end{equation}
    so that $|\psi_i(u)-\tilde\psi_i(u)|<\tilde\eps|\tilde\psi_i(u)|$ as expected. %Finally, deriving $\tilde\psi_i(u)$ gives
        %\begin{equation}
         %   |\psi_i(u)-\tilde\psi_i(u)|<\tilde\eps  |u+1|^\delta K\theta_i^{-\delta(a+1)}\frac{1-\theta_i^{-(\delta(u-a)+u+1)}}{(1-\theta_i^{-u-1})^{1+\delta}} <\tilde\eps |u+1|^\delta K'\theta_i^{-\delta(a+1)}
        %\end{equation}        
    %for some constants $K,\,K'>0$, using that $\theta_i^{-u-1}\leq\theta_i^{-u_m-1}<1$ and that$\theta_i^{-(\delta(u-a)+u+1)}\leq\theta_i^{-(\delta(u_m-a)+u_m+1)}\leq1$. Hence the expected result when $\tilde\eps=\eps/K$.

    Now we want to use this asymptotic equivalence to bound the difference $|a-u_i|$, where $u_i$ is defined as a maximal argument of $\psi_i$. The next step is thus to show that such $(u_i)_i$ exists.
    
    There exist $\tilde K$, $\tilde K'$ such that $\tilde K\theta^{-\delta}\leq J(\theta)^{-\delta}\leq \tilde K'\theta^{-\delta}$. 
    Let $i\geq0$, we can write
        \begin{equation}\label{eq:gendarmeJeffreys}
            \tilde K |\tilde\psi_i(u)| \leq|\psi_i(u)| \leq \tilde K'|\tilde\psi_i(u)|
        \end{equation}
    with 
        \begin{equation}
            |\tilde\psi_i(u)| = \theta_i^{-\delta(a+1)}\frac{|u+1|^{1+\delta}}{|\delta(u-a)+u+1|}\frac{1-\theta_i^{-\delta(u-a)-u-1}}{(1-\theta_i^{-u-1})^{1+\delta}}  \conv{u\rightarrow-\infty}+\infty.
        \end{equation}
    That makes $|\psi_i|$ being a coercive and continuous function on $U$, so that it admits minimal arguments in $U$. We denote by $u_i$ one of them: 
        \begin{equation}
            u_i \in \argmax_{u\in U} \psi_i(u).
        \end{equation}

    We recall that, by concavity of $x\mapsto -x^{-\delta}$, we find that $a$ is the only maximal argument of $\tilde\psi_i$ for any $i$. 
    This way, for $i>i_1$, we write
        \begin{align}
            |\tilde\psi_i(u_i)-\tilde\psi_i(a)|&\leq |\psi_i(u_i)-\tilde\psi_i(u_i)| +  |\psi_i(a)-\tilde\psi_i(a)| + |\psi_i(u_i)-\psi_i(a)| \nonumber\\
            \tilde\psi_i(a) - \tilde\psi_i(u_i) &%\leq \eps(|u_i+1|^\delta+ |a+1|^\delta) \theta_i^{-\delta (a+1)} + \psi_i(u_i)-\psi_i(a),
            \leq \eps(|\tilde\psi_i(u_i)|+|\tilde\psi_i(a)|)+ \psi_i(u_i)-\psi_i(a),
        \end{align}
    which leads to 
        \begin{align}
            2(\tilde\psi_i(a)-\tilde\psi_i(u_i)) &%\leq  \eps(|u_i+1|^\delta+ |a+1|^\delta) \theta_i^{-\delta (a+1)} + \psi_i(u_i)-\tilde\psi_i(u_i)+\tilde\psi_i(a)-\psi_i(a) \nonumber\\
            \leq \eps(|\tilde\psi_i(u_i)|+|\tilde\psi_i(a)|) +  \psi_i(u_i)-\tilde\psi_i(u_i)+\tilde\psi_i(a)-\psi_i(a) \nonumber\\
            \tilde\psi_i(a) - \tilde\psi_i(u_i) &%\leq \eps (|u_i+1|^\delta+ |a+1|^\delta) \theta_i^{-\delta (a+1)}.
            \leq \eps(|\tilde\psi_i(u_i)|+|\tilde\psi_i(a)|).
        \end{align}
    %The above work allows to conclude that
    %Within the last inequality above, 
    Consequently to the convergence of
     $(\theta_i^{\delta(a+1)}\tilde\psi_i(a))_i$ toward a positive limit when $i\to\infty$, we deduce that $\theta_i^{\delta(a+1)}(\tilde\psi_i(a)-\tilde\psi_i(u_i))/|\theta_i^{\delta(a+1)}\tilde\psi_i(u_i)|$ is asymptotically null. This prevents the sequence $(\theta_i^{\delta(a+1)}\tilde\psi_i(u_i))_i$ to admit a non finite subsequential limit, meaning it has to be bounded and to converges to the same limit as $(\theta_i^{\delta(a+1)}\tilde\psi_i(a))_i$, i.e. $-|a+1|^\delta$.

     On another hand, we notice that for any $M>0$, there exist a $M'$ such that for any $u<M'$, 
        \begin{equation}
            \frac{|u+1|^{1+\delta}}{|\delta(u-a)+u+1|}>M
        \end{equation}
     and $|\theta^{\delta(a+1)}\tilde\psi_i(u)|>M$ for any $i\geq0$. Thus, as $(\theta^{\delta(a+1)}\tilde\psi_i(u_i))_i$ has been proven to be bounded, so must be $(u_i)_i$.

     To conclude on that sequence, we denote by $\rho$ a finite subsequential limit of $(u_i)_i$, if $\rho\ne u_m$ then deriving the limit of $\theta^{\delta(a+1)}\tilde\psi_i(u_i)$ leads to 
        \begin{align}
           & -\frac{|\rho+1|^{1+\delta}}{\delta(\rho-a)+\rho+1} = |a+1|^\delta \nonumber\\
           \text{i.e.}\quad & -|\rho+1|(|\rho+1|^\delta-|a+1|^\delta) = |a+1|^\delta \delta(\rho-a);
        \end{align}
    necessarily, $\rho=a$.
    It remains to prove that $\rho=u_m$ is absurd. Indeed, in this case the integrals $\int_{\Theta_i}\theta^{-\delta a+u_i(1+\delta)}d\theta$ converge either to $0$, either to $+\infty$. Therefore, that would make $(\theta_i^{\delta(a+1)}\tilde\psi_i(u_i))_i$ converging either to $-\infty$, either to $0$, which in both case is different to $-|a+1|^\delta$.


    %Eventually, $(u_i)_i$ converges to $a$.

    Let us now work beyond the subset $U$ of $\RR$. First, if $u\in(-1,+\infty)$, the integrals that compose $\psi_i(u)$ both admit finite and positive limits when $i\to\infty$. The limit of $(|\psi_i(u)|)_i$ is moreover bounded from below as a consequence of \cref{eq:gendarmeJeffreys}:
        \begin{equation}\label{eq:minorationpsiu1infty}
            |\psi_i(u)|\geq \tilde K\frac{(u+1)^{1+\delta}}{ \delta(u-a)+u+1}\frac{1-\theta_i^{\delta(u-a)+u+1}}{(1-\theta_i^{u+1})^{1+\delta}} \geq \tilde K'|\log\theta_i|^{-1-\delta}.  %{(\mu+1)^{\delta}}. %\conv{i\rightarrow\infty}\infty
        \end{equation}
    %and the same way, $\psi_i(-1)\geq\tilde K\frac{(\log\theta_i)^{-1-\delta}}{|\delta(1+a)|}$
    Thus, there exists $i_2\geq0$ such that for any $i>i_2$ $|\psi_i(a)|<\tilde K'|\log\theta_i|^{-1-\delta}$, consequently to $\psi_i(a)\equi{i\rightarrow\infty}|a+1|^{\delta}\theta_i^{-\delta(a+1)}\aseq{i\rightarrow\infty}o(|\log\theta_i|^{-\delta-1})$.
    As a result, for any $i>i_2$:
        \begin{equation}
            \sup_{u\in(-1,+\infty)}\psi_i(u)<\psi_i(a)\leq\psi_i(u_i).
        \end{equation}
    Finally, if $u\in(u_m,-1)$, analogously than in \cref{eq:minorationpsiu1infty}, we can write
        \begin{equation}
            |\psi_i(u)|\geq\tilde K(u+1)^{1+\delta}\frac{|\log\theta_i|}{(1-\theta_i^{u+1})^{1+\delta}} \geq \tilde K''\theta_i^{-(u_m+1)(1+\delta)}|\log\theta_i|^{-\delta}.
        \end{equation}
    Once again, we have $\psi_i(a)\aseq{i\rightarrow\infty}o(\theta_i^{-(u_m+1)(1+\delta)}|\log\theta_i|^{-\delta})$ and we can consider $i_3\geq 0$ such that for any $i>i_3$:
        \begin{equation}
            \sup_{u\in(u_m,-1)}\psi_i(u)<\psi_i(a)\leq\psi_i(u_i).
        \end{equation}

    All the work that precedes proves that any sequence $(v_i)_i$ defined by $v_i\in\argmax_{\RR}\psi_i$ converges to $a$.
    

\paragraph{The case $\mathbf{a=-1}$}
In this case, we easily get that $\psi_i(a)\equi{i\rightarrow\infty}-|\log\theta_i|^{-\delta}$. 
When $u<\eta<a$, 
    \begin{equation}
        |\psi_i(u)|\geq\tilde K\frac{|u+1|^{\delta}}{\delta+1} \frac{1-\theta_i^{-(\delta+1)(u+1)}}{(1-\theta_i^{-u-1})^{1+\delta}}\geq \hat K {|\eta+1|^{\delta}}(1-\theta_i^{-(\eta-1)({1+\delta})})
    \end{equation}
and when $u>\tilde\eta>a$,
    \begin{equation}
        |\psi_i(u)|\geq\tilde K\frac{|u+1|^{\delta}}{\delta+1} \frac{1-\theta_i^{(\delta+1)(u+1)}}{(1-\theta_i^{u+1})^{1+\delta}}\geq \hat K' {|\tilde\eta+1|^{\delta}}(1-\theta_i^{(\tilde\eta-1)({1+\delta})}).
    \end{equation}


Thus, for any $\eps>0$, the equations above with $\eta=a-\eps$ and $\tilde\eta=a+\eps$ let state that there exists an $i_1\geq0$ such that for any $i>i_1$:
    \begin{equation}
        \sup_{u\in(-\infty,a-\eps)}\psi_i(u)<\psi_i(a)\quad\text{and} \quad\sup_{u\in(a+\eps,\infty)}\psi_i(u)<\psi_i(a)
    \end{equation}
so that $\argmax_{\RR}\psi_i\subset(a-\eps,a+\eps)$, which let the definition of a sequence $(u_i)_i$ of maximal arguments of $\psi_i$ which converges to $a$.


\paragraph{Q-vague convergence}
The conclusion concerning the Q-vague convergence of the $M(\pi_i^\ast)$ defined from the $u_i$ constructed in the work above: $\pi_i^\ast(\theta)\propto\theta^{u_i}$ is a direct result of \cite[Proposition 2.16]{bioche_approximation_2016}. Indeed, the convergence of sequence $(u_i)_i$ toward $a$, implies that the sequence of our priors converges Q-vaguely to $\varPi^\ast$ such that $\varPi^\ast = M(\pi^\ast)$ with $\pi^\ast(\theta)\propto{\theta^a}$.



\paragraph{Extension to other set $\Theta$}
We shall now demonstrate that the proven result extends itself to the general case: $\Theta=[c,b)$ or $(b,c]$, $b\in\RR\cup\{-\infty,\infty\}$.

We first consider $\Theta=(b,c]$ with $b\in\RR$.
We denote by $\cQ$ the set of priors densities after the substitution $\vartheta = (\theta-b)/(c-b)\in T=(0,1)$:  $\cQ=\{\tilde\pi(\vartheta) = (b-c)\pi(\vartheta (b-c)+b),\,\pi\in\hat\cR\}$.
Thus, $\cQ = \{\pi(\theta)\propto\vartheta^u,\,u\in\RR\}$.
We define the increasing sequence of compact sets $(\Theta_i)_i$ by $\Theta_i = [b+t_i,c]$ with $t_i\conv{i\rightarrow}0$. Therefore, for $\pi\in\hat\cP$, calling $\pi_i$ the renormalized restriction of $\pi$ to $\Theta_i$ gives
    \begin{equation}
        l(\pi_i) = \tilde l(\tilde\pi_i) = C_\delta\int_{T_i}\tilde\pi_i(\vartheta)^{1+\delta}\tilde J(\vartheta)^{-\delta}d\vartheta
    \end{equation}
with $T_i=[t_i,1]$, $\tilde\pi_i(\vartheta) = (b-c)\pi_i(\vartheta (b-c)+b)\in\cQ$ and $\tilde J(\vartheta) = (b-c)J(\vartheta (b-c)+b)\equi{\vartheta\rightarrow0}\vartheta^a$.
Thus, the work done above states that the family of maximal arguments $\tilde\pi_i^\ast$ of $\tilde l$ over $\cQ_i$  provides a sequence of priors $(M(\tilde\pi_i^\ast))_i$ that converges Q-vaguely toward $M(\tilde\pi^\ast)$ where $\tilde\pi^\ast(\vartheta)\propto\vartheta^a$.
Thus, the associate densities $\pi^\ast_i$ maximize $l$ over $\cP_i$ and issues a sequence of priors $(M(\pi^\ast_i))_i$ that converges Q-vaguely toward $M(\pi^\ast)$ where $\pi^\ast(\theta)\propto|\theta-b|^a$.

To treat the other cases, other substitutions with analogous work permit to conclude:
(i) the substitution $\vartheta=(b-\theta)/(b-c)$
when $\Theta=(c,b)$, $b\in\RR$; (ii) the substitution $\vartheta=1/(|\theta-c|+1)$ when 
$\Theta=[c,\infty)$ or $(-\infty,c]$.








\subsection{Proof of \cref{prop:constraints}}

Let us start by the uniqueness.
%Recall the definition of an openly increasing sequence of compact sets $(\Theta_i)_{i\in\NN}$: there exist $i_0\geq0$ and a sequence $(V_i)_{i\geq i_0}$ of open subsets of $\Theta$  such that for any $i\geq i_0$
%    \begin{equation}
%        \Theta_i\subset V_i\subset \Theta_{i+1}.
%    \end{equation}
%This way, $\bigcup_iV_i=\Theta$ and the compacity of $\Theta$ imposes it to  be a finite union, so that $\Theta_i=\Theta$ for any $i\geq i_1$ for some $i_1\geq0$.\\
%
%
%If $(\Theta_i)_{i\in \NN}$ is an increasing sequence of compact subsets that covers $\Theta$, %%%($i_1<i_2\Longrightarrow\Theta_{i_1}\subset\Theta_{i_2}$, $$), 
%it is stationary, i.e. there exist an $i_0$ such that for any $i>i_0$, $\Theta_i=\Theta$.\\ %%%finite subset $\tilde I\subset I$ such that $\bigcup_{i\in\tilde I}\Theta_i=\Theta$ (or, if $I$ is totally ordored, there exist $i_0$ such that for any $i>i_0$, $\Theta_i=\Theta$).\\
%Indeed, by contradiction assuming that for any $n$, $\bigcup_{i\leq n}\Theta_i\ne\Theta$, its complementary is non-empty and contains a non-empty closed subset $F_n\subset\left(\bigcup_{i\leq n}\Theta_i\right)^c$. Therefore, as $\Theta\subset\RR^d$ it is an Hausdorff compact set so that the two closed sets $F_n$ and $G_n=\bigcup_{i\leq n}\Theta_i$ are subsets of two disjoints open sets $U_n$ and $V_n$: $F_n\subset U_n$, $G_n\subset V_n$. This way, $\bigcup_{n\in\NN}V_n=\Theta$ and by compactness there exists $N$ such that $V_N=\Theta$ and so $F_N=\empty$ which is a contradiction.
%
%Any $D_\delta$-reference prior
A prior $\varPi^\ast\in\cP$ is a $D_\delta$-reference prior if and only if its density $\pi^\ast$ maximizes $l$ over $\cR$, where $M(\cR)=\cP$.
As the set $\cP$ is convex, $\cR$ is convex as well. Also, we notice that the function $\pi\mapsto l(\pi)$ is strictly concave, so that its maximizer is unique in the sense that two maximizer are equal $\nu$-a.e. That ensures the $D_\delta$-reference prior is unique.
%$\pi^\ast$ over a convex   such as $\tilde\cP$ must maximize $l$ as a prior on $\Theta$. The mapping $\pi\mapsto l(\pi)$ being strictly convex when $\pi$ is seen as a function in $\cR^\cC$, such maximal argument is unique.


% Regarding the expression of $\pi^\ast$,it can be seen as a direct consequence of the following lemma, which proven later on.

% \begin{lem}\label{lem:constrainstaec0}
%     Calling $U$ the set of positive and a.e. continuous function from $\Theta$ to $\RR$, and $C=\{f:\Theta\to\RR,\, \int_\Theta f=1,\,\forall j,\,\int_\Theta f_jg_j=c_j\}$, if $\pi^\ast=\argmax_{\pi\in C\cap U}l(\pi)$ exists, it verifies
%         \begin{equation}\label{eq:lemconstepiast}
%             \pi^\ast(\theta)  J(\theta)\left(\lambda_0+\sum_{j=1}^p\lambda_jg_j(\theta)\right)^{1/\delta},
%         \end{equation}
%     for some $\lambda_j\in\RR$. Reciprocally, if there exists a $\pi^\ast\in C\cap U$ that satisfies above equation for some $\lambda_j\in\RR$, then it maximizes $l$ over $C\cap U$.
% \end{lem}




% \paragraph{Proof of Lemma {lem:constrainstaec0}}
Regarding the expression of $\pi^\ast$,
let us call $E$ the space of bounded a.e. continuous functions from $\Theta$ to $\RR$ and we equip $E$ with the supremum norm over $E$: $\|f\|=\sup_\Theta|f|$. The set $\Theta$ being supposed compact, the pair $(E,\|\cdot\|)$ constitutes a Banach vector space whose restriction $U$ composed by the positive functions of $E$ is an open and convex subset.
It is possible to see $l$ as being a concave function defined on $U$.


Let us compute the differentiate of $l$ over $U$. One can write $l=\phi_2\circ\phi_1$ with
\begin{equation}
    \phi_1:\pi\in E\longmapsto \pi^{1+\delta}\in E ;\qquad \phi_2:\pi\in E\longmapsto C_\delta \int_\Theta \pi(\theta)|\cI(\theta)|^{-\delta/2}d\theta.
\end{equation}
As $\phi_2$ is a continuous linear mapping from $E$ to $\RR$, $l$ is differentiate while $\phi_1$ is, with for any $h\in E$:
    \begin{equation}
        dl(\pi) = \phi_2\circ d\phi_1(\pi).
    \end{equation}
Fix $\pi\in U$. For an $\eps>0$ there exists $\tilde\eps>0$ such that while $|x|<\|\pi\|$ and $|u|<\tilde\eps$ then $|(x+u)^{1+\delta}-x^\delta-(1+\delta)x^\delta u|<\eps|u|$. Thus for any $h\in E$ such that $\|h\|<\tilde\eps$, we have
    \begin{equation}
        \|\phi_1(\pi+h) -\phi_1(\pi)-(1+\delta)\pi^\delta h\|<\eps\|h\|.
    \end{equation}
We conclude that $\phi_1$ differentiable on $U$ with $d\phi_1(\pi)h = (1+\delta)\pi^\delta h$, for any $\pi\in U$, $h\in E$. This differentiate is additionally continuous, which makes $l$ continuously differentiable as well.

Thus, considering the additional constraint $\int_\Theta\pi(\theta)d\theta=1$, this problem can be treated applying the Lagrange multipliers theorem (see e.g. \citep{zeidler_applied_2012}) to state that there exist $\lambda_0,\dots,\lambda_p\in\RR^{p+1}$ such that
    \begin{equation}
        dl(\pi^\ast)h - \lambda\int_\Theta h(\theta)d\theta - \sum_{i=1}^p\lambda_i\int_\Theta h(\theta)g_i(\theta)d\theta = 0
    \end{equation}
for any $h\in E$. Eventually, as $dl(\pi)h=C_\delta(1+\delta)\int_\Theta\pi(\theta)^{\delta}|\cI(\theta)|^{-\delta/2}h(\theta)d\theta$, we get
    \begin{align}
        \pi^\ast(\theta) = J(\theta)\bigg(\lambda_0+\sum_{i=1}^p\lambda_ig_i(\theta)\bigg)^{1/\delta}.
    \end{align}
% with $J(\theta$

Moreover,
as $l$ is strictly concave and the constraints are linear, the second order condition for the Lagrangian states the reciprocal aspect of this result: calling $C$ the space of functions satisfying the constraints, if $\pi^\ast\in U\cap C$ satisfies the last equation above, it maximizes $l$.


%To conclude, notice that densities of non-negative priors satisfying the constraints constitute %belongs to 
%the closure of $U\cap C$, $C$ designing the set of the functions that satisfy the constraints. As $\pi^\ast$ disclosed above maximizes $l$ over $U\cap C$, with $l$ being continuous on its closure. Therefore, $\pi^\ast$ maximizes $l$ over all that latter, i.e. its associated prior is the $D_\delta$-reference prior over $\tilde\cP$.

To conclude, the elements in $\tilde\cP$ are associated to densities which belong to $E$ because $\Theta$ is compact.
They are non necessarily everywhere positive, but only supposed to be non-negative.
This set of bounded a.e. continuous and non-negative functions satisfying the constraints is actually the closure of $U\cap C$ over which $l$ is continuous.
Therefore, calling $\tilde\cP^\ast$ the set of positive priors in $\tilde\cP$, we have $\argmax_{M(\pi)\in\tilde\cP}l(\pi)=\argmax_{M(\pi)\in\tilde\cP^\ast}l(\pi)$ which is $\pi^\ast$.








\subsection{Proof of \cref{thm:lintoproper}}


Assuming that it exists, denote by $\pi^\ast$ the density of the $D_\delta$-reference prior $\varPi^\ast$ over $\overline\cP$, denote as well
    \begin{equation}
        c = \int_{\Theta}\pi^\ast(\theta)g(\theta)d\theta,\quad Z_i = \int_{\Theta_i}\pi^\ast(\theta)d\theta
    \end{equation}
for any $i\in \NN$, considering an openly increasing sequence $(\Theta_i)_{i\in\NN}$ of compact sets that cover $\Theta$ over which $1>\varPi^\ast(\Theta_i)>0$ for any $i$ such that $\Theta_i\ne\Theta$ (we show later that they exist).
In particular, $\varPi^\ast$ must be the $D_\delta$-reference prior over the set $\cP^c=\{\varPi\in\sM_{\cC}^\nu,\,\int_\Theta gd\varPi =c\}$. %\subset\cP$.

Let $i\in \NN$, if $\pi_i$ is a density on $\Theta_i$ such that 
\begin{equation}
    \int_{\Theta_i}\pi_ig + \frac{1}{Z_i}\int_{\Theta\setminus\Theta_i}\pi^\ast g = \frac{c}{Z_i},
\end{equation}
then the prior whose density $\pi$ on $\Theta$ defined by $\pi=Z_i\pi_i+\pi^\ast\indic_{\Theta\setminus\Theta_i}$ %as being proportional to $\pi_i$ on $\Theta_i$ and on $\pi^\ast$ on $\Theta\setminus\Theta_i$ 
belongs to $\cP^c$.
Therefore, denoting $\pi^\ast_i$, the renormalized restriction of $\pi^\ast$ to $\Theta_i$, $l(\pi^\ast_i)$ must be larger than $l(\pi_i)$ on $\Theta_i$ by definition of the reference prior.

Thus, $\pi^\ast_i$ is a maximal argument of $l$ under the constraints 
    \begin{equation}
        \int_{\Theta_i}\pi_i = 1,\qquad \int_{\Theta_i}\pi_i g=\frac{c}{Z_i}-\frac{1}{Z_i}\int_{\Theta\setminus\Theta_i}\pi^\ast g.
    \end{equation}
Given the result of \cref{prop:constraints}, $\pi^\ast_i$ takes the form of:
    \begin{equation}
        \pi^\ast_i = J\cdot(\lambda^{(1)}_i+\lambda^{(2)}_ig)^{1/\delta}
    \end{equation}
for some $\lambda_i^{(1)}$ and $\lambda_i^{(2)}$.

Then, denoting as well $\pi^\ast_{i+1}=J\cdot(\lambda^{(1)}_{i+1}+\lambda^{(2)}_{i+1}g)^{1/\delta}$, and reminding that $\pi^\ast_{i+1}\propto\pi^\ast_i$ on $\Theta_i$, we deduce that $\lambda^{(1)}_{i+1}=\lambda^{(1)}_{i}$ as well as $\lambda^{(2)}_{i+1}=\lambda^{(2)}_{i}$.
Eventually, $\pi^\ast\propto J\cdot (\lambda_1+\lambda_2g)^{1/\delta}$.
Thus, in the neighborhood of $b$, $\pi^\ast(\theta)\equi{\theta\rightarrow b}K\lambda_1^{1/\delta}$ for some $K\ne0$ if $\lambda_1$ is non-null, which is discordant with the satisfaction of the constraint $\int_\Theta\pi^\ast g<\infty$ in the case where $\int_\Theta Jg=\infty$ or with the constraint $\int_\Theta\pi^\ast =1$ otherwise.


Finally, $\pi^\ast$ is proportional to $Jg^{1/\delta}$ and the value of $c$ must be fixed by
\begin{equation}
    c = \left(\int_\Theta J\cdot g^{1+1/\delta}\right)\cdot\left(\int_\Theta J\cdot g^{1/\delta}\right)^{-1}.
\end{equation}

To finish the proof, we still have to show that $\varPi^\ast$ is not null on any of the $\Theta_i$, or on any of the $\Theta\setminus\Theta_i$. First, there must exist $i_0$ such that $\varPi^\ast(\Theta_{i_0})>0$, so that $\varPi^\ast(\Theta_i)>0$ for any $i\geq i_0$ and the sequence $(\Theta_i)_{i\geq i_0}$ can be considered instead of the initial one. Second, for any $i$, if $\Theta\setminus\Theta_i$ is non-empty then it has a non-empty interior, as a consequence of the definition of openly increasing sequences of compact sets. Thus, as $\varPi^\ast$ is assumed to be positive, $\varPi^\ast(\Theta\setminus\Theta_i)$ is non-null. Hence the result.



\subsection{Proof of \cref{thm:quasipostpropre}}

  Consider an openly increasing sequence of compact sets $(\Theta_i)_{i\in\NN}$ that covers $\Theta$. Let $i\in\NN$, for $\varPi$ to be a $D_\delta$-reference prior over $\cP_i$, its normalized density $\pi_i$ %---associated to its probability density on $\Theta_i$--- 
  %to be a reference prior over $\cP_i$, it 
  must maximize the function $l$ under the constraints $\int_{\Theta_i}\pi_i=1$ and $\int_{\Theta_i}\pi_i g=c_i$. Thus, according to \cref{prop:constraints}, $\pi_i$ can be written as:
     \begin{equation}
         \pi_i(\theta)\propto J(\theta)(\lambda_i^{(1)}+\lambda_i^{(2)}g(\theta))^{1/\delta},
     \end{equation}
 for some $\lambda_i^{(1)}$ and $\lambda_i^{(2)}$.
 Considering, if required, a subsequence of $(\pi_i)_{i\in\NN}$, we can assume that $\lambda_i^{(1)}$ and $\lambda_i^{(2)}$ have constant signs.
 Also, by convexity we write:%
 \newcommand{\li}[1]{\lambda_i^{(#1)}}%
     \begin{align}\label{eq:convexityConsCpig}
         c_i^\delta  & \geq \int_{\Theta_i}J(\theta)(\li1+\li2g(\theta))g(\theta)d\theta\left(\int_{\Theta_i}Jg\right)^{\delta-1}  \\
         &\geq \li1\left(\int_{\Theta_i}Jg\right)^{\delta} + \li2\frac{\int_{\Theta_i}Jg^{2} }{\left(\int_{\Theta_i}Jg\right)^{1-\delta}},\nonumber
     \end{align}
    and
    \begin{align}\label{eq:convexityConspi1}
        1 & \geq \int_{\Theta_i}J(\theta)(\li1+\li2g(\theta))d\theta\left(\int_{\Theta_i}J\right)^{\delta-1} \\
        &\geq \li1\left(\int_{\Theta_i}J\right)^\delta + \li2 \frac{\int_{\Theta_i}Jg}{\left(\int_{\Theta_i}J\right)^{1-\delta}}  . \nonumber
    \end{align}
We want to identify the possible subsequential limits of $\li1$. 
We assume that one exists in $(0,\infty]$, we can consider if required a subsequence to assume that it is the actual the limit of $\li1$. 
This way, \cref{eq:convexityConspi1} does not allow $\li2$ to be non-negative.
Thus, $\li2$ is negative and by convexity,
 %  \begin{equation}
 %       1\geq \li1\left(\int_{\Theta_i}J\right)^\delta + \li2\frac{\left(\int_{\Theta_i} Jg^{1+1/\delta}\right)^{\delta/(1+\delta)}}{\left(\int_{\Theta_i}J\right)^{1-\delta}} \left(\int_{\Theta_i}J\right)^{1/(1+\delta)}
 %   \end{equation}
 %   or
    \begin{equation}\label{eq:convixity2l2leq0}
        c_i^\delta \geq \li1\left(\int_{\Theta_i} Jg\right)^\delta +\li2 \left(\int_{\Theta_i} Jg^{1+1/\delta}\right)^\delta .
    \end{equation}
%
  %  $$\frac{1}{1+\delta}-1+\delta = \frac{\delta^2}{1+\delta}\in(0,\delta)\qquad$$
 Given that $(c_i)_i$ is bounded, and that $\int_{\Theta_i}Jg\conv{i\rightarrow\infty}\infty$ while $\int_{\Theta_i}Jg^{1+1/\delta}$ has a finite limit, we deduce that $\li2$ diverges to $-\infty$.
 However, to ensure the expression of $\pi_i$ to be well defined, we must have $\li1+\li2g(\theta)\geq0$ for any $i$ and any $\theta$. As $g$ is non-null, it necessary comes that $(\li2/\li1)_i$ is a  bounded sequence.
Therefore, 
%\begin{equation}
%    \li2 \aseq{i\rightarrow\infty} o\left(\li1\left(\int_{\Theta_i}Jg\right)^\delta\right)    
%\end{equation}
$\li2 \aseq{i\rightarrow\infty} o\left(\li1\left(\int_{\Theta_i}Jg\right)^\delta\right) $
and the right hand side in \cref{eq:convixity2l2leq0} diverges to $\infty$, which is a contradiction with $(c_i)_i$ being bounded.

As a consequence, any subsequential limit of $(\li1)_i$ must belong to $[-\infty,0]$. Without changing the notations, we can now assume that $(\li1)_i$ has a limit, which we suppose to be strictly negative firstly, also, $(\li1)_i$ can then be assumed to be always negative.
 Thus, still to ensure that $\li1+\li2g(\theta)\geq0$ for any $i$, $\theta$, and reminding that $g(\theta)\conv{\theta\rightarrow b}0$ we deduce $\frac{\li2}{|\li1|}\conv{i\rightarrow\infty}\infty$.
 Then, 
 the sequence of functions $\left(\pi_i/\li2\right)_i$ converges pointwisely to $\pi^\ast$, defined by $\pi^\ast(\theta)\propto J(\theta)g(\theta)^{1/\delta}$ with  $|\frac{1}{\li2}\pi_i(\theta)|\leq C+g(\theta)$ for some $C$, with $g$ being bounded on all compact sets. %\footnote{If $g$ is not continuous but belongs to $\cR^\cC$, it is bounded on every compact. Then, \cite[Proposition 2.16]{Bioche2016} allows to conclude.}.
According to \cite[Proposition 2.15]{bioche_approximation_2016}, what precedes implies that the sequence of priors $(M(\pi_i))_i$ converges Q-vaguely to $M(\pi^\ast)$.

Eventually, if $(\li1)_i$ admits $0$ as a subsequential limit, 
we can assume that it is its actual limit.
Consequently,  the sequence of functions $(\pi_i/\li2)_i$ converges pointwisely to the density $\pi^\ast$ while being bounded  by $g$. Therefore, the sequence of priors $(M(\pi_i))_i$ converges Q-vaguely to $M(\pi^\ast)$.
%
In any cases, $M(\pi^\ast)$ is a quasi-reference prior over $\overline\cP'$.

 
 
 
 











\section{Conclusion}\label{sec:BA:conclusion}



Prior elicitation remains an open topic in Bayesian analysis, with no solution satisfying all criteria (objectivity, computational feasibility, property, ...) simultaneously. 
In this work, we have focused on the objectivity and exploitability of priors.
Specifically, we have provided some solutions for practitioners seeking non-subjective priors, addressing the practical challenges of traditional reference priors, particularly in terms of complexity and proper aspect. Through the example we presented, we demonstrated that simple criteria can lead to the construction of reference priors that are either straightforward to formulate or proper, depending on the case.

Moreover, our results show that one cannot completely dispense with a consideration of the asymptotic properties of Jeffreys prior. This emphasizes the importance of studying this prior for a complete construction of objective priors.
Indeed, although one of our results  offers a way to define in a simpler way the reference prior when it has improper decay rates, Jeffreys' rate still have to be elucidated. Our other result requires to derive the whole Jeffreys prior.

The approximation of the latter thus becomes a problem of a prior importance. Of course, one can thoroughly focus its explicit expression, yet it can be cumbersome to derive. In \cref{chap:varp} we propose a numerical solution for the approximation of the Jeffreys prior.

















\chapter{Variational approximation of generalized reference priors using neural networks}\label{chap:varp}



\begin{abstract}[\hspace*{-10pt}]
    This chapter develops the work
    done by Nils Baillie, during his end-of-study internship that I supervised at CEA Saclay.
    It draws mainly from the submitted work: \fullcite{baillie_variational_2025}
\end{abstract}

\begin{abstract}
    abstract
\end{abstract}

\minitoc


\section{Introduction}

intro 

\section{Variational approximation of the reference prior}






\part{Seismic fragility curve estimation}\label{part:spra}


\chapter{Seismic fragility curves in probabilistic risk assessment studies}\label{chap:frags-intro}



\begin{abstract}
    Seismic fragility curves are key quantities of interest in the seismic probabilistic risk assessment framework because they efficiently describe the fragility of a mechanical system of interest under a seismic excitation. They are studied since the 1980s on various types of data and characterization of the seismic motion and of the structure's response. 
    In this chapter, we propose a review of the numerous methods that exist to estimate the curves. We particularly describe the role played by the characteristics of the available data to select a modeling of the fragility curve.
    This chapter is also the occasion to present some mechanical equipments that will be studied in the following chapters of this manuscript.
\end{abstract}

\minitoc


\section{Introduction}

% The SPRA contains several tools and things

The Seismic Probabilistic Risk Assessment (SPRA) defines a framework and a methodology for the study of seismic structural reliability. %This framework is parrt of the probabilistic 
% As for the Pr
It has been introduced since 1968 \citep{cornell_engineering_1968}
to incorporate the seismic risk evaluation in the probabilistic risk assessment studies.
% and further developed in the
The SPRA has been mostly developed and carried out in the 1980s on nuclear facilities (see e.g. \cite{kennedy_probabilistic_1980,kennedy_seismic_1984}). 
It includes: the determination of the seismic hazard, the analysis of the seismic fragility of structural components, the evaluation of the risks combination and their consequences in the system, as described in the report of the Electric Power Research Institute \citep{epri_seismic_2013}, which provides guidelines for its implementation. %among other ingredients.
%The guidelines for its implementation are thoroughly described in the 
It is since widely used in the nuclear industry (e.g. \cite{ellingwood_validation_1990,park_survey_1998,kennedy_risk_1999}) and its consideration is established among the safety standards adopted by the International Atomic Energy Agency \citep{iaea_probabilistic_2020}.

% Estimating th

Within the SPRA, seismic fragility curves represent a key asset that play a prominent role in the analysis of structural components' fragility step (see  \cite{epri_advanced_2011}).
They express in probabilistic terms the fragility of structures under seismic excitation. We can mention that they are also a tool of interest of performance-based earthquake engineering (PBEE; \cite{ghobarah_performance-based_2001,noh_development_2014}), 
which aims at relating performance objectives to level of damage to the structure.
In any case, they are defined as a function of the seismic hazard, which ---driven by the magnitude (M), the source-site distance (R), and other earthquake parameters--- is reduced to a scalar value derived from the seismic signal: the intensity measure (IM), under the so-called ``sufficiency assumption'' \citep{cornell_hazard_2004,luco_structure-specific_2007}.
In practice, a fragility curve, denoted $P_f$, therefore expresses the probability of failure of a mechanical structure as a function of an IM value of interest such as peak ground acceleration (PGA) or pseudo-spectral acceleration (PSA), among others:
    \begin{equation}
        P_f(a) = \PP(\text{``failure''}|\text{IM}=a).
    \end{equation}
Within the SPRA framework, the fragility curve expression is expected to be combined with the seismic hazard frequency to compute the damage frequency of the component $C_f$:
    \begin{equation}
        C_f =  \int_0^\infty P_f(a)dH(a).
    \end{equation}
If $H$ is the mean annual  distribution of the IM, $C_f$ represents the annual damage frequency of the component.
It should be noted that the sufficiency assumption was introduced to reduce estimation costs since it assumes that the fragility curve of a given structure is identical regardless of the seismic scenario.
Actually, this assumption embeds some uncertainty in the problem since, as shown in \citeauthor{radu_earthquake-source-based_2018}, \citeyearlink{radu_earthquake-source-based_2018,grigoriu_are_2021}, %\cite{radu_earthquake-source-based_2018,grigoriu_are_2021}, 
different seismic scenarios can lead to identical distributions of some IMs, despite having significantly different frequency contents.
The uncertainty rooted in the fragility curve by this assertion can be classified as an \emph{aleatoric} (irreducible) kind of uncertainty
 (we refer to the \cref{chap:intro-english} for the definition of the uncertainty quantification framework).
% belong to the category of
%The evaluation of the fragility curve defined in eq  implies the consideration of this
% Focusing the definition of the fragility curve <


In this second part of the manuscript, we focus on the estimation of seismic fragility curves. 
Thus, we do not question the determination of the seismic hazard distribution $H$, nor the derivation of the damage frequency of the component $C_f$.
Nevertheless, estimating the fragility curve itself %for given mechanical equipment 
remains a daunting task. It is commonly done statistically, using different kind of methods depending 
on the source of data available, their characteristics, and their quantity.
This chapter proposes a review of these methods. 
In the next section,
we define the different kind of data involved when estimating seismic fragility curves. In particular, we present a stochastic seismic signal generator, and  we present how the failure of mechanical equipment is generally defined.
In \cref{sec:intro-frags:models}, different models and methods are presented, they are grouped in two categories: the non-parametric ones and the parametric ones.
Explicit examples of case studies are then given in \cref{sec:intro-frags:casstudies}, from a theoretical one to an experimental one for which really few data are available.
%Given those and the non-exhaustive list of modeling we present from the literature, we question 
Conclusive thoughts are proposed in \cref{sec:intro-frags:conclusion}.






% steps of seismic hzarad determination or r
% The derivation 




% it embedds uncertainty suinc...

% that in the step
%As shown in \cite{Radu2018,Grigoriu2021}, different seismic scenarios can however lead to identical distributions of some IMs, although the underlying seismic signals have significantly different frequency contents. As a result, the sufficiency assumption is not met, especially for non-linear, multimodal structures, etc., with current IMs. 
%
%
% Despite this, it is possible to focus on the definition of the resulting fragility curve, without taking the assumption for granted, which is the case in this work.

% and is widely used in in the nuclear industry







% Seismic fragility curves represent a key one. They were introduced in the 1980s for seismic risk assessment studies carried out on nuclear facilities \citep{kennedy_probabilistic_1980,kennedy_seismic_1984,ellingwood_validation_1990,park_survey_1998,kennedy_risk_1999,cornell_hazard_2004}.


% %IN the SPRA, historic, lot of things are defined,
% %among them, the seismic fragility curves. 
% %They represent an essential tool..
% Also used in PBEE, link with insurance?

% IM, sufficientcy assumption. Different, various scenario and methods
% Quantification of uncertainty?


% This chapter suggests a review of the main methods that exist in the literature 




\section{Data: from seismic signals to equipments failures}\label{sec:intro-frags:data}





In the literature, different sources of data are exploited to estimate seismic fragility curves. 
We can cite, for instance:
(i) expert assessments supported by test data (e.g. \cite{kennedy_probabilistic_1980,kennedy_seismic_1984,zentner_fragility_2017}), (ii) experimental data (e.g. \cite{park_survey_1998}), (iii) empirical data from past earthquakes (e.g. \cite{shinozuka_statistical_2000,straub_improved_2008,lallemant_statistical_2015,buratti_empirical_2017,laguerre_empirical_2024}), and (iv) analytical results obtained from various numerical models using artificial or natural seismic excitations (e.g. \cite{ellingwood_earthquake_2001,kim_development_2004,zentner_numerical_2010,koutsourelakis_assessing_2010,mai_seismic_2017,trevlopoulos_parametric_2019,wang_influence_2020,mandal_seismic_2016,wang_seismic_2018,wang_bayesian_2018,zhao_seismic_2020,katayama_bayesian-estimation-based_2021,gauchy_importance_2021,khansefid_fragility_2023,lee_efficient_2023}).
In every of these studies, each sample in 
the dataset regroups:
\begin{enumerate}
    \item Information about a seismic ground motion. The considered ground motions are sometimes natural and sometimes artificial. The information can take several forms, %(such as the temporal signal), 
    as stated in the introduction it is generally used  to derive one or several scalars called intensity measures (IMs).
    % as stated in the introduction, it is generally used to 
    \item Information about the response of the structure or the equipment of interest to the seismic excitation. To define the fragility curve, this information, must permit to characterize the failure of the studied system.
\end{enumerate}
%(i) information about the seismic ground motion (it can be the temporal signal itself,)

% In the subsection   we pre



Since available records of real seismic excitations for a given site are often scarce, it is common to construct a dataset of artificial earthquake accelerograms using a seismic signal generator.
Various techniques exist for this purpose. According to \citet{rezaeian_stochastic_2008}, the methods can be categorized among the ``source-based'' ones, which model the occurrence of
earthquake rupture at some sources and the propagation of seismic waves to the studied site, and the ``site-based'' ones, which model the seismic signals for the site of interest from the consideration of its characteristics and historical recorded earthquakes.
A review of the first kind is proposed in \cite{zerva_seismic_1988}, and a review of the second can be found in \cite{shinozuka_stochastic_1988}.
As more recent examples for the latter, we can also cite \cite{trevlopoulos_parametric_2019}; and \cite{zentner_enrichment_2012}.
In the following, we present a stochastic seismic signal generator that is proposed by \citet{rezaeian_simulation_2010}. It lies among the site-based models, and has been implemented by \citet{sainct_efficient_2020}, who calibrated it using $97$ real accelerograms selected in the European Strong Motion Database  for a magnitude $M$ such that $5.5 \leq M \leq 6.5$, and a source-to-site distance $R < 20$~km \citep{ambraseys_dissemination_2000}.  An example of one of these real signal is plotted in 
\cref{fig:intro-frags:real-seism}.




As said in the introduction, this thesis does not seek to question thoroughly the seismic hazard, and we aim at providing methods that can be applied to any modeling of the seismic signals. Nevertheless, in the following chapters, our methods are applied and validated on different case studies that are presented later on. These case studies take the form of mechanical equipments that have been submitted to artificial seismic signals generated using the generator implemented by \citet{sainct_efficient_2020}, and presented below.


\begin{figure}[h]
    \centering
    \includegraphics[width=5.625cm]{figures/intro-frags/exseism.pdf}
    \caption{Example of an accelerogram record of a real seismic signal} %: the realtive acceleration $s$ of the ground is .}
    \label{fig:intro-frags:real-seism}
\end{figure}


%the case studies that we present and onto which we will apply our methods are excited using



\subsubsection{Seismic signals generator and IMs}

The seismic excitation generated takes the form of a temporal signal that is the realization $s:t\in[0,T]\mapsto s(t)\in\RR$ of a random process defined on a probability space $(\Omega,\Xi,\PP)$. The generator presented here corresponds to a modulated filtered stochastic white noise with time dependent parameters:
    \begin{equation}
        s(t)= s(t;w,\boldsymbol{\rho},\boldsymbol{\lambda}) = q(t,\boldsymbol\rho)\left[\frac{1}{\sigma_f(t)}\int_{-\infty}^t h(t-\tau,\boldsymbol\lambda(\tau))w(\tau)d\tau \right],
    \end{equation}
where $w$ being a realization of a white noise process $W$. In other terms $W:[0,T]\times\Omega\to\RR$ is such that for all $t_1\ne t_2$, $\EE W(t_1)=0$, $\EE W(t_1)^2=\EE W(t_2)^2$, and $\EE W(t_1)W(t_2)=0$.

The integral in the above equation corresponds to the filtering of $w$, $h(t,\boldsymbol{\lambda})$ being to the impulse response function (IRF) of the linear filter and $\sigma_f$ being its variance: $\sigma_f(t)=\int_{-\infty}^th^2(t-\tau,\boldsymbol{\lambda}(\tau))d\tau$. The IRF is defined by
    \begin{equation}
        h(t-\tau,\boldsymbol{\lambda}(\tau)) = \frac{\omega_f(\tau)}{\sqrt{1-\zeta^2_f}}\exp[-\zeta_f\omega_f(\tau)(t-\tau)]\sin\left(\omega_f(\tau)\sqrt{1-\zeta^2_f}(1-\tau)\right)\indic_{t\geq\tau},
    \end{equation}
where $\boldsymbol{\lambda}(\tau)=(\omega_f(\tau),\zeta_f) $ with $\omega_f(\tau):=\omega_0+\frac{\tau}{T}(\omega_n-\omega_0)$ being the natural frequency and $\zeta_f\in[0,1]$ being a constant damping ratio. The quantities $\omega_0$ and $\omega_n$ are parameters of the IRF.

 The function $q(t,\boldsymbol\rho)$ is the non-negative modulating function, it is defined by
    \begin{equation}
        q(t,\boldsymbol\rho) = \left\lbrace \begin{array}{ll}
            \rho_1t^2/T_1^2 & \text{if\ } 0\leq t< T_1 \\
            \rho_1 & \text{if\ }T_1\leq t< T_2\\
            \rho_1\exp\left[-\rho_2(t-T_2)^{\rho_3}\right] &\text{if\ }T_2\leq t
        \end{array}\right.
    \end{equation}
where $\boldsymbol\rho=(\rho_1,\rho_2,\rho_3,T_1,T_2)\in(0,\infty)^5$.
Therefore, the signal $s$ depends on $w$ and 
$\boldsymbol{\phi}:=(\boldsymbol{\rho},\omega_0,\omega_n,\zeta_f)\in\boldsymbol{\Phi}:=(0,\infty)^7\times[0,1]$.
The generator consists in deriving realizations of $s(\cdot;W,\boldsymbol\phi)$ where 
$\boldsymbol{\phi}$
is stochastic,
its distribution being identified using real acceleration records.
As previously announced,
the records considered are $N_r=97$ real acceleration records form the European Strong Motion Database for a magnitude $M$ such that $5.5\leq M\leq 6.5$ and a source-to-site distance $R<20$~km.
Each of the records corresponds to a realization of $s(\cdot;W,\overline{\boldsymbol{\phi}}_i)$ using a tuple $\overline{\boldsymbol\phi}_i$ as parameters. The distribution of $\boldsymbol{\phi}$ is given by the Gaussian kernel density estimation (see \cite{kristan_multivariate_2011}) of the one of the $(\overline{\boldsymbol\phi}_i)_{i=1}^{N_r}$, i.e. its density $p_{\boldsymbol{\phi}}$ is given by
    \begin{equation}
        p_{\boldsymbol{\phi}}(x) \propto \sum_{i=1}^{N_r}\exp\left( -\frac{1}{2} (x-\overline{\boldsymbol\phi}_i)^\top\Sigma^{-1}(x-\overline{\boldsymbol\phi}_i) \right)\indic_{x\in\boldsymbol{\Phi}},
    \end{equation}
where $\Sigma$ is derived from the $(\overline{\boldsymbol{\phi}}_i)_{i=1}^{N_r}$ (see \cite{kristan_multivariate_2011}).


From a seismic signal $s$ that is a realization of the process described above, different intensity measure (IM) indicators can be derived.
The choice of the appropriate IM to estimate seismic fragility curves
remains a complex question. 
According to \citet{giovenale_comparing_2004}, the appropriateness of an IM must be defined in terms of efficiency, sufficiency, and hazard compatibility.
However, the most efficient or sufficient IM is not the same for two different case studies (see \cite{mackie_probabilistic_2001,hariri-ardebili_probabilistic_2016}). % changes when a different case 
%regarding the studied system, the results given
% study is considered (see ). 
Moreover, while we do not  thoroughly research the best IM in this thesis, we will see in the following chapters that the best choice is not necessarily simply the one that is the most correlated with the structure's response.
Below we propose a non-exhaustive list of common IMs, a more complete one can be found in \cite{luco_structure-specific_2007}:
    \begin{itemize}
        \item the peak ground acceleration (PGA) is defined as $\text{PGA}=\max_{t\in[0,T]}|s(t)|$;
        \item the peak ground velocity (PGV) is defined as $\text{PGV}=\max_{t\in[0,T]}\left|\int_0^ts(\tau)d\tau \right|$;
        \item the peak ground displacement (PGD) is defined as $\text{PGD}=\max_{t\in[0,T]}\left|\int_{0}^{t}\int_{0}^{\tau}s(u)dud\tau\right|$;
        \item the pseudo spectral acceleration (PSA) at frequency $f_L$ and damping ratio $\xi$ is defined as $\text{PSA}=(2\pi f_L)^2\max_{t\in[0,T]}|x(t)|$, where $x$ is the solution of the linear equation
        \begin{equation}\label{eq:intro-frag:ALS}
            x''(t) + 2\xi2\pi f_Lx'(t)+(2\pi f_L)^2x(t) = -s(t).
        \end{equation}
        This IM is component-dependent since, in practice, the damping ratio $\xi$ and the frequency $f_L$ are evaluated from the mechanical characteristic of the studied system. Generally, the frequency $f_L$ considered for deriving the PSA corresponds to the first mode of the structure's displacement under excitation. \Cref{eq:intro-frag:ALS} corresponds to the displacement equation of the linear single degree of freedom system associated to the structure.
    \end{itemize}

For  conducting the numerical experiments of this thesis, $10^5$ seismic signals have been computed using this generator.
In \cref{fig:intro-frags:IM-density}, we draw the distribution of the PGA and the PSA (at frequency $5$~Hz and damping ratio $1\%$) given by this generator. It is compared with the kernel density approximation of the PGA and PSA values of the $97$ real seismic signals that have been used for fitting the generator. %, and with a lognormal distribution with same median and same log deviation. 
These comparisons show that the synthetic signals have realistic features.

\begin{figure}[h]
    \centering
    \includegraphics[width=5cm]{figures/intro-frags/PSA_density.pdf}\ 
    \includegraphics[width=5cm]{figures/intro-frags/PGA_density.pdf}
    \caption{Histograms of IMs derived from $10^5$ generated synthetic signals : the IM is the PSA (left) and the PGA (right). They are compared with the densities of IMs coming from real accelerograms estimated by Gaussian kernel estimation (dashed lines).} %, {and with the densities of a lognormal distribution with same median and same log deviation (solid lines).}}
    \label{fig:intro-frags:IM-density}
\end{figure}



 %the conduction of 
%  the numerical experiments conducted 



%different IMs


% Regarding seismic ground motions, a common methodology



% SAinct et al

% definition of IMs


\subsubsection{Engineering demand parameter and failure}


As evoked in the preamble of this section, numerous data sources exist for identifying the fragility (and the failure) of mechanical structures and components.
In most cases a particular engineering demand parameter (EDP) of the system is focused on. The EDP is a scalar quantity that is observed (numerically or practically) during the seismic excitation, it can be the maximal displacement of a specific part of mechanical equipment for instance. More explicit examples will be given when specific case studies will be presented later on in this chapter. 
In those cases, the failure is defined when the EDP exceeds a threshold limit.

All in all, available datasets often take the form of tuples $((\cS_i,\cD_i))_{i=1}^k$, where $\cS_i$ is the $i$-th seismic signal submitted to the system and $\cD_i$ is the EDP that was observed during the experiment. %The reduced 
%
In the case studies that are treated in this manuscript a specific IM is chosen, and a reduced dataset of the form $((a_i,z_i))_{i=1}^k$ is studied, where $a_i$ is the IM of the $i$-th seismic signal, and $z_i=\indic_{\cD_i> C }$, where $ C $ is the threshold that defines the failure of the system. In other words, $z_i=1$ if the system has failed when submitted to the  $i$-th seismic signal, and $z_i=0$ otherwise.
The dataset permits to statistically estimate the fragility curve of the system of interest. In this thesis, we restrict the study to cases where the observations can be considered independent. Naturally, that restriction excludes some case studies, such as experimental campaigns conducted on concrete structures for instance, since such system get damaged during the experiment, impacting their mechanical behavior. 
As mechanical systems that enter the scope of this thesis, we can cite
%\begin{itemize}
    % \item 
    structures and component  whose response under seismic excitation can be modeled numerically (via finite element analysis for instance), and
    % \item 
    sturdy mechanical equipments that do not get damaged during experimental campaigns such as pipes, electronic devices, stacked structures.
%\end{itemize}
In the last two examples evoked, the failure is not always characterized by observing an EDP, and can be multifactorial. While the studies that are done in this thesis are limited to cases where the system of interest's failure is characterized by an EDP, we emphasize that the methods that we develop and use go beyond these specific cases.



% However, structures and components whose response under seismic excitation can be modeled numerically via finite element methods are


% various mechanical equipment in the nuclear industry are sturdy enough 






\section{Modeling of the fragility curve}\label{sec:intro-frags:models}

Different models and methods have been suggested in the literature to estimate seismic fragility curves.
%Historically, the 
In the genesis of the SPRA, seismic fragility curves were estimated based on expert judgments, principally due to the scarcity of available data to conduct an accurate statistical estimation (see \cite{kennedy_probabilistic_1980}).
Of course, the accuracy of such an estimation is not guaranteed either, and its reliability depends on that of the experts.

Nowadays, estimates of seismic fragility curves are mostly derived using statistical techniques, leveraging (i) dataset that are sometimes less scarce and more precise, and also (ii) statistical techniques that are more efficient. 
The expert judgment still complement those in some works in the literature.


Parametric and non-parametric methods are both used in the literature to estimate seismic fragility curves.
In the former, the probabilistic relation between the failure and the IM is sought. When an EDP is available, authors seek to estimate first the probabilistic relation between the IM and the EDP, to deduce the resulting fragility curve.
In the latter, the fragility curve is assumed to belong to a parameterized set, reducing the problem to the estimation of these parameters.
While the first method is more general in the sense that it covers a wider range of possible estimates, it is also often less efficient than the second. Indeed, reducing the problem to a finite-parameter one allows providing satisfying estimates with fewer data. However, their reliability depends on the trustworthiness of the chosen parametric modeling.

In the following, we review briefly the
state-of-the art on the estimation of seismic fragility curves. We start by a review of non-parametric methods in \cref{sec:intro-frags:subsec-nonparametric}. In that section we also describe a non-parametric estimation that is based on Monte-Carlo estimates, and that can serve as a reference when a large dataset is available.
In \cref{sec:intro-frags:subsec-parametric} we discuss the parametric modelings of the fragility curves, and we present the most common one: the probit-lognormal model.



% , in which we p



% Two main different approaches exist to model the fragility curve. Firstly, one approach consists in estimating 
% It is common to classify the modelings of fragility curves in two categories: the non-parametric ones and the pa








%\subsection{A state-of-the-art}





\subsection{Non-parametric modeling}\label{sec:intro-frags:subsec-nonparametric}


Most of non-parametric models for seismic fragility curves estimation leverage ---if possible--- the knowledge of an EDP that describes more precisely the response of the structure under seismic excitation than just the binary outcome ---failure or non-failure.
In this case it is possible to estimate the conditional distribution of the EDP to the IM (or IMs) via a surrogate modeling of the system $\text{EDP}=\cM(\text{IM})$.

Among common surrogates we can mention the Gaussian processes (GPs). As examples of their use for seismic fragility curves estimation, we can cite \cite{gidaris_kriging_2015}, in which GPs are used to estimate the EDP given multiple parameters characterizing the ground motion; and \cite{gauchy_uncertainty_2024}, in which the GPs-based estimation of the EDP is coupled with a sensitivity analysis of the fragility curve to the mechanical parameters of the studied system.

We also mention polynomial chaos expansion (PCE) as a surrogate numerously used for seismic fragility curves estimation. For instance, PCEs are combined in \cite{mai_surrogate_2016} with non-linear autoregressive with exogenous input models to estimate the temporal response of the mechanical system as a function of the seismic signal.
In \cite{zhu_seismic_2023}, a stochastic PCE is conducted, consisting in the addition of a latent variable and an additive noise to the deterministic PCE expression.
%It is applied to the 

Outside surrogate modeling, what one would call machine-learning-based techniques are also implemented to estimate seismic fragility curves.
For examples, we quote linear regression or generalized linear regression \citep{lallemant_statistical_2015}, classification-based methods (see the review suggested in \cite{kiani_application_2019}) such as  logistic regression (as in \cite{bernier_fragility_2019}) or support vector machine (as in \cite{sainct_efficient_2020}). In the latter work, support vector machine is used to classify EDPs whether they led to failures or non-failures as a function of combinations of IMs.
We also mention artificial networks based methods, such as used in \cite{mitropoulou_developing_2011,wang_seismic_2018}.
 

% Of course, a

\subsubsection{A reference fragility curve constructed with a large dataset}

%All the methods cited above rely on assumptions on the probabilistic relation  between  the structure's response and the seismic excitation.
All the methods cited above provide estimation whose efficiency will depend on the size of the available dataset and the correctness of the assumptions they involve on the probability relation between the system's response and the IMs.
It is possible to minimize such assumptions considerably, trying to estimate directly the probabilities $\PP(\text{``failure''}|\text{IM}=a)$, for all value of $a$, via Monte-Carlo estimates for instance.
Such a methodology must give the best estimation of the fragility curve in terms of robustness, in the sense that it does not rely on any assumption. Of course, one does not have infinitely many data for all value of $a$ to provide an exact estimation of the whole curve, yet it is possible to approximate the curve locally in different sub-areas of the domain $\cA$ in which $a$ lives.
More explicitly consider a dataset $((a_i,z_i))_{i=1}^{N}$ of IMs and binary outcomes ($z_i=1$ if the $i$-th seismic signal led to a failure, and $0$ otherwise), and choose $N_c$ clusters of the $(a_i)_{i=1}^N$ that we denote $(K_j)_{j=1}^{N_c}$: $\bigsqcup_{j=1}^{N_c}K_j=\{a_i,$ $i=1,\dots,N\}$. Then it is possible to approximate the fragility curve evaluated at the centroids $(c_j)_{j=1}^{N_c}$ of the clusters:
    \begin{equation}
        P_f^{\text{MC}}(c_j) = \frac{1}{n_j}\sum_{i,\, a_i\in K_j}z_i,
    \end{equation}
where $n_j$ is the sample size of cluster $K_j$.
This non-parametric estimation is implemented by \citet{trevlopoulos_parametric_2019}, who suggest defining the clusters $(K_j)_{j=1}^{N_c}$ using K-means since the IM values in available datasets are generally not uniformly distributed.

When the number of available data is little and when they are poorly diverse in terms of values of the IM, this estimation method becomes limited for estimating seismic fragility curves. However, in the other case, we consider that it provides a robust result. %In this thesis, 
% In this thesis, when evaluating 
For this reason, while this thesis address the estimation of fragility curves when few data are available, we will evaluate our methods on case studies for which a large validation dataset exists.
The latter is used to derive what we call a reference fragility curve that is identified to $P^{\text{MC}}_f$, and which will be compared with the estimates provided by our methods.






%\subsection{Parametric modeling}

%probit lognormal



\subsection{Parametric modeling: probit-lognormal model}\label{sec:intro-frags:subsec-parametric}

\subsubsection{The probit-lognormal model}


Although based on stronger assumptions on the structure's response than non-parametric methods, parametric fragility curves were historically considered, especially in cases with small dataset limited to binary outcome (i.e. no EDP is available).
%Several model can 
In the SPRA and PBEE frameworks, the so-called probit-lognormal model was chosen, and it remains prevalent to this day (see e.g. \cite{shinozuka_statistical_2000,straub_improved_2008,zentner_numerical_2010,wang_influence_2020,mandal_seismic_2016,zhao_seismic_2020,ellingwood_earthquake_2001,kim_development_2004,mai_seismic_2017,trevlopoulos_parametric_2019,katayama_bayesian-estimation-based_2021,lee_efficient_2023}).
% This model is presented 
This model consists in introducing a parameter $\theta=(\alpha,\beta)$ and defining the fragility curve as follows
    \begin{equation}\label{eq:intro-frag:probit}
        P_f(a) = \PP(\text{``failure''}|\text{IM}=a) = \Phi\left( \frac{\log a-\log\alpha}{\beta} \right),
    \end{equation}
where $\Phi$ is the cumulative distribution function of a standard normal variable. An example of such curves is given in \cref{fig:intro-frags:exfrags}.
We mention that, sometimes, the parametrization of the model is slightly different. As recalled by \cite{zentner_fragility_2017}, the distinguishing of epistemic and aleatoric uncertainty in $\beta$ is commonly suggested, leading to rewriting \cref{eq:intro-frag:probit} with $\beta=\sqrt{\beta^2_U+\beta^2_R}$. 
% The EPRI guide   suggested standard values 
The literature suggests that the uncertainty embedded in $\alpha$ is only epistemic.

\begin{figure}[h]
    \centering
    \includegraphics[width=5.225cm]{figures/intro-frags/exfrags.pdf}
    \caption{Examples of probit-lognormal fragility curves.}
    \label{fig:intro-frags:exfrags}
\end{figure}


Different strategies exist to estimate the two parameters, namely the median $\alpha$ and the log standard deviation $\beta$. Historically, maximum likelihood estimation (MLE) techniques are recommended.  %with confidence interval derivation via bootstrapping are recommended (see for instance ).
When the data are independent, the bootstrap technique can be used to obtain confidence intervals relating to the size of the sample considered (e.g. \cite{shinozuka_statistical_2000,zentner_numerical_2010,wang_influence_2020}). %, ZENTNER20101614, WangF2020}). 
% \cite{Shinozuka2000, ZENTNER20101614, WangF2020}. 
To describe this technique,
we define first the associated statistical model: the observations are modeled as realizations of the random variable $(A,Z)\in\cA\times\{0,1\}$. It is supposed that observations are independent conditionally to $\theta$ and that $(A,Z)$ are distributed conditionally to $\theta\in\Theta=(0,\infty)^2$ as:
    \begin{equation}
            A|\theta\sim A\sim H,\quad\text{and}\quad Z|A,\theta\sim\cB\left(\Phi\left(\beta^{-1}\log\frac{a}{\alpha}\right)\right)
    \end{equation}
where $H$ is the distribution of $A$ and $\cB(p)$ denotes a Bernoulli distribution of mean  $p$.
Thus, the likelihood $\ell_k$ given observations $(\mbf a^k,\mbf z^k)$ with $\mbf a^k=(a_i)_{i=1}^k$ and $\mbf z_k=(z_i)_{i=1}^{k}$ is expressed as
    \begin{equation}
        \ell_k(\mbf z^k,\mbf a^k|\theta) = \prod_{i=1}^k\ell(z_i,a_i|\theta) = \prod_{i=1}^k\Phi\left(\frac{\log a_i-\log\alpha}{\beta}\right)^{z_i}\left(1-\Phi\left(\frac{\log a_i-\log\alpha}{\beta}\right)^{1-z_i}\right)h(a_i),
    \end{equation}
where $h$ is the p.d.f. of $A$. Since the $(h(a_i))_i$ are constants of $\theta$ and are not known in general, it is common to consider the following alternative form of the likelihood:
\begin{equation}
    \ell_k(\mbf z^k|\mbf a^k,\theta) = \prod_{i=1}^k\ell(z_i|a_i,\theta) = \prod_{i=1}^k\Phi\left(\frac{\log a_i-\log\alpha}{\beta}\right)^{z_i}\left(1-\Phi\left(\frac{\log a_i-\log\alpha}{\beta}\right)^{1-z_i}\right).
\end{equation}
The maximum likelihood estimator is $\theta^{\text{MLE}}(\mbf z^k,\mbf a^k)=\argmax_{\theta\in\Theta}\ell_k(\mbf z^k,\mbf a^k|\theta)$.
The bootstrap technique consists in deriving a stochastic estimator $\hat\theta_k^{\text{BMLE}}(\mbf z^k,\mbf a^k)$ whose distribution is defined from expressing it as: $\hat\theta^{\text{BMLE}}(\mbf z^k,\mbf a^k)=\theta^{\text{MLE}}((z_{U_i},a_{U_i})_{i=1}^k)$, where $U_1,\dots,U_k$ are i.i.d. random variables distributed w.r.t. a uniform distribution in $\{1,\dots,k\}$.



 
% we consider the statisct

% let us start by defining formally the statistical probit-lognormal model: 



\subsubsection{About the implementation of the Bayesian framework to estimate $\theta=(\alpha,\beta)$}

As a strategy that allows to estimate the parameters defining the fragility curve, 
the Bayesian framework has recently become increasingly popular in seismic fragility analysis (see e.g. \cite{gardoni_probabilistic_2002,wang_bayesian_2018,katayama_bayesian-estimation-based_2021,koutsourelakis_assessing_2010,damblin_approche_2014,tadinada_structural_2017,kwag_computationally_2018,jeon_parameterized_2019,tabandeh_physics-based_2020}). 
It is praised for its capacity to solve the irregularity issues encountered when estimating fragility curves with classical methods. For instance, the MLE-bootstrap method is known to lead to unrealistic estimates such as unit-step functions.
However, a challenge remains when %The difficulty that remains when 
implementing the Bayesian framework, and 
lies in the selection of the prior.
As a matter of fact, as announced since the \cref{chap:intro-english}, selecting a prior is a critical step in Bayesian analysis, %especially in a context where (i) the datasets of interest are light, and (ii) the methodology must be auditable.
and when exploring the literature, a wide range of different consideration can be found regarding its construction. % prior
%

%In earthquake engineering, Bayesian inference is often used to update existing log-normal fragility curves previously obtained through various approaches, assuming independent distributions for the prior values of $\alpha$ and $\beta$, such as log-normal distributions. 
%
For example, in \cite{tadinada_structural_2017} and \cite{kwag_computationally_2018}, the median prior values come from equivalent linearized mechanical models. In \cite{wang_bayesian_2018}, both aleatory and epistemic uncertainties are taken into account in the parametric model originally introduced in \cite{kennedy_probabilistic_1980}: an artificial neural network is trained and used to characterize (i) the aleatory uncertainty and (ii) the prior median value of $\alpha$, while the associated epistemic uncertainty is taken from the existing literature. The log-normal prior distribution of $\alpha$ is then updated with empirical data. 
In \cite{katayama_bayesian-estimation-based_2021}, the results of incremental dynamic analysis are used to obtain a prior value of $\alpha$, whereas the prior value of $\beta$ is determined through a parametric study. This results in satisfactory convergence, whatever its target value, before application to practical problems.
In \cite{straub_improved_2008}, the authors mainly focus on the implications for fragility analyses of statistical dependencies within the data. The prior is defined as the product of a normal distribution for $\ln(\alpha)$, and the improper distribution $1/\beta$ for $\beta$. The definition of the normal distribution is based on engineering assessments. This prior was preferred to $1/\alpha$ on the grounds that it led to unrealistically large posterior values of $\alpha$. A sensitivity analysis is further performed to examine the impact of the choice of the prior distribution on the final results.
Finally, we note that the Bayesian framework is also relevant for fitting numerical models (e.g., mathematical expressions based on engineering assessments or physics-based models) to experimental data in order to estimate fragility curves \citep{gardoni_probabilistic_2002,tabandeh_physics-based_2020} or meta-models such as logistic regressions \citep{koutsourelakis_assessing_2010,jeon_parameterized_2019}.







% In the SPRA framework, Bayesian inference is often used to update existing probit-lognormal fragility curves previously obtained through various approaches, assuming probit-lognormal distribution only for $\alpha$ \cite{WANG2018232}, 
% independent distributions for the prior values of $\alpha$ and $\beta$ such as uniform distributions \cite{KOUTSOURELAKIS2010}, the product of a normal distribution for $\ln(\alpha)$ and the improper distribution $1/\beta$ for $\beta$ \cite{Straub2008}, etc.


% in \cite{wang_bayesian_2018}, $\beta$ is replaced by in eq  to 


%\cite{Shinozuka2000,Lallemant2015,Straub2008,ZENTNER20101614, WangF2020, MANDAL201611, WANGZ2018, WANG2018232, ZHAO2020103, ELLINGWOOD2001251, KIM2004, Mai2017, TREVLOPOULOS2019,Katayama2021,LEE2023}


\section{Example of case studies}\label{sec:intro-frags:casstudies}

    \subsection{An elasto-plastic oscillator}\label{sec:intro-frags:elastoplastic}

    The first case study that we present is 
    a single-degree-of-freedom elasto-plastic oscillator with kinematic hardening.
    %It represents a con
    This simple mechanical system illustrates the essential features which can be found in the nonlinear responses of some real-world structures under seismic excitation and has, for this reason, already been used in several studies \citep{trevlopoulos_parametric_2019,sainct_efficient_2020,gauchy_importance_2021}.
    In addition, it provides reference results as a reasonable numerical cost.
    It is depicted in \cref{fig:intro-frags:elasto}, and
    its motion equation when submitted to an excitation $s$ is the following:
    \begin{equation}
        x''(t) +2\xi2\pi f_L x'(t)+f^{\text{nl}}(t) = -s(t),
    % \ddot{y}(t) + 2 \zeta \omega_{\text{L}}\dot{y}(t) + f(t) = -s(t) \ ,
    \end{equation}
    %with $s(t)$ a seismic signal, $\dot{y}(t)$ and $\ddot{y}(t)$ respectively the relative velocity and acceleration of the mass, 
    with $x''$ and $x'$ respectively the relative velocity and acceleration of the mass,  $\xi$ the damping ratio, $f_L$ the circular frequency, and $f^{\text{nl}}$ the nonlinear resisting force.

    \begin{figure}[t]
        \centering
        \includegraphics[width=6cm]{figures/intro-frags/KBEPO_rheo.pdf}
        \caption{Elasto-plastic oscillator of mass $m$ with kinematic hardening, with parameters $f_{\text{L}} = 5$ Hz and $\xi = 2\%$. The yield limit is $Y = 5.10^{-3}$~m, and the post-yield stiffness is $20\%$ of the elastic stiffness, i.e., $a = 0.2$.} 
        \label{fig:intro-frags:elasto}  
    \end{figure}

    
    
    The relevant EDP is the absolute maximum value of the mass’ displacement, i.e., $\text{EDP}=\max_{t\in [0, T]}|x(t)|$, where $T$ is the duration of the seismic excitation. In \cref{fig:intro-frags:oscillclouds}, we plot that EDP that has been derived for the $10^5$ seismic signals generated and presented in \cref{sec:intro-frags:data}.

    \begin{figure}[t]
        \centering
        \includegraphics[width=5cm]{figures/intro-frags/oscill/cloudPSA.pdf}
        \includegraphics[width=5cm]{figures/intro-frags/oscill/cloudPGA.pdf}
        \caption{Results of the $10^5$ simulations. Each cross is an element of the dataset (IM, EDP) where the IM is the PSA (left) and the PGA (right). Different critical rotation thresholds $C$ are plotted in dashed lines. They yield different proportions of failures in the dataset: respectively 95$\%$ (red), $90\%$ (purple) and $85\%$ (blue).}
        \label{fig:intro-frags:oscillclouds}
    \end{figure}


    Since a very large number of data are available for this case study, it is possible to derive a reference fragility curve using the non-parametric method that is described in \cref{sec:intro-frags:subsec-nonparametric}.
    Such reference fragility curves are plotted in \cref{fig:intro-frags:oscillrefs}, for different thresholds $ C $ that define the failure.
    In general, the failure criterion that corresponds to the $90\%$-level quantile of the maximum displacements calculated with the $10^5$ artificial signals is chosen, i.e., $C = 8.0 \; 10^{-3}$~m.
    The reference 
    curves are compared with the probit-lognormal fragility curves (described in \cref{sec:intro-frags:subsec-parametric}) where $\theta$ is estimated by MLE using the full dataset as well. This comparison is presented for both the PSA and the PGA as IM.
    It demonstrates that the probit-lognormal model allows a good approximation of the reference curve.

    \begin{figure}[h]
        \centering
        \includegraphics[width=5cm]{figures/intro-frags/oscill/refs_PSA.pdf}
        \includegraphics[width=5cm]{figures/intro-frags/oscill/refs_PGA.pdf}
        \caption{{Reference non-parametric fragility curves obtained via Monte Carlo estimates (dashed lines) surrounded by their $95\%$ confidence intervals, for different critical rotation threshold $C$ with (left) the PSA and (right) the PGA as IM.} The thresholds yield different proportions of failures in the dataset: respectively $95\%$ (red), $90\%$ (purple) and $85\%$ (blue).
        For each value of $C$ are plotted (same color, solid line) the corresponding probit-lognormal MLE.}
        \label{fig:intro-frags:oscillrefs}
    \end{figure}


    
    \subsection{A piping system from a pressurized water reactor}\label{sec:intro-frags:piping}

    {The case study presented in this section is a piping system that was tested on the Azalee shaking table at the EMSI laboratory of CEA/Saclay, as shown in \cref{fig:ASG}-left. \Cref{fig:ASG}-right depicts the finite element model (FEM), based on beam elements and implemented through the proprietary FE code CAST3M \citep{cea_cast3m_2019}. The validation of the FEM was carried out thanks to an experimental campaign described in \cite{touboul_seismic_1999}.}

    The mock-up comprises a carbon steel TU42C pipe with an outer diameter of 114.3 mm, a thickness of 8.56 mm, and a 0.47 elbow characteristic parameter. This pipe, filled with water without pressure, includes three elbows, with a valve-mimicking mass of 120 kg, constituting over 30\% of the mock-up's total mass. One end of the mock-up is clamped, while the other is guided to restrict displacements in the X and Y directions. Additionally, a rod is positioned atop the specimen to limit mass displacements in the Z direction (refer to \cref{fig:ASG}-right). During testing, excitation was applied exclusively in the X direction.



\begin{figure}[h]
		\centering		
		\includegraphics[width=5.2cm]{figures/intro-frags/ASG.jpg}
		\hspace{1cm}
		\includegraphics[width=2.5cm]{figures/intro-frags/ASG_FEM.pdf}
		\caption{(left) Overview of the piping system ---which is part of a French pressurized water reactor--- on the Azalee shaking table and (right) associated finite element model.}
		\label{fig:ASG}
	\end{figure}




    {In order to conduct comparative performance studies, numerous simulations have been performed. They were carried out from a subset of $8\cdot10^4$ of the $10^5$ artificial seismic signals. Nevertheless, as in practice the piping system is located in a building, the artificial signals were filtered using a fictitious 2\% damped linear single-mode building at 5 Hz, which corresponds to the first eigenfrequency of the 1\% damped piping system.  In such a situation, for some seismic signals, the behavior of the piping system is nonlinear. Regarding the nonlinear constitutive law of the material, a bilinear law exhibiting kinematic hardening was used to reproduce the overall nonlinear behavior of the mock-up with satisfactory agreement compared to the results of the seismic tests \citep{touboul_seismic_1999}. Following the recommendation in \cite{touboul_enhanced_2006}, we consider that the engineering demand parameter (EDP) is the out-of-plane rotation of the elbow near the clamped end of the mock-up. As a result we have a dataset of $8\cdot10^4$ independent tuples of the form (IM, EDP) for different IMs. In \cref{fig:asg:scattersIMs} are plotted the elements of the datasets (PSA, EDP) and (PGA, EDP), along with different critical thresholds.}
    %The binary data are defined from the condition that failure happens when the EDP is larger than the threshold value.}

    \begin{figure}[h]
        \centering%
        \includegraphics[width=5cm]{figures/intro-frags/asg/cloud_PSA_light.pdf}%
        \includegraphics[width=5cm]{figures/intro-frags/asg/cloud_PGA_light.pdf}%
        \caption{Results of the $8\cdot10^4$ numerical simulations. Each cross is an element of the dataset (IM, EDP) where the IM is the PSA (left) and the PGA (right). Different critical rotation thresholds $C$ are plotted in dashed lines. They yield different proportions of failures in the dataset: respectively 95$\%$ (red), $90\%$ (purple) and $85\%$ (blue).}
        \label{fig:asg:scattersIMs}
        \end{figure}
    
    The complete set of $8\cdot 10^4$ simulations provides a satisfactory dataset to derive a reference of the seismic fragility curve for this case study, still using the non-parametric method depicted in \cref{sec:intro-frags:subsec-nonparametric}.
    % In 
    In \cref{fig:asg:reference-frags}, we compare this non-parametric reference with the parametric fragility curve using the probit-lognormal model (\cref{sec:intro-frags:subsec-parametric}) where $\alpha$ and $\beta$ are estimated by MLE using the $\cdot 10^4$ data items as well. {This comparison is presented considering different critical rotation thresholds $ C $ of the equipment for both the PSA and the PGA. First, it should be noted that, with the PGA as IM, it is not possible to completely describe the fragility curve. For the maximum PGA values observed, the failure probabilities stagnate between 0.5 and 0.8 depending on the failure criterion considered. Therefore, the PGA is not the most suitable IM of the two. For the type of structure considered here, this point is well documented in the literature. Then, the comparisons demonstrate, in our setting, the adequacy of a probit-lognormal modeling of the fragility curves, even with high failure thresholds.} Its bias with non-parametric fragility curve exists but remains limited. When the number of observations is small, this model bias is negligible in front of the uncertainty on the estimates.


    
    
    \begin{figure}[h]
        \centering
        \includegraphics[width=5cm]{figures/intro-frags/asg/refs_PSA.pdf}
        \includegraphics[width=5cm]{figures/intro-frags/asg/refs_PGA.pdf}
        \caption{{Reference non-parametric fragility curves obtained via Monte Carlo estimates (dashed lines) surrounded by their $95\%$ confidence intervals, for different critical rotation threshold $C$ with (left) the PSA and (right) the PGA as IM.} The thresholds yield different proportions of failures in the dataset: respectively $95\%$ (red), $90\%$ (purple) and $85\%$ (blue).
        For each value of $C$ are plotted (same color, solid line) the corresponding probit-lognormal MLE.}
        \label{fig:asg:reference-frags}
        \end{figure}

    
    %The others serve the illustration.
    %These values are turned into binary outcomes ---failures or successes--- depending on them exceeding a critical threshold. 
    %Response here corresponds to the binary outcome: failure or success.

    
   
    


    \subsection{Stacked structure for storage of packages}\label{sec:intro-frags:stacked}

% \cite{beylat_contribution_2020}
    % \subsection{Appropriate modeling with how many data and what data}


    The case study considered hereafter concerns a freestanding stacked structure composed of three pallets intended for the storage of packages. As shown in \cref{fig:intro-frags:EDEN}, it is a square-based structure 3 meters high with a slenderness of 2.4.
    
    
    
    The mass of one package is equal to 265~kg, whereas the mass of one pallet is equal to 60 kg. The total mass of the structure is equal to $3,360$~kg. The pallets are made of 3~mm-thick hollow-section aluminum square tubes. These tubes are welded and form the base and uprights of the pallets. The base of a pallet supports four freestanding packages, whereas the uprights support the upper freestanding pallet(s). The stacked structure was studied in \cite{beylat_contribution_2020} by means, among others, of a complete experimental campaign, which was carried out on the 1D shaking table ``Vesuve'' of CEA/Saclay. In particular, the stack was subjected to 21 of the 97 real signals that were used to fit the generator depicted in \cref{sec:intro-frags:data}.
    For this structure, the EDP corresponds to the maximal displacement of the top of the stack.
    % the failure criterion relates to an excessive displacement. %of the top of the stack. 
    An example of a test result is shown in \cref{fig:intro-frags:EDEN}, which depicts the horizontal displacements over time of the top of the stack. The initial position is indicated in red, while the different positions in time are indicated in blue. {Due to uplift, sliding, and rotation motions, the top of the stack exceeds, for 2 of the 21 tests, the admissibility criterion, which is materialized in black. }
    Only a little number of experimental result are available for this case study, so that it is not possible to derive a reference fragility curve here.





    \begin{figure}[h]
		\centering		
		\includegraphics[width=4.5cm]{figures/intro-frags/EDEN.jpg}
		\hspace{0.5cm}
		\includegraphics[width=7cm]{figures/intro-frags/EDEN_R169.pdf}
		\caption{(left) Overview of the stacked structure placed on the Vesuve 1D shaking table and (right) example of test result: horizontal displacements of the top of the stack when subjected to seismic excitation in the X direction.}
		\label{fig:intro-frags:EDEN}
	\end{figure}  


\section{Conclusion: which case study and which data for a Bayesian estimation of fragility curves?}\label{sec:intro-frags:conclusion}

% The estimation of seismic fragility curves is an open question that is 


In the literature, many works address the estimation of seismic fragility curves, introducing many approaches to handle that task.
Nevertheless, the appropriate method to provide robust estimates heavily depends on the characteristics of the studied equipments, on the 
type of data that are available, and on their number.
% As a matter of fact, we have cited numerou
% The data curation is itself a main step that has to be taken into account when conducting a methodology of seismic fragility curves estimation. 
In this chapter, we have described standard forms of datasets considered in the literature, and we have depicted how they are used to providing approximation of fragility curves using different approaches.

In the worst case, the dataset is %, limiting the use of non-parametric methods
%and 
made of tuples composed by (i) a value of the IM, and (ii) a binary outcome about the system's response: failure or non-failure.
In those cases, the non-parametric models provide limited estimation, and a parametric model, such as the prominent probit-lognormal one has to be favored.
If the dataset is also scarce, then most classical frequentist methods become unsatisfying as well, letting the Bayesian framework to arise as a cornerstone.
% leading to 

However, 
a major issue connects all the studies 
% while a plethora of studies 
suggesting a Bayesian estimation of seismic fragility curves that we reviewed: the prior selection is not thoroughly questioned. Moreover, the choices done regarding its construction are hard to justify, and one could question the reliability of the provided estimates.
% , we observed
% they all suffer from their prior choice, 

% questionable priors
% choices

%when reviewing the literature, we observed that 

All in all, this chapter and that last thought introduces the essential interplay that exists between the \cref{part:ref-theory} and the \cref{part:spra} of this manuscript.
Since our work is sought to be applied on case studies taken from the nuclear industry, we aim at providing a reliable and auditable methodology when we estimate seismic fragility curves.
The expected  auditability also concerns the choice of the prior, which hence must be constructed in a way that ensures its objectivity.












% In this case, a paramterized model is prominent







% We have described 







% \section{Conclusion}






\chapter{Bayesian estimation of seismic fragility curves using a reference prior}\label{chap:prem}




\begin{abstract}[\hspace*{-10pt}]
    This chapter draws mainly on the published work: \fullcite{van_biesbroeck_reference_2024}  % Ce chapitre reprend principalement les travaux publiés dans: 
\end{abstract}

\begin{abstract}
    abstract
\end{abstract}


\minitoc

\section{Introduction}


\section{Probit-lognormal model and Bayesian framework}

We remind that the probit-lognormal model of the fragility curve has been described in the \cref{chap:frags-intro}.
It defines the fragility curve of the mechanical system of interest as 
\begin{equation}\label{eq:PEM:probitfrag}
    P_f(a)=\PP({``failure''}|IM=a) = \Phi\left(\frac{\log a-\log\alpha}{\beta}\right),
\end{equation}
where $\Phi$ is the c.d.f. of a standard Gaussian, and $\alpha$, $\beta$ are parameters that we seek to estimate. To be precise, $\alpha\in(0,\infty)$ is the median and $\beta\in(0,\infty)$ is the log standard deviation of the curve. We denote $\theta=(\alpha,\beta)\in\Theta=(0,\infty)^2$.

In statistical terms, we consider that the failure of the equipment is the realization of a random variable $Z$, which takes values in $\{0,1\}$ ($1$ for failure, $0$ for non-failure). We also denote by $A$ the random variable of the IM. It takes value in a set $\cA=(0,\infty)$ and is supposed to follow a distribution $H$. Conditionally to $\theta$, the tuple $Y=(Z,A)$ follows a distribution defined by $A\sim H$ and $Z|A,\theta\sim\cB(P_f(a))$, where $\cB(p)$ denotes the Bernoulli distribution of parameter $p$, and $P_f(a)$ is defined in \cref{eq:PEM:probitfrag}.

We recall that given realizations $(\mbf z_k,\mbf a_k)$, where $\mbf z_k=(z_i)_{i=1}^k$, $\mbf a_k=(a_i)_{i=1}^k$, of the r.v. $Y$, this model admits the following likelihood:
\begin{equation}
    \ell_k(\mbf z^k|\mbf a^k,\theta) = \prod_{i=1}^k\ell(z_i|a_i,\theta) = \prod_{i=1}^k\Phi\left(\frac{\log a_i-\log\alpha}{\beta}\right)^{z_i}\left(1-\Phi\left(\frac{\log a_i-\log\alpha}{\beta}\right)\right)^{1-z_i}.
\end{equation}




We also recall that the reference prior theory is comprehensively introduced in the \cref{chap:intro-ref}. 





\section{Jeffreys prior construction in the probit-lognormal model}


    \subsection{Derivation of the Jeffreys prior}


    \subsection{Practical implementation}



    \subsection{Thorough study of the prior's decay rates}




\section{Competing approaches and estimation tools}



    \subsection{Bayesian estimates of the seismic fragility curve}

    \subsection{Competing prior taken from the literature}

    \subsection{Maximum likelihood estimation with bootstrapping}




\section{Limits of the estimates given by the three approaches: the curse of degeneracy}


\section{Performance evaluation metrics}



\section{Numerical applications}


\subsection{Case study 1: the elasto-plastic oscillator}

\subsection{Case study 2: the piping system from a pressurized reactor}

\subsection{Case study 3: the stacked structure for storage of packages}



\section{Conclusion}




\chapter{Appropriate constraint incorporation in the probit-lognormal reference prior%reference priors  and implementation of a properly constrained reference prior for estimating fragility curves
}\label{chap:constrained-frags}




\begin{abstract}[\hspace*{-10pt}]
    This chapter draws mainly on the published work: \fullcite{baillie_bayesian_2025}  % Ce chapitre reprend principalement les travaux publiés dans: 
\end{abstract}

\begin{abstract}
    abstract
\end{abstract}

\minitoc


\section{Introduction}


intro



The rsults in the appendix... the curse of degeneracy prevents  from using a more correlated IM. Which  is paradoxal since, ..


\chapter{Design of experiments with constrained reference priors for robust inference% in a Bayesian framework
}\label{chap:doe}





\begin{abstract}[\hspace*{-10pt}]
    This chapter draws mainly on the submitted work: \fullcite{van_biesbroeck_robust_2025}  % Ce chapitre reprend principalement les travaux publiés dans: 
\end{abstract}

\begin{abstract}
    abstract
\end{abstract}


\minitoc

\section{Introduction}

intro



\part{Conclusion \& perspectives}\label{part:conclusion}


\begin{mtchideinmaintoc}[-1]
\chapter{Conclusion \& perspectives}
\end{mtchideinmaintoc}
% \refstepcounter{chapter}
% \addcontentsline{toc}{chapter}{}%
% \markboth{Conclusion \& perspectives}{}
% \addtocounter{chapter}{1}
% \addstarredchapt{}
% \mtcaddchapter
% \adjustmtc
%\decrementmtc


\begin{abstract}[Foreword]
    In this part, we provide a general conclusion to this thesis and this manuscript. Following the structure below, we express the conclusive thoughts that terminate this work.
\end{abstract}


\minitoc

% \listofconcSec
%\parttoc
% \adjustmtc
% \renewcommand{\thesection}{}

% \mtcaddchapter
% \setcounter{section}{0}
% \adjuststc
% \section*{test}
% \addtocontents{lof}{test}

% \concSec{Test}






% \section{Summary}

% % Dans le chapitre 1, nous avons apporté un a priori au contenu de ce manuscrit.
% In this last part, we propose an \emph{a posteriori} analysis of the research conducted during this thesis, and developed in  \cref{part:ref-theory} and \cref{part:spra}.
% The \emph{a priori} of this content was elucidated in \cref{chap:intro-english} through an analysis of the problem and an analysis of the approaches that it invited, in our opinion, to explore.
% This way, we remind that 6 open questions where elucidated?, we recapitualte those below:\\[-5pt]





% \ques{i}{: How can one define and support the objectivity of a prior?}


% \ques{ii}{: What are the limits of the implementation of such non-informative priors, and how to reconcile their use with practical needs?}

% \ques{iii}{: How to build and derive such priors in practice?}

% \ques{iv}{: in the context of SPRA, what are the forms of these objective priors within a model for seismic fragility estimation}

% \ques{v}{: What are the consequences entailed by the lack of information in this model and how to solve them?}

% \ques{vi}{: How to finally leverage the most the different sources of information in the whole Bayesian workflow of the studied model?}





% %Naturally, these 6 questions address two different domains that are distinc from a first sight?




% %Ici, nous conduisons l'analyse a posteriori, au regard des résultats ui font suite au deux parties qui précèdent.


% % Pour la construction de l'a priori à ce contenu, nous avons conduit au chapitre 1 une analyse de la problématique, et une reflexion sur les approches qu'elle semblait inviter à mener.
% % De cette sorte, on rappelle que cette analyse donnait lieu à l'expression de 6 questions ouverts, que nous récapitulons ci-dessous.









% Naturallement, ces 6 questions adressent 2 mondes a première vue distincts : la théorie des priors de référence ainsi que l'étude de courbes de fragilité sismique.
% Ainsi, les questions, l'analyse de sujet et de la problématique a d'abord été enrichie d'une analyse profonde de l'etat de l'art de chacun de ces deux domaines.
% Cette distinction a permis de structurer les travaux conduit pendant cette thèse ainsi que le manuscrit selon s'il se ratachait plutôt à l'un ou à l'autre des deux domaines. 
% C'est pourquoi dans les deux sections qui suivent (\cref{sec:concl:refpr,sec:concl:frags}) nous faerrons le point sur les aspects contributifs de ce travail 
% à chacun des deux sujets.
% Cependant, nous appuyons aussi l'aspect contributif propre de l'interaction entre les deux sujet, novateur et qui est la finalité de la réponse à la problématique du sujet, nous en parlons en \cref{sec:concl:general}.
% Ces travaux sont vaste etpuisqu'ils appréhendent un large pannel de choses, ils ouvrent par la même grand nombre de pistes. Nous discontons de celles-ci en \cref{sec:concl:perspectives}.







\section{Summary}




In this final chapter, we present an \emph{a posteriori} analysis of the research conducted throughout this thesis, that we developed in \cref{part:ref-theory} and \cref{part:spra}. The \emph{a priori} takes the form here of the 
motivations and preliminary considerations that were outlined in \cref{chap:intro-english}, through a detailed examination of the problem and a discussion of the methodological avenues that, in our view, invited for further exploration.
To structure this investigation, we recall that we identified six guiding questions that shaped the architecture of the thesis. These are recapitulated below:\\[-5pt]


\ques{i}{: How can one define and support the objectivity of a prior?}


\ques{ii}{: What are the limits of the implementation of such non-informative priors, and how to reconcile their use with practical needs?}

\ques{iii}{: How to build and derive such priors in practice?}

\ques{iv}{: in the context of SPRA, what are the forms of these objective priors within a model for seismic fragility estimation}

\ques{v}{: What are the consequences entailed by the lack of information in this model and how to solve them?}

\ques{vi}{: How to finally leverage the most the different sources of information in the whole Bayesian workflow of the studied model?}

These six questions naturally span two domains that may appear distinct at first glance: the reference prior theory and the study of seismic fragility curves. Accordingly, the initial formulation of the research questions and the analysis of the core problem were complemented by an in-depth review of the state of the art in both fields.

This distinction helped to structure the work carried out in this thesis, as well as the organization of the manuscript itself, with each part aligned more closely with one of the two domains. Consequently, in the following sections (\cref{sec:concl:refpr,sec:concl:frags}), we reflect separately on the contributions made to each area. However, particular emphasis is also placed on the original contribution outlined from the interaction between the two domains. This interaction represents an innovative aspect that ultimately address the main problem addressed by the thesis, we discuss this point in \cref{sec:concl:general}.
These works are various and given the wide range of issues they engage with, several open directions for future research emerge. These are discussed in \cref{sec:concl:perspectives}.




\section{On the contributions to the reference prior theory}\label{sec:concl:refpr}

% L'objectivisme
% % La qualification d'objectivité 
% reste un idéal pas très bien défini dans le workflow Bayésien. La litérature s'accorde plutôt en règle général sur le fait qu'on s'en approche lorsque l'on favorise les données et les informations qu'elles apportent en ce sens devant tout autre incorporation subjective.
% Il est difficile de conclure si les priors de référence sont des priors objectifs, néanmoins, leur construction cherche à minimiser l'introduction de subjectivité par le prior.

% Au final, on peut donc voir un prior objectif comme un idéal, et le cadre des priors de référence apporte un indice de quantification d'une proximité de cet idéal.
% On a tenté dans le \cref{chap:ref-generalized} d'étendre la portée de cet indice, en cherchant à le rendre plus général, nos résultats ont prouvé que la solution du problème définissant les priors de référence %(i.e., quel est le prior de ref) 
% est robuste à notre généralisation. Dans la pluspart des cas, sous un minimum d'hypothèses, le prior de référence est le prior de Jeffreys.
% On répond alors ici à la \textbf{question i}. 


% Pour contrer les limitations connues et nombreuses du prior de Jeffreys on doit alors se replonger dans le questionnement de l'objectivité voulue. Finalement, nombreuses sont les études ou les priors ``les plus obectifs possible'' ne sont pas tant ceux qui sont attendus.
% Trop complexes, trop peu simplement implémentable, ou trop peu informatifs donnant lieu à des posteriors impropres... Il faut dans le cas ou ces problèmes sont limitant à l'étude altérer l'objectivisme et accepter de la conditionner à la réalisation de son but.
% C'est sous cette conslusion que nous sommes venus à la réponse à la \textbf{question ii} telle qu'elle est apportée par le \cref{chap:constrained-prior}.
% Les contraintes que nous y proposons à incorporer au cadre des problèmes definissant les priors de référence sont pensées pour peu déformer l'objectivité, tout en amélirant la praticabilité.
% Il s'agit dans ce chapitre de résultats théoriques nouveaux qui viennent enrichir la définition de priors qui s'appuient sur le canvas des priors de référence.


% Enfin, pour parvenir aux implementations couteuses numériquement nous proposons dans une dernière contribution une méthode d'approximation du prior de référence. 
% Cette méthode apporte une réponse à la \textbf{question iii}, et est développée dans le \cref{chap:varp}.
% Comme dans le chapitre qui le précède, elle vient avec un nouveau coùt à l'objectivisme puisque (i) elle impose la parametrisation implicite du prior au travers d'un réseau de neurones, et (ii) elle ne promet pas et ne garantie pas une approxiation parfaite du meilleur prior (i.e., le ``plus objectif'') parmis ces priors implicitement paramétrés.
% Cependant, cette implémentation nouvelle dans sa forme et son scope, permet un nouveau pas dans le domaine de l'inférence bayésienne objective (ou en tout cas proche d'être objective) en pratique.
% Il s'agit en effet du dernier chainons manquant pour allier ensemble les réponses des précédentes qustions dans un cadre tractable.



% Nous concluont ainsi, en ayant observé que l'objectivisme a un coût sur l'usage et vice versa.
% Cependant, nous souhaitons insister sur la capacité des travaux conduits dans la partie I de ce manuscrit à allier au maximum ces deux idéaux, le plus souvent en cherchant à atteindre la praticité minimale en perdant le moins d'objectivité possible.




% \section{2 2}




Objectivism in the Bayesian workflow refers to the quest for an elusive ideal. % within the Bayesian workflow. 
In general, the literature agrees that objectivity is best approached when data and the information they provide are prioritized over any subjective incorporation. It is difficult to assert definitively whether reference priors are truly objective priors; nevertheless, their construction is explicitly designed to minimize the influence of subjective thoughts.

Thus, one can view an objective prior as an ideal, and the framework of reference priors offers a clue on a quantification about how closely a given prior approaches that ideal. In \cref{chap:ref-generalized}, we sought to extend the scope of this measure by proposing a generalization of its fundamental definition. %the foundational problem that defines reference priors. 
Our results demonstrated that the solution to this generalized problem remains robust. In most cases, under minimal assumptions, the reference prior remains equivalent to Jeffreys prior. This investigation provided our answer to \textbf{question i}.

Actually, to address the many known limitations of Jeffreys prior, it is necessary to question the amount of objectivity that is truly expected for the problem of interest. %revisit the notion of objectivity itself. 
In practice, it is often found that the ``most objective'' priors are not always the most desirable. They may be too complex, impractical to implement, or insufficiently informative leading, for example, to improper posteriors. When these issues become limiting, objectivism must be revisited and conditioned by the practical goals of the analysis. This is the conclusion we arrived at in \cref{chap:constrained-prior}, which responds to \textbf{question ii}.
In that chapter, we propose incorporating constraints into the reference prior framework. These constraints are carefully designed to preserve objectivity as much as possible, while improving practical usability. They represent novel theoretical contributions that enrich the definition of priors built upon the foundation of reference priors, but tailored for real-world applicability.

Finally, in response to \textbf{question iii}, we address the challenge of computational cost associated with reference prior implementation. In this contribution, developed in \cref{chap:varp}, we propose a method for approximating reference priors. As for the preceding chapter, this method introduces a new cost to the objectivity, namely (i) it imposes the implicit parameterization of the prior through a neural network, and (ii) it does not guarantee that this parameterization will perfectly approximate the optimal (i.e., ``most objective'') prior. %—it offers a promising practical advancement.
This approach, novel in both form and scope, represents a significant step toward enabling objective (or near-objective) Bayesian inference in practice. It also serves as the last link that connects the answers to the preceding questions within a tractable framework.

We therefore conclude that objectivity and usability are often in tension: quest for objectivity comes at a cost to practical application, and vice versa. However, we emphasize the capacity of the work presented in the \cref{part:ref-theory} of this manuscript to reconcile these two ideals as much as possible —--generally by aiming for minimal practicality with the least possible sacrifice of objectivity.








% Bien que nous n'ayons pas de déféition d'un prior parfaitement objetif, les soltions apportées 



% Cette solution théorique qui prend la forme d'un prior bien fixe directement les limites de l'emploi de la théorie. 











% Le théorie des priors de références




\section{On the contributions to the estimation of seismic fragility curves}\label{sec:concl:frags}

% L'estimation de courbes de fragilité sismique est un sujet largement étudié puisque c'est une problématique concrète et qui a un impact réel.
% Cela la rend tout auant critique. Dans cette thèse, nous nous sommes limités au cas d'étude du modèle probi-lognormal omniprésent lorsque les données observées sont binaire.
% Ce modèle est en effet suffisemant riche pour que la conduite de travaux à son sujet ait suffit à l'obtention de contributions riches à la problématique.

% Tout d'abord, la construction de prior compréhensivement tailorés pour l'estimation bayésienne de ce modèle était une piste inexplorée, que nous avons décidé d'investiguer sous le scope des prios de références. C'est ainsi que nous avons construit, calculé, et pleinement étudié le prior de Jeffreys pour ce modèle. Plus qu'apporter ainsi une reponse à la \textbf{question iv}, nous avons dans le \cref{chap:prem} d'autant plus démontré que ce prior présentait des résultat bien plus performant en terme d'efficacité et d'efficience que deux approches competitives sérieuses de la littérature. %Bien qu'il n'ait pas été designé par la performance 

% Malgré tout, nous avons étudié par la même occasion la vraisemblance de ce modèle, nous soulignons que les taux asymptotiques de cette derniere n'avaient pas été étudiés comme ceci auparavant.
% Cette étude nous a permis de définir le phénomène de dégénrescence, observable dans les estimations issues de nombreuses méthodes. Il s'agit d'un phénomène qui survient lorsque les données informent peu le modèle, et cela met en péril l'emploi du prior objectif.
% On a alors proposé une nouvelle construction de prior, en s'appuyant toujours sur la théorie des priors de référence, et en particulier sur certaines des contributions que nous avions menées dans la première partie du manuscrit. En effet, la limite imposée par la dégénérescence se rapporte à une des limitations envisagées en \textbf{question ii}. C'est alors dans le \cref{chap:constrained-frags} que nous mettons en oeuvre l'application de cette méthodologie au modèle probit-lognormal des courbes de fragilité, apportant une réponse à la \textbf{question v}.


% Enfin, nous avons conclu notre étude en cherchant à optimiser au mieux toutes les sources d'informations qui viennent à disposition de l'estimation des courbes de fragilité dans le modèle probit-lognormal.
% C'est ainsi qu'à l'information a priori nous avons ajouté dans le \cref{chap:doe} une optimisation de l'information issue des données via une planification d'experience.
% Cette méthodologie peut-être vue comme une méthodologie ''ultime'' d'estimation des courbes de fragilité en ce sens, d'autant que, appuyée par un prior toujours de référence, on sait en assurer une certine auditabilité chère au milieu d'application. Elle apporte alors une réponse qui va même au delà de la \textbf{question vi}.
% Cette méthode répond en effet à beaucoup de problématiques du sujet à savoir: la dégénérescence et les estimations difficiliement exploitable qu'elle induit sont rapidement effacées, l'estimation est rapidement robuste et le prior est oublié, le biais induit par la consideration du modèle probit-lognormal est étudié et rapidement atteint.
% En notre sens, nous apportons la meilleur des réponse à l'estimation de courbe de fragilité à peu de données en employant ce modèle, et proprosons une limite à partir de laquelle un modèle différent est à favoriser.



% \section{3 2}


The estimation of seismic fragility curves is a widely studied topic due to its concrete and impactful applications, making it both relevant and critically important. In this thesis, we focused on the well-established probit-lognormal model, commonly used when observed data about structural responses are binary. The study of this model %is a problem that is sufficiently rich for the conduction of wor
is a sufficiently rich topic to lead to significant contributions to our problem

% to serve as a meaningful case study, and our investigations within its scope have led to significant contributions.

First, the construction of comprehensively tailored priors for Bayesian estimation under this model had remained largely unexplored. We chose to address this gap through the scope of reference prior theory. Specifically, we derived, computed, and thoroughly analyzed the Jeffreys prior for the probit-lognormal model. This work, developed in \cref{chap:prem} does not only answer \textbf{quesion iv}, but it also demonstrated that this prior performs significantly better in terms of both efficiency and accuracy compared to two serious competing approaches from the literature.

In parallel, we conducted a detailed study of the likelihood function of the model. Notably, the asymptotic behavior of this likelihood had not been analyzed in this form before. This analysis led us to identify and formalize the phenomenon of degeneracy, a critical issue that arises when the data provide insufficient information to inform the model. 
Degeneracy is frequently observed in estimates from many methods and
undermines the reliability of objective priors, among others. In response, we proposed a new construction of the prior, still supported by reference prior theory, and based on some of the theoretical contributions made in the \cref{part:ref-theory} of this manuscript. The limitation introduced by degeneracy is directly related to the concerns raised in \textbf{question ii}. The application of this new methodology to the probit-lognormal model is developed in \cref{chap:constrained-frags}, thereby providing a concrete response to \textbf{question v}.

Finally, we concluded our study by aiming to optimally integrate all sources of information available for estimating fragility curves within the probit-lognormal framework. This way, in \cref{chap:doe}, we augmented prior information with an optimization of data-based information through a formal experimental design strategy. This methodology can be seen as a kind of “ultimate” approach to fragility curve estimation. Moreover, since it is used alongside a reference prior, it guarantees a level of auditability that is valued in the SPRA. This work provides an answer that extends beyond \textbf{question vi}. % not only addresses **Question vi**, but extends beyond it.
Indeed,
the proposed methodology tackles many of the central issues identified throughout this research: (i) degeneracy and the resulting unreliable estimates are quickly resolved; (ii) robustness is achieved early in the inference process; (iii) the prior information becomes negligible in front of  the data-based one; and (iv) the bias inherent to the probit-lognormal model is characterized and rapidly quantified. We believe this method offers the most effective approach to fragility curve estimation under conditions of limited data when using this model. Furthermore, we provide a practical threshold beyond which it becomes advisable to consider an alternative model.















% Bien qu'omniprésent, 












\section{General outlook}\label{sec:concl:general}


Finalement, plus que contribuer d'une part la théorie des priors de référence et d'autre part à l'estimation de courbes de fragilité sismique, nous avons dans cette thèse mis en lumière le lien entre ces deux mondes. Au final, les recherches sur chacuns de ces deux axes ce sont alimantées l'un l'autre.


Cette reflexion est essentielle puisqu'elle démontre à la fois l'intérêt d'avancer dans une voie  % Les travaux ercutant dans leur individualité sont loins d'être isolé et s'inscrivent dans un tout motivé, réel et startégique



Ici, les problèmes observés lors de


En vue d'un estimation objective des courbes de fragilité sismiques, le travail prime alors comme un tout.
Et nous dirions que nous avons proposé un cadre général pour à la fois construire des priors objectifs, à la fois les rendre 
Le tout







\section{Perspectives and connections with other fields}\label{sec:concl:perspectives}

De nombreuses perspectives se voient ouvertes pas les travaux menés dans cette thèese et les résultats obtenus.

D'une part sur l'approfondisssmeent théorique des priors de réféerences en analyse Bayésienne objective. 
Nous dirions que son cadre saurait encore être enrichi



Aussi, notre intrinsection conduite entre priors de réfférence et courbes de fragilté dans des cas concrêt a démontré comment cette théorie et ses méthodes sait constituer un point d'enchrage à l'explicabilité d'estimation dans un modèle. 



Enfin, concernant l'estimation de courbe de fragilité sismique, peu d'appronfissement du modèle probit-lognormal sont envisagé. Cependant, l'estimation selon d'autres cadres, d'autres datasets et/ou d'autres modèles reste ouverte et reste un cadre dans lequel une étude objective peut-être conduite. On peut citer...














\section{Final words}\label{sec:concl:final}


Nous n'avons pas dans cette thèse contruit le prior objecctif par essence, ni même estimé la corube de fragilité exacte de fait.
Cependant, en s'adressant à la fois séparément mais aussi très poreuseent à ces deux questions nous nous sommes approché au plus près possible de ces deux idéaux.
%Nous croyons que ce raprochement s'est fait en contribuant à la fois indépendament au domaine des courbes de fragilité 
% Nous croyons que
Nous concluons en appuyant que les méthodologies nouvelles apportées par cette thèse, ainsi que les résultats théoriques, formels, et expérimentaux qui les accompagnent démontre de leur capacité à intelligement balancer les raprochements aux deux idéaux de sorte à répondre pour le mieux à la problématique définie par le sujet de thèse.
% Ceci étant d'autant plus appuyé par le fait qu'une part substantielle de ces résultats a été publiée après peer reviewing.

% Nous insistons que tous nos résultats ont été longuement analysés et relus, et tous 
%De cette manière, ces résultats, dont la portée dans leur domaine a déjà ou est en cours d'être jugée par le peer-reviewing, ont une portée qui va au delà.Chacun de ces résultats
Pour chacun des résultats qui fait parti de ceux qui articulent cette conclusion, nous insistons sur la rigueur apporté à notre approche, de sorte à appuyer au mieux la percutance du travail à la problématique.
Chacune des contributions individuelle a été soumise, et est dans certain cas déjà validées, à la communauté.
Le sérieux de notre démarche et de ces derniers mots est ainsi supporté.









% \section{Outlook}


% \adjusmptc

\newpage




\appendix
\part*{Appendices}\label{part:appendix}
\addtocontents{toc}{\bigskip}
% \addcontentsline{toc}{part}{Appendix}

%\renewcommand{\thechapter}{\Alph{chapter}}

\chapter{Influence of the IM choice on fragility curves estimated using a reference prior}\label{app:chap:uncecomp}
%in a Bayesian framework based on reference prior



\begin{abstract}[\hspace*{-10pt}]
    This appendix is a postprint of the published work: \fullcite{van_biesbroeck_influence_2023}  % Ce chapitre reprend principalement les travaux publiés dans: 
\end{abstract}

\begin{abstract}
    The work presented in this appendix
    investigates the effect of the choice of the IM to Bayesian estimations of seismic fragility curves.
    % complements the one conducted in \cref{chap:prem}.
    As in the \cref{part:spra} of this manuscript, we study the probit-lognormal model of fragility curves with datasets for which the structural responses are limited to binary outcomes (i.e., failures or non-failures).
    We implement a Bayesian approach that is based on an objective prior resulting from an approximation of the Jeffreys prior density.
%
    Considering two different IMs (PGA vs. PSA), we highlight that a more correlated IM to the structural response is more likely to yield degenerate scenarios. 
    Such degeneracy compromises fragility curves estimations, either with Bayesian or classical frequentist methods.
    The consequences of these results on the estimates are thoroughly investigated on a case study, and compared for different approaches including ours.
    %  e effect of the correlation of the IM with the structural response on degenerate scenario
    % complicating its consideration.
    % The comparison between the use of the IMs is done from a thorough study of 
    % The same approaches are implemented to estimate seimsic fragility curves namely, we conduct a Bayesian estimation of the probit-lognormal model using an objective prior that is based on approximation of the Jeffreys prior density. This approach is compared with another prior of  
    We precise that 
    %While 
    the case study considered here is identical only in terms of geometry to the piping system presented in \cref{chap:frags-intro}. %, it has a different behavior. % in this work. 
    Indeed, a linear behavior of the structure is assumed here  to accentuate the phenomena we wish to highlight, by increasing the correlation between its response and the IMs.
    % not exactly the same as the piping system presented in \cref{chap:frags-intro}. % and studied in some other chapters. 
    % It is geometrically identical, however, we assume in this work that its behavior is linear elastic in order to accentuate the phenomena we wish to highlight, by increasing the level of correlation between its response and the IMs.
    %  this work, a linear behavior of the structure is assumed to increase the correlation between its response and the IM, in order to highlight phenomena of interest.
%
%
    % The piping system we consider here as a case study is geometrically the same as the one presented in Chapter 7. It is also subjected to the same seismic signals. However, we assume that its behavior is linear elastic in order to accentuate the phenomena we wish to highlight, by increasing the level of correlation between its response and the IMs.
\end{abstract}

\minitoc



% \section*{Foreword}
% \addcontentsline{toc}{section}{Foreword}


\section{Introduction}


Seismic fragility curves are key quantities of the Seismic Probabilistic Risk Assessment (SPRA) studies carried out on mechanical structures. They were introduced in the 1980s for safety studies of nuclear facilities (see e.g.
\cite{kennedy_probabilistic_1980,kennedy_seismic_1984,park_survey_1998,kennedy_risk_1999,cornell_hazard_2004}).
%\cite{Kennedy1980,KENNEDY198447, PARK1998, KENNEDY1999, Cornell2004}). 
They express the probability of failure of the mechanical structure conditional to a scalar value derived from the seismic ground motions ---called Intensity Measure (IM)--- such as the Peak Ground Acceleration (PGA) or the Pseudo-Spectral Acceleration (PSA) for a fixed frequency and damping. In practice, various data sources can be exploited to estimate fragility curves, namely: expert judgments supported by test data \citep{kennedy_probabilistic_1980,kennedy_seismic_1984,park_survey_1998,zentner_fragility_2017}, experimental data \citep{park_survey_1998,gardoni_probabilistic_2002,choe_closed-form_2007}, results of damage collected on existing structures that have been subjected to an earthquake \citep{shinozuka_statistical_2000,lallemant_statistical_2015,straub_improved_2008}  and analytical results given by more or less refined numerical models using artificial or real seismic excitations (see e.g. \cite{zentner_numerical_2010,wang_influence_2020,mandal_seismic_2016,wang_seismic_2018,wang_bayesian_2018,zhao_seismic_2020}). Parametric fragility curves were historically introduced in the SPRA framework because their estimates require small sample sizes. The probit-lognormal model has since become the most widely used model (see e.g. \cite{shinozuka_statistical_2000,lallemant_statistical_2015,straub_improved_2008,zentner_numerical_2010,wang_influence_2020,mandal_seismic_2016,wang_bayesian_2018,wang_seismic_2018,zhao_seismic_2020,ellingwood_earthquake_2001,kim_development_2004,mai_seismic_2017,trevlopoulos_parametric_2019,katayama_bayesian-estimation-based_2021}).
Several strategies can be implemented to fit the median, $\alpha$, and the log standard deviation, $\beta$, of the model. Some of them are compared in \cite{lallemant_statistical_2015} highlighting advantages and disadvantages.
When the data is binary ---i.e., when it indicates failure or not--- \citet{lallemant_statistical_2015} recommended maximum likelihood estimation (MLE). When the data are independent, the bootstrap technique can be used to obtain confidence intervals relating to the size of the sample considered (e.g. \cite{shinozuka_statistical_2000,zentner_numerical_2010,wang_influence_2020}). 

Among the various other methods not mentioned in this short introduction, the Bayesian framework has recently become increasingly popular in seismic fragility analysis (see e.g. \cite{gardoni_probabilistic_2002,wang_seismic_2018,katayama_bayesian-estimation-based_2021,koutsourelakis_assessing_2010,damblin_approche_2014,tadinada_structural_2017,kwag_computationally_2018,jeon_parameterized_2019,tabandeh_physics-based_2020}). 
It actually allows to solve irregularity issues encountered in the estimation of the parametric fragility curves. MLE-based methods can indeed lead to unrealistic or degenerate fragility curves such as unit step functions when the data availability is sparse. Those problems are especially encountered when resorting to complex and detailed modeling due to the calculation burden or when dealing with tests performed on shaking tables, etc. In earthquake engineering, Bayesian inference is often used to update probit-lognormal fragility curves obtained beforehand by various approaches, by assuming independent distributions for the prior values of $\alpha$ and $\beta$ such as log-normal distributions (see e.g. \cite{tadinada_structural_2017,kwag_computationally_2018,wang_seismic_2018,katayama_bayesian-estimation-based_2021,straub_improved_2008}).

This work follows the one presented in \cref{chap:prem}, which deals with Bayesian problems in which only few binary data are available. Using the reference prior theory as a support, the authors have proposed an objective approach to choose the prior and to simulate a posteriori fragility curves. This approach led to the Jeffreys prior and the authors have shown the robustness and advantages of the Jeffreys prior in terms of regularization (no degenerate estimations) and stability (no outliers of the parameters) for fragility curves estimation. Since this prior depends only on the characteristics of the ground motion ---the distribution of the IM of interest--- its calculation is then suitable for any equipment of an industrial installation subjected to the same seismic hazard. So, in this work, we are interested in the influence of the choice of the IM ---PGA vs. PSA--- on the convergence of the estimates, for a given magnitude (M) - source-to-site distance (R) scenario and a given mechanical structure.

The appendix is organized as follows. \Cref{uncIM:sec:pb} presents the statement of the problem from the Bayesian viewpoint. A review of the objective prior theory is presented in \cref{uncIM:sec:objprior}. The principal achievements of \cref{chap:prem} on which we rely are summarized in \cref{uncIM:sec:construction}. \Cref{uncIM:sec:tools} is dedicated to reviewing estimation tools and benchmarking metrics used to support comparisons with classical approaches of the literature. They are implemented in \cref{uncIM:sec:application} on a case study, a piping system. A conclusion is proposed in \cref{uncIM:sec:conclusion}.





   
\section{Bayesian problem} \label{uncIM:sec:pb}

    A probit-lognormal model is often used to approximate fragility curves.
    In this model the probability of failure given the IM takes the following form:
        \begin{equation} \label{uncIM:eq:Pfa}
            P_f(a)=\PP(\mbox{`failure'}|\text{IM}=a) = \Phi\left(\frac{\log a-\log\alpha}{\beta}\right) ,
        \end{equation}
    where $\alpha, \beta\in (0,+\infty)$ are the two model parameters and $\Phi$ is the cumulative distribution function of a standard Gaussian variable. In the following we denote $\theta=(\alpha,\beta)$. 
    In the Bayesian point of view  $\theta$ is considered as a random variable. Its probability density function is denoted by $\pi$ and called the prior density, it is supposed to be defined on a set $\Theta\subset (0,+\infty)^2$.
    
    The statistical model consists in the observations of independent realizations $(a_1,z_1),\dots,$ $(a_k,z_k)\in\cA\times \{0,1\}$, $\cA \subset (0,\infty)$, $k$ being the dataset size. For the $i$th seismic event, $a_i$ is its observed IM and $z_i$ is the observation of a failure ($z_i$ is equal to one if failure has been observed during the $i$th seismic event and it is equal to zero otherwise). % =\indic_{\mbox{\scriptsize  `failure'}}$.
    The joint probability density of the pair $(a,z)$ conditionally to $\theta$ has the form:
        \begin{equation}
            p(a,z|\theta) %= p(a|\theta)p(z|a,\theta)=
            = h(a)\ell(z|a,\theta) ,
        \end{equation}
    where $h(a)$ denotes the p.d.f. of the IM and $\ell(z|a,\theta)$ is the density of a Bernoulli distribution whose parameter (the probability of failure) depends on $a$ and $\theta$ as expressed by \cref{uncIM:eq:Pfa}.
    The product of the conditional distributions $\ell(z_i|a_i,\theta)$ is the likelihood of the model:
        \begin{equation} \label{uncIM:eq:likelihood}
            \ell_k(\mbf z^k|\mbf a^k,\theta) = \prod_{i=1}^k \ell(z_i|a_i,\theta)= \prod_{i=1}^k \Phi\left(\frac{\log{a_i}-\log{\alpha}}{\beta}\right)^{z_i}\left(1-\Phi\left(\frac{\log{a_i}-\log{\alpha}}{\beta}\right)\right)^{1-z_i}  ,
        \end{equation}
    denoting $\mbf a^k=(a_i)_{i=1}^k$, $\mbf z^k=(z_i)_{i=1}^k$. The posterior density of $\theta$  can be computed by Bayes theorem: %. The resulting posterior distribution is: 
        \begin{equation} \label{uncIM:eq:posterior}
            p(\theta|\mbf z^k,\mbf a^k)=\frac{\ell_k(\mbf z^k|\mbf a^k,\theta)\pi(\theta)}{\int_\Theta \ell_k(\mbf z^k|\mbf a^k,\theta')\pi(\theta') d\theta'}.
        \end{equation}

    We recall that the likelihood is said to be degenerate in some cases: (i) when only failures or only non-failures are observed, and (ii) when an IM threshold allows separating failures from non-failures among the observations (i.e., $\exists a\in\cA,\, \forall i,\, a_i<a\iff z_i=0$).
    The degeneracy is defined in \cref{chap:prem} (\cref{sec:PREM:degeneracy}). In those cases, the likelihood decay rates make both priors considered in this study yielding improper posteriors.
    


\section{Reference prior theory} \label{uncIM:sec:objprior}

    To choose a non-subjective prior, we consider as in \cref{chap:prem} a so-called reference prior. It consists in choosing the prior that maximizes the mutual information indicator $\sI^k$ which expresses the information provided by the data to the posterior, relatively to the prior. In other words, this criterion seeks the prior that maximizes the ``learning'' capacity from observations. We refer to the \cref{part:ref-theory} of the manuscript for more details. The mutual information indicator can be expressed as a function of the prior density: %is defined by:
    \begin{equation}
        \sI^k(\pi) =
        \int_{(\cA\times \{0,1\})^k} D(p(\cdot|\mbf z^k,\mbf a^k)||\pi)p(\mbf z^k,\mbf a^k)d\lambda^{\otimes k}(\mbf a^k,\mbf z^k),
         \label{uncIM:eq:defI}
    \end{equation}
    where $p(\mbf z^k,\mbf a^k) = \int_{\Theta} \ell_k(\mbf z^k|\mbf a^k,\theta' ) \prod_{l=1}^kh(a_l) \pi(\theta')d\theta'$ and the reference measure $\lambda$ is the product of the Lebesgue measure over $\cA$ and the discrete measure $\delta_0+\delta_1$ over $\{0,1\}$. %: we have $\int_{ \cA \times \{0,1\}} \psi(a,z)\lambda(da,dz) = \int_{\cal A} \psi(a,0) da+\int_{\cal A} \psi(a,1) da$ for any test function $\psi$.
The indicator in \cref{uncIM:eq:defI} is based on a divergence $D$ between the posterior and the prior densities, which is known to numerically express this idea of the information provided by one distribution to another one.
    This divergence can be the Kullback-Leibler divergence or a $\delta$-divergence, for instance (see \cref{chap:intro-ref} and \cref{chap:ref-generalized}).
    % \begin{equation}
    %     KL(p||\pi)=\int_{\Theta} p(\theta)\log\frac{p(\theta)}{\pi(\theta)} d \theta.
    %      \label{uncIM:eq:defKL}
    % \end{equation}

    A suitable definition of a reference prior is suggested  as the solution of an asymptotic optimization of this mutual information metric.
    It has been proved that, under some mild assumptions which are satisfied in our framework, the Jeffreys prior, whose density is defined by  
       \begin{equation}
        J(\theta)\propto\sqrt{|\det\cI(\theta)|} ,
         \label{uncIM:eq:jeff}
    \end{equation}
    is the reference prior, with $\cI$ being the Fisher information matrix:
    \begin{equation}
        \cI(\theta)_{i,j}
            = -\int_{\cA \times \{0,1\}} \ell(z|a,\theta)\partial^2_{\theta_i\theta_j} \log \ell(z|a,\theta)h(a) \lambda(da,dz).
    \end{equation}
    %The property $\cI(\theta)^k=k\cI(\theta)$ (with $\cI(\theta)=\cI(\theta)^1$) makes $J$ independent of $k$, as its definition only stands up to a multiplicative constant.
    The Jeffreys prior is already well known in Bayesian theory for being invariant by a re-parametrization of the statistical model.
    This property is essential as it makes the choice of the model parameters $\theta$ without any incidence on the resulting posterior.
    
\section{Jeffreys prior construction}  \label{uncIM:sec:construction}

    \subsection{Jeffreys prior calculation} \label{uncIM:sec:jeffcalc}
    
         In a first step, we compute the Fisher information matrix $\cI(\theta)$ in our model. 
        Here, $\theta=(\alpha,\beta)\in \Theta$ and 
            \begin{equation}
                \cI(\theta)_{i,j}= -\int_{\cA \times \{0,1\}} \ell(z|a,\theta)\partial^2_{\theta_i\theta_j} \log p(z|a,\theta) h(a)\lambda(da,dz)
            \end{equation}
        for $i,j\in\{1,2\}$, with $\theta_1=\alpha$, $\theta_2=\beta$, i.e.,
            \begin{align}
%            \nonumber  &
                \log \ell(z|a,\theta) %\\ &
                = z\log\Phi\left(\frac{\log a-\log\alpha}{\beta}\right) + (1-z)\log\left(1-\Phi\left(\frac{\log a-\log\alpha}{\beta}\right)\right).
            \end{align}
	From \cref{chap:prem}, the information matrix $\cI(\theta)$ is given by:
        \begin{equation}
            \cI(\theta)=\begin{pmatrix}
            \frac{1}{\alpha^2\beta^2}(A_{01} + A_{02}) & \frac{1}{\alpha\beta^3}(A_{11}+A_{12}) \\
            \frac{1}{\alpha\beta^3}(A_{11}+A_{12}) & \frac{1}{\beta^4}(A_{21}+A_{22})
        \end{pmatrix}  ,
        \end{equation}
with
           \begin{equation} \label{uncIM:eq:Aij}
        \begin{aligned}
            %A_{11} &= \int_\cA\Phi'(\gamma)d\PP_A(a) & A_{12} &= \int_\cA\log\frac{a}{\alpha}\Phi'(\gamma)d\PP_A(a) \\
            A_{01} &= \int_\cA\frac{\Phi'(\gamma(a))^2}{\Phi(\gamma(a))}h(a)da,
            & A_{02} &= \int_{\cA}\frac{\Phi'(\gamma(a))^2}{\Phi(-\gamma(a))}h(a)da,\\
            A_{11} &= \int_\cA\log\frac{a}{\alpha}\frac{\Phi'(\gamma(a))^2}{\Phi(\gamma(a))}h(a)da,
            & A_{12} &= \int_{\cA}\log\frac{a}{\alpha}\frac{\Phi'(\gamma(a))^2}{\Phi(-\gamma(a))}h(a)da,\\
            A_{21} &= \int_\cA\log^2\frac{a}{\alpha}\frac{\Phi'(\gamma(a))^2}{\Phi(\gamma(a))}h(a)da,
            & A_{22} &= \int_{\cA}\log^2\frac{a}{\alpha}\frac{\Phi'(\gamma(a))^2}{\Phi(-\gamma(a))}h(a)da,\\
        \end{aligned}
        \end{equation}
and $\gamma(a)=\beta^{-1}\log\frac{a}{\alpha}$.
        
        The Jeffreys prior is known to be improper in numerous common cases (i.e., it cannot be normalized as a probability). In the present case, its asymptotic behavior is computed for different limits of $\theta = (\alpha,\beta)$ in \cref{chap:prem}, which shows that it is indeed improper. 
        Regarding the posterior, it is improper when the likelihood is degenerate, and proper otherwise.
        %This characteristic is not an issue, as our work focuses on the posterior which is proper as proved in \cref{chap:prem}. This property is essential as MCMC algorithms would not make any sense if the posterior were improper.        

        \subsection{IMs and practical implementation} \label{uncIM:sec:practseism}
        
In this work we use $10^5$ artificial seismic signals generated using the stochastic generator defined in \cite{rezaeian_stochastic_2010} and implemented in \cite{sainct_efficient_2020} from 97 real accelerograms selected in the European Strong Motion Database for $5.5 \leq \text{M} \leq 6.5$ and $\text{R} < 20$ km. Enrichment is not a necessity in the Bayesian framework ---especially if a sufficient number of real signals is available--- but it allows comparisons with the reference method of Monte-Carlo for simulation-based approaches as well as comparative studies of performance. For instance, \cref{uncIM:fig:IM} shows that the synthetic signals have the same PGA distribution as the real ones as well as the PSA which is here calculated at 5~Hz for 1\% damping ratio (see \cref{uncIM:sec:application} for justification). Moreover, the asymptotic expansions provided in \cref{chap:prem} give complementary and essential insight into the Jeffreys prior. They evince that its behavior in $\alpha$ is similar to that of a log-normal distribution having the same median as that of the IM with a variance which is the sum of the variance of the IM and of a term which depends on $\beta$. \Cref{uncIM:fig:IM} illustrates also this result for the two IMs.

       \begin{figure}[!ht]
        \centering
        {\includegraphics[width=5cm]{figures/PREM/PGAjeff.pdf}}
        {\includegraphics[width=5cm]{figures/PREM/PSAjeff.pdf}}        
        \caption{Comparison of a sectional view of the Jeffreys prior density w.r.t. $\alpha$ (black) with the approximated distributions of real accelerograms via Gaussian kernel estimation (red), the histograms of the generated signals (blue) and the log-normal approximations (purple) for the PGA (left) and for the PSA (right).}
         \label{uncIM:fig:IM}
    \end{figure}



  


        
        In practice, due to the use of Markov Chain Monte Carlo (MCMC) methods to sample the \emph{a posteriori} distribution, the prior density must be evaluated (up to a multiplicative constant) many times in the calculations. Because of its computational complexity due to the integrals to be computed, we performed evaluations of the prior on an experimental design based on a fine-mesh grid of $\Theta$ (here $(0,+\infty)^2$) and to build an interpolated approximation of the Jeffreys prior density from this design. This strategy is more suitable for our numerical applications and very tractable because the domain $\Theta$ is only two-dimensional. \Cref{uncIM:fig:jeff_prior} shows plots of the Jeffreys prior densities. 
%To be precise, $500\times500$ prior values have been computed for $\alpha\in[10^{-5},10]$ and $\beta\in[10^{-3},2]$. A linear interpolation has been processed from these.
        
        \begin{figure}[!ht]
            \centering
            {\includegraphics[height=3.5cm]{figures/PREM/Jeff_prior_PGA-2.pdf}}\hspace*{0.5cm}
            {\includegraphics[height=3.5cm]{figures/PREM/Jeff_prior_PSA-1.pdf}}
            % [PGA]{\includegraphics[width=6cm]{figures/uncIM/Jeff_.pdf}}
            % [PSA]{\includegraphics[width=6cm]{figures/uncIM/Jeffreys_SA.pdf}}
            \caption{The Jeffreys priors calculated from PGA (left) and PSA (right) on subdomains of $\Theta=(0,+\infty)^2$.}
             \label{uncIM:fig:jeff_prior}
        \end{figure}

\section{Estimation tools, competing approaches and benchmarking metrics} \label{uncIM:sec:tools}

    In this section, we first present the Bayesian estimation tools and the Monte-Carlo reference method to which we refer to evaluate the relevance of the probit-lognormal model. Then, to evaluate the performance of the Jeffreys prior, we present two competing approaches that we implement. On the one hand, we apply the MLE method widely used in literature, coupled with a bootstrap technique. On the other hand, we apply a Bayesian technique implemented with the prior introduced by \citet{straub_improved_2008}. Performance evaluation metrics are then defined.

    \subsection{Fragility curves estimations via Monte-Carlo}
     \label{uncIM:sec:reference}
        We assume that a validation dataset $(\mbf a^{\mathrm{MC}},$ $\mbf z^{\mathrm{MC}})$ $ =$ $ ( (a_i^{\mathrm{MC}})_{i=1}^{N^{\mathrm{MC}}} $, $(z_i^{\mathrm{MC}})_{i=1}^{N^{\mathrm{MC}}})$ is available. For nonparametric estimations, good candidates are Monte-Carlo (MC) estimators which estimate the expected number of failures locally w.r.t. the IM. We first need to define a subdivision of the IM values and to estimate the failure probability on each of the sub-intervals. Regular subdivisions are not appropriate because the observed IMs are not uniformly distributed. We follow the suggestion by \citet{trevlopoulos_parametric_2019} to take clusters of the IM using K-means. 
        Given such $N_c$ clusters $(K_j)_{j=1}^{N_c}$, the Monte-Carlo fragility curve estimated at the centroids $(c_j)_{j=1}^{N_c}$ is expressed as
            \begin{equation} \label{uncIM:eq:refMC}
                P_f^{\mathrm{MC}}(c_j) = \frac{1}{n_j}\sum_{i,\,a_i^{\mathrm{MC}}\in K_j}z_i^{\mathrm{MC}}  , 
            \end{equation}
        where $n_j = {\rm Card}(i,\,a_i^{\mathrm{MC}}\in K_j)$ is the size of cluster $K_j$.
        An asymptotic confidence interval for this estimator can also be derived using its Gaussian approximation. It is accepted that a MC-based fragility curve is a reference curve because it is not based on any assumption.

    \subsection{Fragility curves estimations in the Bayesian framework}  \label{uncIM:sec:BayesFram}
        From \cref{uncIM:eq:posterior} \emph{a posteriori} samples of $\theta$ can be obtained by MCMC methods. We have implemented an adaptive Metropolis-Hastings (M-H) algorithm with Gaussian transition kernel and covariance adaptation \citep{haario_adaptive_2001}. Such an algorithm allows simulating from a probability density known up to a multiplicative constant. The \emph{a posteriori} samples of $\theta$ can be used to define credible intervals for the probit-lognormal estimates of the fragility curves.

    \subsection{Competing approaches for performance evaluation} \label{uncIM:sec:Competing}
    
        %\subsubsection{Multiple MLE by bootstrapping} \label{uncIM:sec:bootstrap}
            {\bf Multiple MLE by bootstrapping.}
            The best known parameter estimation method is the MLE, defined as the maximal argument of the likelihood derived from the observed data:
            \begin{equation} \label{uncIM:eq:MLE}
                \hat\theta^{\mathrm{MLE}}_k = \argmax_{\theta\in\Theta} \ell_k(\mbf z^k|\mbf a^k, \theta).
            \end{equation}
            A common method for obtaining a wide range of $\theta$  estimates is to compute multiple MLE by bootstrapping. Denoting the dataset size by $k$, bootstrapping consists in doing $L$ independent draws with replacement of $k$ items within the dataset. Those lead to $L$ different likelihoods from the $k$ initial observations, and so to $L$ values of the estimator which can be averaged. In the context of fragility curves, this method is widespread (see e.g. \cite{shinozuka_statistical_2000,lallemant_statistical_2015,gehl_influence_2015,baker_efficient_2015,wang_influence_2020}). The convergence of the MLE and the relevance of this method is stated in \cite{van_der_vaart_asymptotic_1992}. Nevertheless the bootstrap method is often limited by the irregularity of the results for small values of $k$ (see e.g. \cite{zentner_fragility_2017}). In this context, the $L$ values of $\theta$ are used to define confidence intervals for the probit-lognormal estimates of the fragility curves.
 
        
        %\subsubsection{Prior suggested by \citeauthor{Straub2008} \cite{Straub2008}} \label{uncIM:sec:posterioriSimul}
        {\bf Prior suggested by \citet{straub_improved_2008}.}
        This prior, called SK prior, is defined as the product of a normal distribution for $\ln(\alpha)$ and the improper distribution $1/\beta$ for $\beta$, namely its density is:
                \begin{equation} \label{uncIM:eq:Straubprior}
                    \pi_{SK}(\theta)\propto\frac{1}{\alpha\beta} \exp\Big( -\frac{(\log\alpha-\mu)^2}{2\sigma^2}\Big).
                \end{equation}
        In \cite{straub_improved_2008} the parameters $\mu$ and $\sigma$ of the log-normal distribution are chosen to generate a non-informative prior. For a fair comparison with the approach proposed in this appendix, we decided to choose $\mu$ and $\sigma$ being equal to the mean and the standard deviation of the logarithm of the IM, whether the PGA or the PSA. This choice is consistent with the fact that the Jeffreys prior is similar to a log-normal distribution with these parameters (see \cref{uncIM:fig:IM}). The prior density $\pi_{SK}(\theta)$ is plotted in \cref{uncIM:fig:Straubprior} for the two IMs.

            \begin{figure}[!ht]
                \centering
                {\includegraphics[width=5cm]{figures/PREM/SK_prior_PGA.pdf}}\hspace*{0.5cm}
                {\includegraphics[width=5cm]{figures/PREM/SK_prior_PSA.pdf}}                
                % [PGA]{\includegraphics[width=6cm]{figures/uncIM/SK.pdf}}
                % [PSA]{\includegraphics[width=6cm]{figures/uncIM/SK_prior_SA.pdf}}                
                \caption{Density of the prior suggested by \citet{straub_improved_2008} scaled on log-normal estimations of the PGA (left) and of the PSA (right) distributions.}
                 \label{uncIM:fig:Straubprior}
            \end{figure}

            An analysis of the posterior which results from SK prior is given in \cref{chap:prem}. It shows that the posterior is always improper. This statement jeopardizes the validity of \emph{a posteriori} simulations using MCMC methods if we consider the whole domain $\Theta=(0,+\infty)^2$. This issue is nevertheless manageable with the consideration of a truncation w.r.t. $\beta$.
            
            
    \subsection{Benchmarking metrics} \label{uncIM:sec:metrics}
    
        In order to evaluate the performance of the proposed approach, we consider two quantitative metrics which can be calculated for each of the methods described in \cref{uncIM:sec:Competing}.
    We consider the sample $(\mbf z^k,\mbf a^k) $. We denote by $a \mapsto P_f^{|\mbf z^k,\mbf a^k}(a)$\vspace*{-4pt} the random process defined as the fragility curve conditional to the sample (the probability distribution of $P_f^{|\mbf z^k,\mbf a^k}(a)$\vspace*{-4pt} is inherited from the \emph{a posteriori} distribution of $\theta$). For each value $a$ the $r$-quantile of the random variable $P_f^{|\mbf z^k,\mbf a^k}(a)$ is denoted by $q_r^{|\mbf z^k,\mbf a^k}(a)$.  
    We define:
    \begin{itemize}
            \item The quadratic error: %{\bf METTRE A JOUR AVEC $\frac{1}{A_{\rm max}}$}
                \begin{equation} \label{uncIM:eq:quaderror}
                    \cE^{|\mbf z^k,\mbf a^k} = \EE\big[\| P_{f}^{|\mbf z^k,\mbf a^k} - P_f^{\mathrm{ref}} \|_{L^2}^2|\mbf z^k,\mbf a^k \big] ,\qquad \| P\|_{L^2}^2 = \frac{1}{A_{\rm max}} \int_{0}^{A_{\rm max}} P(a)^2 da .
                \end{equation}
%                \begin{equation} \label{uncIM:eq:quaderror}
%                \hspace*{-0.3in}
%                    \cE^{Q|\mbf z^k,\mbf a^k} = \EE\big[\| P_{f}^{|\mbf z^k,\mbf a^k} - P_f^{\mathrm{MLE}} \|_{L^2}^2|\mbf z^k,\mbf a^k \big] = \int_{0}^{A_{\rm max}} \EE\big[ (P_{f}^{|\mbf z^k,\mbf a^k}(a) - P_f^{\mathrm{MLE}} (a))^2 |\mbf z^k,\mbf a^k \big] \frac{da}{A_{\rm max}}.
%                \end{equation}
    $P_f^{\mathrm{ref}}$ stands for non-parametric estimate of the fragility curve $P^{\text{MC}}_f$ derived using the method described in \cref{uncIM:sec:reference} from a validation dataset.  %the log-normal estimate of the fragility curve obtained by MLE with all the database available. We further check that this estimate is close to the reference curve obtained by MC.             
            \item  The $1-r$ square credibility width: % conditional width of the $1-r$ credible zone for the fragility curve:
            %; or the conditional width of the $1-r$ confidence interval for maximum likelihood estimators
                \begin{equation} \label{uncIM:eq:scaleerror}
                    \cW^{|\mbf z^k,\mbf a^k} = \|q_{1-{r/2}}^{|\mbf z^k,\mbf a^k} - q_{{r/2}}^{|\mbf z^k,\mbf a^k}\|_{L^2}^2  .
                    %= \int_{0}^{A_{\rm max}} ( q_{1-{r/2}}^{|\mbf z^k,\mbf a^k} (a) - q_{{r/2}}^{|\mbf z^k,\mbf a^k}(a))^2 \frac{da}{A_{\rm max}}  .
                \end{equation}
        \end{itemize}

    To estimate such variables, we simulate a set of $L$ %=5000$ 
    fragility curves $( P_f^{\theta_i|\mbf z^k,\mbf a^k})_{i=1}^L\vspace*{-4pt}$ where $(\theta_i)_{i=1}^L$ is a sample of the \emph{a posteriori} distribution of the model parameters obtained by MCMC. The   empirical quantiles $\hat q_r^{|\mbf z^k,\mbf a^k}(a)$  of $( P_f^{\theta_i|\mbf z^k,\mbf a^k}(a))_{i=1}^L$  are approximations of the quantiles $q_r^{|\mbf z^k,\mbf a^k}(a)$ of the random variable $P_f^{|\mbf z^k,\mbf a^k}(a)$.
    We derive\vspace*{-4pt}:
        \begin{itemize}
            \item The approximated quadratic error:
                \begin{equation} \label{uncIM:eq:quaderrorapprox}
                    \cE^{|\mbf z^k,\mbf a^k} \approx\frac{1}{ L}\sum_{i=1}^L\| P_{f}^{\theta_i|\mbf z^k,\mbf a^k} - P_f^{\mathrm{MLE}} \|_{L^2}^2.
                \end{equation}
            \item The approximated $1-r$ square credibility width: % conditional width of the $1-r$ credible zone for the fragility curve:
                \begin{equation} \label{uncIM:eq:scaleerrorapprox}
                    \cW^{|\mbf z^k,\mbf a^k} \approx \|\hat q_{1-{r/2}}^{|\mbf z^k,\mbf a^k} - \hat q_{{r/2}}^{|\mbf z^k,\mbf a^k} \|_{L^2}^2.
                \end{equation}
        \end{itemize}
        The normalized $L^2$ norms are normalized integrals over $a\in [0,A_{\rm max}]$ which are approximated numerically using Simpson's interpolation on a regular subdivision $0=A_0<\dots<A_p=A_{\rm max}$. In the forthcoming section we use   $A_0=0$, $A_{\rm max}=24\,\mathrm{m/s^2}$ for the PGA and $A_{\rm max}=50\,\mathrm{m/s^2}$ for the PSA with $p=200$.

        For the MLE with bootstrapping, we can define a conditional quadratic error as in \cref{uncIM:eq:quaderrorapprox} and conditional width of the $1-r$ confidence interval as in \cref{uncIM:eq:scaleerrorapprox} using a bootstrapped sample $(\theta_i)_{i=1}^L$.


\section{Numerical application} \label{uncIM:sec:application}

\subsection{Presentation of the piping system and its correlation with the IMs}

This case study concerns a piping system that is a part of a secondary line of a French Pressurized Water Reactor. This piping system was studied, experimentally and numerically, as part of the ASG program \citep{touboul_seismic_1999}.  \Cref{uncIM:fig:ASG} shows a view of the mock-up mounted on the shaking table Azalee of the EMSI laboratory of CEA Saclay whereas the Finite Element Model (FEM) ---based on beam elements--- is shown in \cref{uncIM:fig:ASG}-right. The latter has been implemented with the homemade FE code CAST3M~\citep{cea_cast3m_2019} and has been validated via  an experimental campaign.%seismic tests.

	\begin{figure*}[!ht]
		\centering		
		\includegraphics[width=4.8cm]{figures/intro-frags/ASG.jpg}
		\hspace{1cm}
		\includegraphics[width=2.3cm]{figures/intro-frags/ASG_FEM.pdf}
		\caption{(left) Overview of the piping system on the Azalee shaking table and (right) Mock-up FEM.}
		 \label{uncIM:fig:ASG}
	\end{figure*}

One end of the mock-up is clamped whereas the other is supported by a guide in order to prevent the displacements in the X and Y directions. Additionally, a rod is  placed on the top of the specimen to limit the mass displacements in the Z direction (see \cref{uncIM:fig:ASG}-right). In the tests, the excitation is only imposed in the X direction. For this study, the artificial signals are filtered by a fictitious $2\%$ damped linear single-mode building at $5$ Hz, the first eigenfrequency of the $1\%$ damped piping system. As failure criterion, we consider excessive out-of-plane rotation of the elbow located near the clamped end of the mock-up, as recommended in \cite{touboul_enhanced_2006}. The critical rotation considered here is equal to $1.6^{\circ}$. This is the level quantile $90\%$ of a sample of size $10^5$ of numerical simulations carried out assuming a linear behavior of the piping system. A linear behavior is considered to simply highlight the influence of the choice of IM. Indeed we use on the one hand the PGA and on the other hand the PSA of the initial set of synthetic signals (i.e not filtered signals), calculated at 5 Hz and 1\% damping ratio. 

For the two IMs, \cref{uncIM:fig:ref-ASG} shows the comparisons between the reference MC-based fragility curves $P_f^{\mathrm{MC}}$ and their probit-lognormal estimates $P_f^{\mathrm{MLE}}$, both estimated from a validation database of $10^5$ simulations results. In both cases, the probit-lognormal fragility curves are good approximations of the reference ones.

\begin{figure}[!h]
    \centering
     {\includegraphics[width=5cm]{figures/uncIM/ref_ASG_PGA.pdf}}
     {\includegraphics[width=5cm]{figures/uncIM/ref_ASG_PSA.pdf}}
    \caption{Reference fragility curves $P_f^{\mathrm{MC}}$ compared with $P_f^{\mathrm{MLE}}$ computed using the full dataset generated ($10^5$ items) for the PGA (left) and for the PSA (right).%The red crosses represent the observations.
    }
     \label{uncIM:fig:ref-ASG}
\end{figure}




\paragraph{Correlation of the structure's response with the IMs}
\Cref{uncIM:fig:scatterplots_PSA_PGA} shows that the PSA is clearly better correlated with the response of the structure than the PGA. 
This correlation is remarkable in \cref{uncIM:fig:ref-ASG} as well, where the reference fragility curves $P_f^{\text{MC}}$ (see \cref{uncIM:sec:reference}) are plotted. 
Indeed, the PSA is a more discriminating indicator of the state of the structure than the PGA. This results in a ``flatter'' fragility curve when the PGA is used. In other words, with the PSA, the probability is greater that its values correspond to failure probabilities close to 1 or 0. 
% Indeed,  when the IM is the PSA, its knowledge discriminates more the knowledge of the state of the structure, this results in a fragility curve that is more flat in the case of the PGA than in the case of the PSA.
% Therefore, the distribution of the PSA (see \cref{uncIM:fig:IM}) gives high probability to IM values where the probability that the structure fails is either close to $1$ either close to $0$.
As a consequence, random samples of PSAs are more likely to yield degenerate likelihoods, as shown in \cref{uncIM:fig:degeneracy-prob}.



% when the IM is the PSA, given the distribution of that IM (see \cref{uncIM:fig:IM}), 

% there is a large probability that sampled IM 


    \begin{figure}[!ht]
        \centering
         {\includegraphics[width=5cm]{figures/uncIM/cloud_PGA.pdf}}
         {\includegraphics[width=5cm]{figures/uncIM/cloud_PSA.pdf}}
        \caption{Scatter plots of the out-of-plane elbow rotation as a function of (left) PGA and (right) PSA for a linear seismic behavior of the piping system.}
         \label{uncIM:fig:scatterplots_PSA_PGA}
    \end{figure}

    \begin{figure}[!h]
        \centering
        \includegraphics[width=5.2cm]{figures/uncIM/degeneracy_proba.pdf}
        \caption{Proportion of degeneracy yielded when drawing randomly a data sample in the database, as a function of the size $k$ of the drawn sample.}\label{uncIM:fig:degeneracy-prob}
    \end{figure}
    
  
    
\subsection{Results and discussion}    
    
   \Cref{uncIM:fig:ASG-curves-PGA,uncIM:fig:ASG-curves-SA} present the results for the PGA and the PSA as IM respectively. \nameCrefs{uncIM:fig:ASG-curves-PGA}~\ref{uncIM:fig:ASG-curves-PGA}-(a) and (b) (resp. \namecrefs{uncIM:fig:ASG-curves-SA}~\ref{uncIM:fig:ASG-curves-SA}-(a) and (b)) show the metrics  $\cE^{|\mbf z^{k},\mbf a^{k}}$, $\cW^{|\mbf z^{k},\mbf a^{k}}$ for each of the three methods, for $1-r=95\%$, and for $k$ varying from $5$ to $100$. Their averages and confidence intervals are derived from $200$ draws of the datasets. \nameCrefs{uncIM:fig:ASG-curves-PGA}~\ref{uncIM:fig:ASG-curves-PGA}-(c) and \ref{uncIM:fig:ASG-curves-SA}-(c) present examples of fragility curves credibility (or confidence for the MLE) intervals for the three methods introduced in \cref{uncIM:sec:Competing} for $k = 30$, in comparison with $P_f^{\mathrm{MC}}$. 
   Those are computed from generated pairs $(\alpha,\beta)$ whose scatter plots are presented in \namecrefs{uncIM:fig:ASG-curves-PGA}~\ref{uncIM:fig:ASG-curves-PGA}-(d) and \ref{uncIM:fig:ASG-curves-SA}-(d).
   %Finally, figures~\cref{uncIM:fig:ASG-curves-PGA}-(d) and \cref{uncIM:fig:ASG-curves-SA}-(d) present examples of scatter plots of the pair $(\alpha , \beta)$ generated with the three methods.
   
   In addition, to get a better overview on the results, we also show in \cref{uncIM:fig:ASG_CoV} the coefficients of variation of the two parameters $(\alpha , \beta)$ as a function of both IM and $k$.
   
   Generally speaking, these results show, as expected, that when the IM is more correlated to structural response, the differences between the methods are less marked, although in detail there may be some small differences depending on the sample size. {\Cref{uncIM:fig:ASG_CoV} clearly shows that an IM more correlated to the response of the structure induces a lower variability of the estimate of the median of the probit-lognormal model. This is not the case for the log standard deviation whose estimate is affected by samples which are more degenerate with this kind of IM. %, namely which are partitioned into two disjoint subsets when classified according to IM values: the subset for which there is no failure and the other one for which there is failure, these two subsets possibly being present in the same sample or separately. 
   Degenerate samples affect
%    Such degeneracy affects 
   all methods to varying degrees but in all cases, the Jeffreys prior outperforms SK prior and MLE.}
   
  Whereas the SK prior is calibrated to look like the Jeffreys prior, \namecrefs{uncIM:fig:ASG-curves-PGA}~\ref{uncIM:fig:ASG-curves-PGA}-(d) and \ref{uncIM:fig:ASG-curves-SA}-(d) show that many outliers ---i.e., large values of the pair ($\alpha,\beta$)--- are simulated with the SK prior. These values explain that the credible intervals of the fragility curves and the metrics $\cE^{|\mbf z^{k},\mbf a^{k}}$ and $\cW^{|\mbf z^{k},\mbf a^{k}}$ are larger with the SK prior. This observation is both supported by \cref{uncIM:fig:ASG_CoV} and by the calculations provided in \cref{chap:prem} regarding the asymptotic rates in $\beta$. Indeed, in \cref{chap:prem} is shown a better convergence of the Jeffreys prior toward $0$ when $\beta\to\infty$. This better  asymptotic behavior results in posteriors which happen to give a lower probability to outlier points  ---phenomenon particularly noticeable when the dataset is small--- as well as to the weight of the likelihood within the posterior.
  
  Although the intervals compared ---that of the Bayesian framework and that of the MLE--- are not of the same nature ---credible interval for the first \emph{versus} confidence interval for the second--- these results clearly illustrate the advantage of the Bayesian framework over the MLE for small samples. Indeed, irregularities appear in the MLE method that are characterized by null estimates of $\beta$, which result (i) in important coefficient of variation of $\beta$  (\cref{uncIM:fig:ASG_CoV}) and (ii) in ``vertical'' confidence intervals on fragility curves estimations (\namecrefs{uncIM:fig:ASG-curves-PGA}~\ref{uncIM:fig:ASG-curves-PGA}-(c) and \ref{uncIM:fig:ASG-curves-SA}-(c)). {When few failures are observed, some samples ---both initial and bootstrapped samples--- are degenerate, as explained earlier.} As no prior is considered in the MLE-based approach, the likelihood can then be easily maximized with $\beta=0$. In \cref{chap:prem}, it is proven that such scenarios result in degenerate likelihood. This last statement is perceived better through \namecrefs{uncIM:fig:ASG-curves-PGA}~\ref{uncIM:fig:ASG-curves-PGA}-(d) and \ref{uncIM:fig:ASG-curves-SA}-(d). The zero-degenerate $\beta$ values that result from the MLE appear clearly. This leads to a confidence interval generally larger than the credible intervals, except for very small values of $k$ ($k \simeq 20$) when the IM is well correlated with the response of the structure, i.e. with the PSA. Indeed, with a perfect IM - which only exists if we know the structural response itself - the fragility curve is degenerate and has the form of a unit step function. The associated confidence interval is of null size, since in this case it does not require any sample to estimate the fragility curve. So, although apparently better, such confidence intervals are meaningless since they are based on degenerate estimates.
  

  
%\newpage 
          
    \begin{figure}[!h]
        \centering
        {\includegraphics[width=5cm]{figures/uncIM/err_quadra_ASG_PGA.pdf}}\ %
        {\includegraphics[width=5cm]{figures/uncIM/err_cred_ASG_PGA.pdf}} \\
        \makebox[5cm]{(a)}\ \makebox[5cm]{(b)}\\
        {\includegraphics[width=5cm]{figures/uncIM/curves_ASG_PGA.pdf}}\ %           
        {\includegraphics[width=5cm]{figures/uncIM/scatter_ASG_PGA.pdf}}\\
        \makebox[5cm]{(c)}\ \makebox[5cm]{(d)}
         \caption{(a) average values of $\cE^{|\mbf z^k,\mbf a^k}$ (dashed lines) surrounded by their $95\%$-confidence intervals; (b) average values of $\cW^{|\mbf z^k,\mbf a^k}$ (dashed lines) surrounded by their $95\%$-confidence intervals; (c) %the dashed lines plot the fragility curve estimates and 
         the shaded areas show the $95\%$ credible (for Bayesian estimation) or confidence (for MLE) intervals resulting from a total of $5000$ simulations of $\theta$ using the three methods considered for $k = 30$; (d) scatter plots of the generated $\theta$ to estimate the fragility curves for the three methods ($1000$ points from the $5000$ $\theta=(\alpha,\beta)$ generated are plotted); the green cross in (d) plots $\theta^{\mathrm{MLE}}$ which is derived from the validation dataset; the green line in (c) refers to $P^\text{MC}_f$, plottd along with its $95\%$ intervals (dashed lines). Here the PGA is used as IM.}
           \label{uncIM:fig:ASG-curves-PGA}
    \end{figure}

%\newpage             
        
    \begin{figure}[!h]
        \centering
        {\includegraphics[width=5cm]{figures/uncIM/err_quadra_ASG_PSA.pdf}}\ %
        {\includegraphics[width=5cm]{figures/uncIM/err_cred_ASG_PSA.pdf}} \\
        \makebox[5cm]{(a)}\ \makebox[5cm]{(b)}\\
        {\includegraphics[width=5cm]{figures/uncIM/curves_ASG_PSA.pdf}}\ %           
         {\includegraphics[width=5cm]{figures/uncIM/scatter_ASG_PSA.pdf}}\\
         \makebox[5cm]{(c)}\ \makebox[5cm]{(d)}\\[-8pt]
         \caption{Same as in \cref{uncIM:fig:ASG-curves-PGA}, but here the PSA is used as IM.}
           \label{uncIM:fig:ASG-curves-SA}
    \end{figure}


    As proven in \cref{chap:prem}, in the Bayesian context, the same degenerate samples also produce degeneracy but ``less marked'' than for MLE, as the phenomena is regularized by the prior distribution. %, consequently to the regularity brought by the prior. 
    As the results show, this affects SK prior more than Jeffreys one. This is observed, in particular, when the IM is very well correlated with the response of the structure since, in this case, it is more probable to obtain this kind of degenerate samples. Therefore, when $k < 40$, the credible intervals are slightly larger with the PSA than with the PGA. This is also confirmed by the results shown in \cref{uncIM:fig:ASG_CoV}. So, counter-intuitively, when very few data are available, a less well-specified problem from the point of view of the choice of the IM leads to better convergences of the estimates, because it produces fewer degenerate samples. However, this remains confined to very small sample sizes and therefore cannot be considered representative. Note that, in all cases, the Jeffreys prior outperforms SK prior and MLE.


%\newpage 

    \begin{figure}[!h]
        \centering
        {\includegraphics[width=5cm]{figures/uncIM/coeff_alpha_PGA.pdf}}\ %
        {\includegraphics[width=5cm]{figures/uncIM/coeff_alpha_PSA.pdf}} \\
        \makebox[5cm]{(a)}\ \makebox[5cm]{(b)}\\
        {\includegraphics[width=5cm]{figures/uncIM/coeff_beta_PGA.pdf}}\ %           
         {\includegraphics[width=5cm]{figures/uncIM/coeff_beta_PSA.pdf}}\\
         \makebox[5cm]{(c)}\ \makebox[5cm]{(d)}\\[-8pt]
         \caption{Average coefficient of variation of $\alpha$ (a-b) and of $\beta$ (c-d) for the PGA and the PSA. For each value of $k$, $200$ samples of size $k$ have been used to compute the average values of the resulting $200$ coefficients of variation from $5000$ estimates of $\theta$ each.}
           \label{uncIM:fig:ASG_CoV}
    \end{figure}

% \clearpage

% \newpage

\section{Conclusion}  \label{uncIM:sec:conclusion}

Assessing the seismic fragility of Structures and Components when little data is available is a daunting task. The Bayesian framework is known to be efficient for this kind of problems. Nevertheless, the choice of the prior remains tricky because it has a non-negligible influence on the  \emph{a posteriori} distribution and therefore on the estimation of any quantity of interest linked to the fragility curves.

Using the reference prior theory to define an objective prior, we have derived the Jeffreys prior for the probit-lognormal model with binary data which indicate the state of the structure (e.g. failure or non-failure). In doing so, this prior is completely defined, it does not depend on an additional subjective choice.

Since this prior depends only on the characteristics of the ground motion ---the distribution of the IM of interest--- its calculation is then suitable for any equipment of an industrial installation subjected to the same seismic hazard. So, in this work, we were interested in the influence of the choice of the IM ---PGA vs. PSA--- on the convergence of the estimates, for a given (M,R) seismic scenario and a given mechanical structure, namely a piping system.

The results show, as expected, that when the IM is more correlated with the structural response, the differences between methods are less marked, although in detail there may be some small differences depending on sample size, due to possible degenerate samples. {These results testify to the fact that an IM more correlated to the response of the structure essentially induces a lower variability of the estimate of the median of the probit-lognormal model. This is not the case for the log standard deviation whose estimate is affected by samples which are more degenerate with this kind of IM. Such degeneracy affects all methods,} however, in all cases, the Jeffreys prior outperforms the classical approaches of the literature both in terms of regularization (absence of
degenerate estimation) and stability (absence of outliers when sampling the \emph{a posteriori} distribution of the parameters).


\chapter{Design of experiments for a low fidelity model of seismic fragility curves}\label{app:chap:ESAIM}



\begin{abstract}[\hspace*{-10pt}]
    This appendix is a postprint of the accpeted work: \fullcite{van_biesbroeck_design_2025}  % Ce chapitre reprend principalement les travaux publiés dans: 
\end{abstract}

\begin{abstract}
    %Seismic fragility curves are key quantities of interest for Seismic Probabilistic Risk Assessment studies. They express the probability of failure of a mechanical structure of interest conditional to a scalar value derived from the ground motion signal coined Intensity Measure. 
%In the literature, Bayesian approaches have emerged to 
In this appendix, we propose 
an efficient modeling for the estimation of seismic fragility curves in the Bayesian context that is not based on the common probit-lognormal model. % of fragility curves.
%enable their estimation within the difficult context of limited data availability.  Yet, the probit- lognormal modeling over which most of them are based requires the use of computationally expensive {Markov chain Monte Carlo} methods for providing Bayesian estimators.  %
% In this work, we propose an efficient modeling for the estimation of fragility curves in the Bayesian context, 
We implement instead a low fidelity model of the structure's response to the ground motion signal and an objective prior. In this work, we do not limit the knowledge about the state of the structure to a  binary outcome.
The analytical expression of our modeling {allows} fast generation of estimates. Also, the representative bias arisen by the modeling choice is partly handled with a  design of experiments methodology. Finally, our approach is evaluated on a real case study, and the results highlight its ability to overcome the irreducible bias when coupled with the design of experiments we propose.
\end{abstract}


\minitoc

\section{Introduction}




The probabilistic seismic risk assessment framework (SPRA) introduced in the 1980s for the nuclear industry is based on the estimation of seismic fragility curves, for the structures and components (SCs) of interest \citep{kennedy_probabilistic_1980,kennedy_seismic_1984,park_survey_1998,kennedy_risk_1999,cornell_hazard_2004}. These curves are defined as the conditional probability that an engineering demand parameter (EDP) -- such as the interstory drift ratio -- exceeds a limit threshold, given a scalar value derived from the seismic ground motion and called intensity measure (IM). The IM can be for instance the peak ground acceleration (PGA) or a pseudo-spectral acceleration (PSA) evaluated for a given frequency and damping ratio \citep{ciano_role_2020,sainct_efficient_2020,ciano_novel_2022}. As explained in \cite{cornell_hazard_2004}, it is therefore assumed that the seismic hazard, on a given site, can be reduced to such a single indicator.


Practitioners have several data sources at their disposal to estimate fragility curves, namely: expert judgments supported by test data \citep{kennedy_probabilistic_1980,kennedy_seismic_1984,park_survey_1998,zentner_fragility_2017}, experimental data \citep{park_survey_1998,gardoni_probabilistic_2002,choe_closed-form_2007}, results of damage collected on existing structures that have been subjected to earthquakes \citep{shinozuka_statistical_2000,lallemant_statistical_2015,straub_improved_2008} as well as analytical results given by more or less refined numerical models using synthetic or real seismic excitations \citep{zentner_numerical_2010,wang_influence_2020,mandal_seismic_2016,wang_bayesian_2018,wang_seismic_2018,zhao_seismic_2020}. Over the years, many methods have been developed to estimate these curves \citep{shinozuka_statistical_2000,lallemant_statistical_2015,zentner_fragility_2017}. Nowadays, even though machine learning techniques are becoming very popular  \citep{park_rapid_2014,seo_use_2013,gidaris_kriging_2015,wang_seismic_2018,sainct_efficient_2020,gauchy_estimation_2022}, parametric fragility curves historically introduced in the SPRA framework are ubiquitous in practice and the log-normal model is the most widely used model due to its proven ability to handle limited data \citep{shinozuka_statistical_2000,lallemant_statistical_2015,straub_improved_2008,zentner_numerical_2010,wang_influence_2020,mandal_seismic_2016,hariri-ardebili_probabilistic_2016,wang_bayesian_2018,wang_seismic_2018,zhao_seismic_2020,ellingwood_earthquake_2001,kim_development_2004,mai_seismic_2017,trevlopoulos_parametric_2019,katayama_bayesian-estimation-based_2021}.

Different strategies can be implemented to estimate the parameters that define the fragility curve in the log-normal model. Among these we distinguish the Bayesian framework
%Among the different strategies that can be implemented to estimate the parameters that define the fragility curve in the log-normal model we distinguish the Bayesian framework 
\citep{gardoni_probabilistic_2002,wang_seismic_2018,katayama_bayesian-estimation-based_2021,koutsourelakis_assessing_2010,damblin_approche_2014,tadinada_structural_2017,kwag_computationally_2018,jeon_parameterized_2019,tabandeh_physics-based_2020,lee_efficient_2023}. 
This framework is interesting because it allows to solve the irregularity issues encountered %\sout{when the data availability is sparse} 
{when few data are available}. This occurs with the widely used maximum likelihood estimation coupled with a bootstrap technique to estimate a confidence interval when the data are binary, that is, when they represent the failing or non-failing state of the structure (as in the studies conducted in the \cref{part:spra} of this manuscript). In practice, binary data are encountered 
%those problems are especially encountered when resorting to complex and detailed modeling due to the calculation burden or 
when dealing with tests performed on shaking tables for instance.

In earthquake engineering, within the SPRA framework, Bayesian inference is often used to update log-normal fragility curves obtained beforehand by various approaches, by assuming independent distributions for the prior values of the parameters, such as log-normal distributions for instance \citep{tadinada_structural_2017,kwag_computationally_2018,wang_seismic_2018,katayama_bayesian-estimation-based_2021,straub_improved_2008}. In this thesis, based on the reference prior theory, we proposed the use of an objective prior, in order to remove any subjectivity that could legitimately lead to inevitable open questions on the influence of the \emph{a priori} on the quantities of interest. In all these approaches, the use of Markov chain Monte Carlo (MCMC) methods is nevertheless necessary to sample the \emph{a posteriori} distribution of the parameters, which can prove cumbersome to implement, particularly if we want a rapid first estimate of a fragility curve with limited data. 

%To circumvent this, we propose in this work an effective approach for estimating fragility curves, which avoids the use of the MCMC method, and which is based on the use of a low fidelity model to represent the structural response. Since the low-fidelity model is a linear model, we also propose an experimental design algorithm to minimize representation bias.\\

We circumvent this problem in this work by proposing an effective approach for the estimation of fragility curves, which avoids the use of the MCMC method. We rely on a low-fidelity linear model between the logarithm of the Engineering Demand Parameter and the one of the IM \citep{lallemant_statistical_2015,hariri-ardebili_probabilistic_2016,zentner_fragility_2017,ghosh_seismic_2020}. Supported by the Bayesian framework, our model benefits from a fully analytical form; that former allows an efficient implementation and a solution for fast generation of estimates with limited data. The reliability of the Bayesian scheme w.r.t.{ }its prior choice is answered as well with the derivation of an objective prior derived on the basis of the reference prior theory (see the \cref{part:ref-theory} of this manuscript). Finally, since the low-fidelity model is a linear model, we also propose a sequential planning of experiments strategy to minimize the representation bias. 
The design we suggest relies on the maximization of the information brought by the observation of a new data item onto the posterior distribution. That one is measured through global sensitivity indices described in~\cite{da_veiga_global_2015}.
%arisen from global sensitivity analysis
%that serves as measure of the information transmission by a new data item.


The remainder of this appendix is organized as follows:
the statement of our low-fidelity modeling strategy for the fragility curves estimation in a Bayesian framework is presented in \cref{lowdoe:sec:modeling}.
After a brief review devoted to global sensitivity analysis, we describe in \cref{lowdoe:sec:PE} our design for a sequential planning of experiments, taking the global sensitivity indices as a support.
\Cref{lowdoe:sec:application} is dedicated to the implementation of our methodology on a case study from the nuclear industry. Finally, \cref{lowdoe:sec:discussion} which precedes the conclusion offers a discussion on the performance of our method.
%a discussion about the performances.






\section{Low fidelity model for fragility curves}\label{lowdoe:sec:modeling}
    \subsection{Linear regression model}
    
    The fragility curve which we seek to estimate is defined by
\begin{equation}
    P_f(a) = \PP(\mathrm{EDP}>C\,|\mathrm{IM}=a).
\end{equation}
Up to a refinement of a multiplicative constant in the definition of the engineering demand parameter (EDP), $C$ can be supposed to be equal to $1$ in what follows.
The EDP is supposed to be correlated with the intensity measure (IM) as follows
\begin{equation}
    \log \mathrm{EDP} = \rho\log \mathrm{IM} + \epsilon ,
\end{equation}
where the random variable $\epsilon$ follows 
the distribution $\cN(\mu,\sigma^2)$.
Here $\rho$ is supposed unknown, as well as $\mu$ and $\sigma$. 
%Conditionally to a set of different values of the IM, the resulting values of the EDP are supposed to be independent. 




    
    \subsection{Likelihood}\label{lowdoe:sec:likelihood}

We have at our disposal a data-set composed by the tuples $(a_i,y_i)_{i=1}^k$, $a_i\in\cA\subset(0,\infty)$ denoting the measured IM from {the $i$-th seismic ground motion} signal and $y_i\in\cY\subset(0,\infty)$ denoting the structure's EDP. Conditionally to the parameter $\theta=(\rho,\mu,\sigma)$, we suppose the observations to be independent and identically distributed. As $a_i$ follows a distribution assumed to admit a density $a\mapsto p(a)$ w.r.t.{ }the Lebesgue measure and to be independent of $\theta$, the distribution of the $y_i$ is known conditionally to $(a_i,\theta)$:
    \begin{equation}
        \log y_i|a_i,\theta \sim \cN(\rho\log a_i+\mu,\sigma^2)    .
    \end{equation}
The likelihood for the parameter $\theta$ is therefore:
    \begin{equation}
        \ell_k^0(\hatmbf y,\hatmbf a|\theta) = \prod_{i=1}^k \frac{1}{\sqrt{2\pi\sigma^2}}\exp\left( -\frac{(\hat y_i-\rho\hat a_i-\mu)^2}{2\sigma^2} \right)p(\hat a_i),
    \end{equation}
where $\hat y_i$ (resp. $\hat a_i$) denotes $\log y_i$ (resp. $\log a_i$) and $\hatmbf y$ (resp. $\hatmbf a$) denotes the vector $(\log y_i)_{i=1}^k$ (resp. $(\log a_i)_{i=1}^k$).

%for example, they are primarily introduced to separate the parameters and thus to make the posterior distribution
{
This likelihood introduces a challenge due to the lack of clear separation among the three parameters that constitute $\theta$. Within the Bayesian framework, which we develop later, this challenge could result in a posterior distribution that is hardly tractable. We address this issue by introducing the following quantities:
}
%To benefit from a likelihood that separates better the parameters and so that would make a posterior distribution more tractable later, we consider the following quantities: 
%To benefit from a more tractable likelihood, we consider the following quantities derived from the data:
    \begin{equation}
        \rho_k = \frac{\Cov_k(\hatmbf y,\hatmbf a)}{\Var_k\hatmbf a}, \quad
        \mbf z = \tilde P_{\mbf a}^{\top}(\hatmbf y - \rho_k\hatmbf a) ,
    \end{equation}
where $\Cov_k$, $\Var_k$ respectively denote the empirical covariance and variance, and the matrix $\tilde P_{\mbf a}$ is defined in \cref{lowdoe:app:sec:DiagUa}.
%{$\rho_k$ is the natural slope estimator, and $\mbf z$ is the normalized addition of the residuals and the bias.} 
Their conditional distributions are given by:
    \begin{align}
        &  \rho_k|\mbf a,\theta \sim \cN\left( \rho,\frac{\sigma^2}{k\Var_k\hatmbf a} \right), \\
        &   \mbf z | \mbf a, \theta \sim  \cN(\mu\tilde P_{\mbf a}^{\top}\mbf 1,\sigma^2D);\quad D=\diag\left(1,\dots,1,\frac{\hatmbf a^{\top}\hatmbf a}{k\Var_k\hatmbf a}\right), %\cN(\mu\mathbf{1}, \sigma^2 U_{\mbf a})
    \end{align}
where $\mbf a$ is the vector $(a_i)_{i=1}^k$, $\mbf 1$ denotes the vector of $\RR^k$ composed only of ones, and $\diag(\lambda_1,\dots,\lambda_{k-1}) $ refers to the diagonal matrix of $\RR^{(k-1)\times(k-1)}$ whose diagonal coefficients are the $(\lambda_i)_{i=1}^{k-1}$. 
%matrix $U_{\mbf a}$ is given by
% \begin{equation}
%     U_{\mbf a} = I - \frac{\hatmbf a(\frac{1}{2}\hatmbf a - \overline{\hatmbf a})^{\top}  }{k\Var_k\hatmbf a} - \frac{(\frac{1}{2}\hatmbf a - \overline{\hatmbf a}) \hatmbf a^{\top}  }{k\Var_k\hatmbf a}\ ;\quad \overline{\hatmbf a} = \frac{1}{k}\sum_{i=1}^k \log a_i.
% \end{equation}

Let us denote by $(w_i)_{i=1}^k$ the orthogonal columns of the matrix $P_{\mbf a}$ defined in \cref{lowdoe:app:sec:DiagUa}. Noticing $\rho_k = w_k^{\top}\hatmbf y$ and $\mbf z$ is spanned by the $(w_{i})_{i=1}^{k-1}$, we deduce that $\mbf z$ and $\rho_k$ are independent conditionally to $(\mbf a, \theta)$, and that the knowledge of $(\mbf z,\rho_k,\mbf a)$ is equivalent to the one of $(\mbf y,\mbf a)$ or $(\hat{\mbf y},\hat{\mbf a})$.
Thus, the likelihood issued from the observation of $(\mbf z, \rho_k, \mbf a)$ is 
\begin{equation}\label{lowdoe:eq:likelihood}
     \ell_k(\mbf z,\rho_k,\mbf a|\theta) = p(\mbf a)\frac{\|\hatmbf a-\overline{\hatmbf a}\|}{\sqrt{2\pi\hatmbf a^{\top}\hatmbf a}\sigma}\exp\left(-\frac{(z_{k-1}-\mu \sqrt{k})^2}{2\sigma^2\frac{\hatmbf a^{\top}\hatmbf a}{\|\hatmbf a-\overline{\hatmbf a}\|^2}}\right)\exp\left({-\frac{(\rho_k-\rho)^2}{2\sigma^2/(k\Var_k\hatmbf a)}}\right)\prod_{i=1}^{k-2}\frac{1}{\sqrt{2\pi}\sigma}\exp\left(-\frac{z_i^2}{2\sigma^2}\right),
\end{equation}
where $\overline{\hatmbf a} = \frac{1}{k}\sum_{i=1}^k \log a_i$.


    \subsection{Prior and posterior}\label{lowdoe:sec:priorposterior}
%


Within a Bayesian context, the parameter of interest $\theta$ is itself a random variable taking values in a space $\Theta\subset\RR^2\times(0,\infty)$ and following a distribution called the prior.
We take as a support the reference prior theory (see the \cref{part:ref-theory} of this manuscript) to justify the choice of the Jeffreys's prior for $\theta$, derived from the likelihood expressed in \cref{lowdoe:eq:likelihood}.
Conditionally to $\mbf a$, that former is a Gaussian density, making the associated Fisher information matrix being:
       \begin{equation}    
        \cI(\theta) = \int_{\cA^k}\left(\begin{array}{ccc}
             k\frac{\|\hatmbf a - \overline{\hatmbf a}\|^2}{\sigma^2\hatmbf a^{\top}\hatmbf a}&  0&0\\
             0&k\frac{2}{\sigma^2}&0 \\
             0&0& \frac{k\Var_k\hatmbf a}{\sigma^2}
        \end{array}\right) \prod_{i=1}^kp(a_i)d a_i.
    \end{equation}
The Jeffreys' prior being defined as the one whose density $J$ w.r.t.{ }the Lebesgue measure is proportional to $\sqrt{|\cI(\theta)|}$, we obtain
\begin{equation}
    J(\theta)\propto \frac{1}{\sigma^3}.
\end{equation}

Finally, the posterior distribution of $\theta$ is given by its density, which is proportional to the product of the likelihood (from \cref{lowdoe:eq:likelihood}) with the prior:
    \begin{equation}\label{lowdoe:eq:posterior}
        p(\theta|\mbf z, \mbf a,\rho_k) \propto \frac{1}{\sigma^{k+3}} \exp\left({-\frac{\sum_{i=1}^{k-2}z_i^2}{2\sigma^2}}\right) \exp\left({-\frac{k\|\hatmbf a-\overline{\hatmbf a}\|^2}{\hatmbf a^{\top}\hatmbf a}\frac{(z_{k-1}k^{-1/2}-\mu)^2}{2\sigma^2}}\right)
            \exp\left({-\frac{(\rho_k-\rho)^2}{2\sigma^2/k\Var_k\hatmbf a}}\right).
    \end{equation}
We recognize the above as a product of square inverse gamma distributions. More precisely, $\sigma^{-2}$ follows a gamma distribution, and $\mu$ and $\rho$ follow independent Gaussian distributions conditionally to $\sigma$.

This posterior allows to elucidate the distribution of what expresses the fragility curve: 
\begin{equation}
    P_f(a)|\theta \sim \PP( \hat y>0|\hat a,\theta) = \Phi\left(\frac{\rho\log a+\mu}{\sigma}\right) ,
\end{equation}
with $\Phi$ being the c.d.f. of a standard Gaussian distribution.

Its distribution is known \emph{a posteriori}, given that
$\frac{\rho\log a+\mu}{\sigma}$, conditionally to $(a,\mbf a,\mbf y)$ for any $a\in\cA$, is distributed as the sum
of a variable with Gaussian distribution and the square root of a variable with Gamma distribution (both variables being independent):
\begin{align}
    &\frac{\rho\log a+\mu}{\sigma}\big| a,\mbf a,\mbf y \sim \cN\left(0,\frac{\log^2 a}{k\Var_k\hatmbf a} + \frac{\hatmbf a^{\top}\hatmbf a}{k\|\hatmbf a -\overline{\hatmbf a}\|^2} \right) + \left(\rho_k\log a + \frac{z_{k-1}}{\sqrt{k}}\right)\Gamma^{1/2}(\tilde c,\tilde d) \label{lowdoe:eq:apostcf} , \\
    &\tilde c = k/2, \qquad \tilde d = \frac{1}{2}\sum_{i=1}^{k-2}z_i^2.
\end{align}



\section{Sensitivity index for design of experiments}\label{lowdoe:sec:PE}

\subsection{A review of the global sensitivity analysis}

Global sensitivity analysis (GSA) %\sout{provides an essential scope for computer experiment based studies} \clem
{is a cornerstone of uncertainty quantification studies of computer simulators}. It aims at quantifying how the uncertainties within the observed output of a model are influenced by the uncertainties of one or several of its inputs \citep{iooss_review_2015}.
More formally, in classical GSA settings, a system outputs an observed variable $Y$, supposed to be a function of input variables $Y=\eta(X_1,\dots,X_p)$, where the input $X_i$'s are assumed to follow a known distribution and to be mutually independent. 
Since the first indices introduced by \citet{sobol_sensitivity_1993}, GSA's tools measure statistically how $Y$ is impacted by one or some of the $X_i$ \citep{da_veiga_basics_2021}.
Global sensitivity indices \citep{da_veiga_global_2015} are quantities whose class regroups a large range of these tools.
%
%Several of those measuring tools are regrouped within the class of the global sensitivity indices~\cite{DaVeiga2015}: 
According to their definition, letting $D$ be a dissimilarity measure between probability distributions, the impact of input $X_i$ onto the output $Y$ can be derived as
    \begin{equation}
        S_i = \EE_{X_i}\left[D(\PP_{Y}||\PP_{Y|X_i})\right] ,
    \end{equation}
where $\PP_Y$ is the distribution of $Y$, $\PP_{Y|X_i}$ is the conditional distribution of $Y$ given $X_i$. The choice of $D$ can depend on the expected properties. A classical example is to set $D(P||Q)=\|\EE_{X\sim P}[X]-\EE_{X\sim Q}[X]\|^2$ which gives the un-normalized Sobol' index \citep{sobol_sensitivity_1993}.



\subsection{Sequential planning of experiments via global sensitivity index maximization}

Following the idea of \citet{da_veiga_global_2015}, a judicious data acquisition strategy would be to minimize the sensitivity that the posterior would get from the observations. This way, the IM $a_{k+1}$ that has to be chosen for the next simulation which would output an EDP $y_{k+1}$, after having observed $(\mbf y, \mbf a)=(y_i,a_i)_{i=1}^k$ is the one such that the following index is maximized:
    \begin{equation}\label{lowdoe:eq:dataacq1}
        \EE_{y_{k+1}|a_{k+1},\mbf y,\mbf a}[D(\PP_{\theta|\mbf y,\mbf a} || \PP_{\theta|y_{k+1},a_{k+1},\mbf y,\mbf a} )],
    \end{equation}
where $\EE_{y_{k+1}|a_{k+1},\mbf y,\mbf a}$ is the expectation with respect to the distribution of $y_{k+1}$ given $a_{k+1},\mbf y,\mbf a$.
Within the GSA's scope, this makes the next experiment being chosen as the one such that the resulting observation of the structure's response provides the most impact onto the parameter of interest $\theta$. Sequentially, the observations are chosen to maximize the evolution of the posterior distribution. We invite one to notice that this viewpoint joins the reference prior theory one. Indeed, the reference prior is built to be the one such that the posterior distribution is expected to evolve the most from the prior. %~\cite{VanBiesbroeckBA2023}.

If the relation between the logarithm of the EDP and that of the IM is ``very close" to a linear relation, \cref{lowdoe:eq:dataacq1} is sufficient to improve the learning of the fragility curve. In other words, a strategy based on this equation makes it possible to sufficiently explore the space of the IMs, in order to maximize their empirical variance and thus reduce the \emph{a posteriori} variance of the estimation of the fragility curve, all things being equal (cf. \cref{lowdoe:eq:apostcf}). Note that there is no mathematical proof of this in this appendix, but it has been tested numerically. 

In practice, since the linear model is expected to be biased, the way to reduce the bias is to localize the learning, even if it is not optimal with respect to the \emph{a posteriori} variance of the estimation of the fragility curve. In our work, the locality of interest corresponds to the values of IMs for which the fragility curve evolves ``significantly" from 0 to 1. For this reason, we propose a refinement of the data acquisition strategy of \cref{lowdoe:eq:dataacq1} which includes the researched information:
    \begin{equation}\label{lowdoe:eq:ALs}
        \EE_{ s_{k+1}|a_{k+1},\mbf y,\mbf a}[D(\PP_{\theta|\mbf y,\mbf a} || \PP_{\theta|s_{k+1},a_{k+1},\mbf y,\mbf a} )],
    \end{equation}
with $s_{k+1}=\indic_{\hat y_{k+1}>0}$.

%Different dissimilarity measures are suggested and compared in \cref{lowdoe:sec:results}.
As a dissimilarity measure, we suggest the following, defined as a  Sobol' index of the fragility curve:
    \begin{align}
    \nonumber
        D(\PP_{\theta|\mbf y,\mbf a} || \PP_{\theta|s_{k+1},a_{k+1},\mbf y,\mbf a} ) &= \left\|\EE[P_f|\mbf y,\mbf a]-\EE[P_f|s_{k+1}, a_{k+1}, \mbf y,\mbf a]\right\|_{L^2}^2\\
        &= \int_\cA\left|\EE[P_f(a)|\mbf y,\mbf a]-\EE[P_f(a)|s_{k+1}, a_{k+1}, \mbf y,\mbf a]\right|^2d a ,\label{lowdoe:eq:index}
    \end{align}
where for any $a\in\cA$, $P_f(a)=\Phi\left((\rho\log a+\mu)/\sigma\right)$ inherits from the distribution of $\theta$. {Conditionally to $(\mbf y,\mbf a)$, its distribution has been elucidated in \cref{lowdoe:sec:priorposterior}. %For the computation of $\EE[P_f(a)|s_{k+1},a_{k+1},\mbf y,\mbf a]$, notice that
Also, 
\begin{equation}
    p(\theta|s_{k+1}, a_{k+1},\mbf y,\mbf a)  =  \frac{p(s_{k+1}|a_{k+1},\mbf y,\mbf a,\theta)}{\EE[ p(s_{k+1}|a_{k+1},\mbf y,\mbf a,\theta)|\mbf y,\mbf a]}%p(a_{k+1})
        %p(\tilde\theta|\mbf y,\mbf a)d\tilde\theta}
        p(\theta|\mbf y,\mbf a)
\end{equation}
%\cy{il y a $\tilde\theta$ qui se balade au dénominateur}
with $p(s_{k+1}|a_{k+1},\mbf y,\mbf a,\theta)=P_f(a_{k+1})^{s_{k+1}}(1-P_f(a_{k+1}))^{1-s_{k+1}}$. Thus, samples of $\theta$ conditionally to $(\mbf y,\mbf a)$ allow the approximation of both expectations in \cref{lowdoe:eq:index} by Monte-Carlo averages. The integrals in $a$ are estimated by Simpson's rule.
In the following example, a regular subdivision of $\cA=[0,A_{\rm max}]$ is suggested (see \cref{lowdoe:sec:benchamrk}).
}


{
The calculation of this index necessitates several initial observations. Actually, the derivations of both the likelihood and the posterior, as outlined in \cref{lowdoe:sec:likelihood,lowdoe:sec:priorposterior}, require that $k > 2$ and $a_1 \ne a_2$ (refer to \cref{lowdoe:app:sec:DiagUa}). In our experiments, we randomly select $k_0 = 3$ initial seismic signals with distinct IMs from their original distribution. The planning of experiment strategy is then sequentially implemented to select subsequent IM values by maximizing the numerical approximation of the index expressed in \cref{lowdoe:eq:ALs}. The optimization in one dimension is carried out using the BFGS algorithm.
}




%For computation purposes, the expectations are estimated by Monte-Carlo averages using the posterior distribution expressed in equation (\cref{{lowdoe:eq:posterior}), and the integrals in $a$ are estimated by Simpson's rule.

%\section{Consistency of the index?}



\section{Numerical application}\label{lowdoe:sec:application}

\subsection{Case study presentation}\label{lowdoe:sec:casestudy}

This case study concerns the seismic behavior of a piping system forming part of the secondary line of a French pressurized water reactor. \Cref{lowdoe:fig:ASG} presents a perspective of the mock-up positioned on the Azalee shaking table at the EMSI laboratory of CEA/Saclay. Simultaneously, \cref{lowdoe:fig:ASG}-right depicts the finite element model (FEM), employing beam elements and implemented through the proprietary FE code CAST3M \citep{cea_cast3m_2019}. The validation of the FEM was carried out thanks to an experimental campaign described in \cite{touboul_seismic_1999}.

The mock-up comprises a carbon steel TU42C pipe with an outer diameter of 114.3 mm, a thickness of 8.56 mm, and a 0.47 elbow characteristic parameter. This pipe, filled with water without pressure, includes three elbows, with a valve-mimicking mass of 120 kg, constituting over 30\% of the mock-up's total mass. One end of the mock-up is clamped, while the other is guided to restrict displacements in the X and Y directions. Additionally, a rod is positioned atop the specimen to limit mass displacements in the Z direction (refer to Figure 1-right). During testing, excitation was applied exclusively in the X direction.

The numerous simulations carried out for this case study were obtained with artificial seismic signals generated with the stochastic generator proposed by \citet{rezaeian_stochastic_2010}. This generator implemented in \cite{sainct_efficient_2020} was calibrated from 97 real accelerograms selected in the European Strong Motion Database for a magnitude $M$ such that $5.5 \leq M \leq 6.5$, and a source-to-site distance $R < 20$~km \citep{ambraseys_dissemination_2000}. Note that enrichment is not a necessity in the Bayesian framework -- especially if a sufficient number of real signals is available -- but it allows comparative performance studies, such as those presented in this work.

As in practice the piping system is located in a building, the artificial signals were filtered using a fictitious 2\% damped linear single-mode building at 5 Hz, which corresponds to the first eigenfrequency of the 1\% damped piping system. The chosen failure criterion is based on the assessment of excessive out-of-plane rotation of the elbow near the clamped end of the mock-up, following the recommendation in \cite{touboul_enhanced_2006}. {The chosen IM is the PSA which is calculated here at 5 Hz for a damping ratio of 1\%.}

In order to evaluate the effectiveness of the proposed method, we considered the nonlinear seismic behavior of the piping system. Regarding the nonlinear constitutive law of the material, a bilinear law exhibiting kinematic hardening was used to reproduce the overall nonlinear behavior of the mock-up with satisfactory agreement compared to the results of the seismic tests \citep{touboul_seismic_1999}. 

In this work, the critical rotation threshold is set at $C = 4.1^{\circ}$, representing the 90\%-level quantile derived from a sample of $10^4$ nonlinear numerical simulations.

Finally, the fragility curve that we will call ``reference" in the following was obtained by Monte-Carlo averages on clusters of the IM using the K-means algorithm, following the suggestion of \citet{trevlopoulos_parametric_2019}, from the $10^4$ data that we dispose. 
The estimation procedure is presented in \cref{chap:frags-intro}. In this method, the average goes along its confidence intervals; in the computation of the metrics that are suggested in the next section, the average is considered as the reference. %\cy{J'enlèverai la dernière phrase ?}
%In this reference are also presented the results of comparison with the Monte-Carlo method.

\begin{figure*}[!ht]
    \centering		
    \includegraphics[width=4.8cm]{figures/intro-frags/ASG.jpg}
    \hspace{1cm}
    \includegraphics[width=2.3cm]{figures/intro-frags/ASG_FEM.pdf}
    \caption{(left) Overview of the piping system on the Azalee shaking table and (right) Mock-up FEM.}
    \label{lowdoe:fig:ASG}
\end{figure*}


\subsection{Benchmarking metrics}\label{lowdoe:sec:benchamrk}

In order to evaluate the effects that our planning of experiments method has over the fragility curve estimates, we consider four quantitative metrics described hereafter. Those are later implemented on the \emph{a posteriori} estimates conditional to two types of dataset: one derived from our planning of experiments methodology, and one without.
Considering a sample $(\mbf a, \mbf y)$, we denote
by $a \mapsto P_f^{|\mbf a, \mbf y}(a)$ the random process defined as the fragility curve conditionally to the sample. $P_f^{|\mbf a,\mbf y}(a)=\Phi((\rho\log a+\mu)/\sigma)$ inherits from the \emph{a posteriori} distribution of $\theta$.  
For each value $a$ the $r$-quantile of the random variable $P_f^{|\mbf a,\mbf y}(a)$ is denoted by $q_r^{|\mbf a,\mbf y}(a)$, and its median is denoted by $m^{|\mbf a,\mbf y}(a)$.  
Also, we take into account the reference fragility curve $a\mapsto P_f^{\mathrm{ref}}(a)$, evoked in \cref{lowdoe:sec:casestudy}; and we consider a bounded set $\cA=[0,A_{\max}]$ for the IM, the truncation is set to the maximal IM within the database of $10^4$ seismic signals we have at disposal for this work, as disclosed in \cref{lowdoe:sec:casestudy}. % generated from the stochastic simulator of~\cite{Rezaeian2010} and implemented in~\cite{Sainct2020}.
    We define:
    \begin{itemize}
        \item The square bias to the median: $\displaystyle{\cB^{|\mbf a,\mbf y} = \|m^{|\mbf a,\mbf y}- P_f^{\mathrm{ref}}\|^2_{L^2};\quad \text{where}\quad\|P\|^2_{L^2} = \frac{1}{A_{\max}}\int_{0}^{A_{\max}}P(a)^2 d a.}$
           % \begin{equation}
           %     \cB^{|\mbf a,\mbf y} = \|m^{|\mbf a,\mbf y}- P_f^{\mathrm{ref}}\|^2_{L^2};\quad \text{where}\quad\|P\|^2_{L^2} = \frac{1}{A_{\max}}\int_{0}^{A_{\max}}P(a)da.
           % \end{equation}
        \item The quadratic error: $\displaystyle{\cE^{|\mbf a,\mbf y} = \EE\left[\|P_f^{|\mbf a,\mbf y}-P_f^{\mathrm{ref}} \|^2_{L^2}|\mbf a,\mbf y\right].}$
            %\begin{equation}
            %    \cE^{|\mbf a,\mbf y} = \EE[\|P_f^{|\mbf a,\mbf y}-P_f^{\mathrm{ref}} \|^2_{L^2}|\mbf a,\mbf y].
            %\end{equation}
        \item The $1-r$-square credibility width: $\displaystyle{\cW^{|\mbf a,\mbf y} = \|q_{1-{r/2}}^{|\mbf a,\mbf y} - q_{{r/2}}^{|\mbf a,\mbf y}\|_{L^2}^2  .}$
            %\begin{equation}
            %    \cW^{|\mbf a,\mbf y} = \|q_{1-{r/2}}^{|\mbf a,\mbf y} - q_{{r/2}}^{|\mbf a,\mbf y}\|_{L^2}^2  .
            %\end{equation}
        \item The $1-r$-coverage probability: % zone's trustworthiness: 
        $\displaystyle{\cP^{|\mbf a,\mbf y} = \frac{1}{A_{\max}}\int_0^{A_{\max}}\indic_{P_f^{\mathrm{ref}}(a)\in \left[q_{1-{r/2}}^{|\mbf a,\mbf y}(a), q_{{r/2}}^{|\mbf a,\mbf y}(a) \right] } d a.}$
            %\begin{equation}
            %    \cT^{|\mbf a,\mbf y} = \frac{1}{A_{\max}}\int_0^{A_{\max}}\indic_{P_f^{\mathrm{ref}}(a)\in \left[q_{1-{r/2}}^{|\mbf a,\mbf y}(a), q_{{r/2}}^{|\mbf a,\mbf y}(a) \right] } da.
            %\end{equation}
    \end{itemize}
For the forthcoming implementation of these metrics, numerous \emph{a posteriori} samples of the process $P_f^{|\mbf a,\mbf y}$ are generated from their known distribution (see \cref{lowdoe:eq:posterior}) and serve the computation of the medians, quantiles and means through Monte-Carlo derivation. The integrals are approximated numerically from Simpsons' interpolation
{on sub-intervals of regular size  $0=A_0<\dots<A_p=A_{\rm max}$. In our computations, we use  $A_{\rm max}=55\,\mathrm{m/s^2}$, and $p=200$.}



\subsection{Numerical results}\label{lowdoe:sec:results}

%
\Cref{lowdoe:fig:examples} shows examples of fragility curve estimations based on different dataset sizes. 
%Both estimations differ only from their observation samples. 
The results presented in \cref{lowdoe:fig:examples}-(c) come from our planning-of-experiments (PE) methodology while the results presented in \namecrefs{lowdoe:fig:examples}~\ref{lowdoe:fig:examples}-(a) and~\ref{lowdoe:fig:examples}-(b) come from independent random samples, with IMs that have been drawn w.r.t. their original standard distribution or w.r.t. a uniform distribution. These qualitative results clearly illustrate the contribution of our methodology.

%Indeed, under a standardly drawn sample (understand here that the observed IM are independently sampled  w.r.t.{ }their original distribution), a fast convergence of the estimate is performed when the number of observation rises, with credibility intervals that get thinner accordingly.
%However, such strong convergence targets an incorrect value, that we suppose being due to the  bias of the linear model.

When samples are randomly drawn from the original IM distribution {or from a uniform distribution}, the results show a rapid convergence of the estimates -- in the sense that the associated credibility intervals decrease rapidly -- towards biased estimations of the fragility curves. Conversely, when the samples come from the PE methodology, the bias decreases but the convergence is slower.

These observations are confirmed on a larger scale by the results presented in \cref{lowdoe:fig:errors,lowdoe:fig:credibility}. 
These are issued from computations of the metrics $\cB^{|\mbf a,\mbf y}$, $\cE^{|\mbf a,\mbf y}$, $\cW^{|\mbf a,\mbf y}$ and $\cP^{|\mbf a,\mbf y}$ described in \cref{lowdoe:sec:benchamrk}, for various observation sets $(\mbf a,\mbf y)$. They compare the performances of our PE methodology with the two methods that are based on independently drawn observations: the one that involves  IM samples drawn from their standard distribution, and the one that involves IM samples drawn from a uniform distribution over the range $[0,A_{\rm max}]$.

\Cref{lowdoe:fig:errors} shows empirical comparisons of the bias and of the quadratic error between the method involving a design of experiments and without. These two results clearly illustrate that the PE approach outperforms the standard {and the uniform} approaches. Although the quadratic error is strongly related to the variance of the estimates it is significantly offset by the fact that the bias is smaller with the PE approach. Indeed, \cref{lowdoe:fig:credibility}-left illustrates that the $95\%$-square credibility width is smaller with the standard {and uniform} approaches than with the PE-based approach. This good result nevertheless masks a lack of robustness of the standard {or uniform}  approaches since the estimate turns out to be strongly biased, as shown in \cref{lowdoe:fig:credibility}-right. This figure shows indeed the coverage probability for the both methods, as a function of the dataset size. It measures the average inclusion of the reference fragility curves to the \emph{a posteriori} credibility intervals.

%Secondly, that last point is emphasized by what is shown in  Indeed, the standard method is there exposed to provide credibility intervals that get strongly thinner than the ones benefiting from P.E. 
%Yet, \cref{lowdoe:fig:credibility}-right unveils their lack of robustness under the first method. That last figure plots the coverage probability of both methods as a function of the dataset size. It measures the average belonging of the reference within the \emph{a posteriori} credibility intervals. The P.E. handles to avoid the trap of the model bias within the reliable credibility intervals it issues.




\begin{figure}
    \centering%
    \includegraphics[width=5cm]{figures/low-doe/curves_standard.pdf}%
    \includegraphics[width=5cm]{figures/low-doe/curves_unif.pdf}%
    \includegraphics[width=5cm]{figures/low-doe/curves_PE.pdf}\\
    \hspace*{1.7em}{\hspace{\stretch{1}}(a)\hspace{\stretch{2}}(b)\hspace{\stretch{2}}(c)\hspace{\stretch{1}}\ }
    % \begin{tabular}{ccc}
    %    (a)  & (b) & (c) \\
    %      & 
    % \end{tabular}
    % \includegraphics[width=0.36\linewidth]{figures/low-doe/ASG_nlin_PSA_refMC/ex_stan_IC_ref.pdf}\hspace*{1em}%
    % \includegraphics[width=0.36\linewidth]{figures/low-doe/ASG_nlin_PSA_refMC/ex_PE_IC_ref.pdf}%    
    \caption{Examples of fragility curve estimations for different number of observations, with the PSA considered as IM. Are plotted the $95\%$ credibility interval for 3 datasets of respective sizes 20 (blue), 80 (orange), and 200 (red); the reference curve $P_f^{\mathrm{ref}}$ is drawn in magenta and is accompanied with the confidence interval of the procedure in dashed lines.
    The observations are chosen (a) w.r.t.{ }the standard distribution of the IM; 
    {or (b) w.r.t.{ }a uniform distribution on $[0,A_{\rm max}]$};
    or (c) using our planning of experiments method. The green crosses represent 50 pairs $(a_i,\indic_{y_i>C})$ drawn for each method.}
    \label{lowdoe:fig:examples}
\end{figure}


\begin{figure}[h]
    \centering%
    \includegraphics[width=5cm]{figures/low-doe/errB.pdf}%
    \includegraphics[width=5cm]{figures/low-doe/errE.pdf}%    
    \caption{Confidence intervals and means w.r.t.{ }the 
    $(\mbf a,\mbf y)$ for (left) the square bias to the median $\cB^{|\mbf a,\mbf y}$ and (right) the quadratic error $\cE^{|\mbf a,\mbf y}$; as a function of the number of observations. For each value of $k=5,10,\dots,200$, a number of $L=200$ dataset have been drawn following the standard distribution of the IM firstly (for the blue curves), {following a uniform distribution on $[0,A_{\rm max}]$ secondly (for the orange curves)}, and following the planning of experiments method thirdly (for the red curves). }
    \label{lowdoe:fig:errors}
\end{figure}


\begin{figure}[h]
    \centering%
    \includegraphics[width=5cm]{figures/low-doe/errW.pdf}%
    \includegraphics[width=5cm]{figures/low-doe/errP.pdf}%    
    \caption{(left) $95\%$-confidence intervals and means w.r.t.{ }$(\mbf a,\mbf y)$ for the $95\%$-square credibility width $\cW^{|\mbf a,\mbf y}$, as a function of the number of observations. (right) mean w.r.t.{ }$(\mbf a,\mbf y)$ for the $95\%$-coverage probability $\cP^{|\mbf a,\mbf y}$, as a function of the number of observations. For each value of $k=5,10,\dots,200$, $L=200$ datasets have been drawn following the standard distribution of the IM firstly (for the blue curves), {following a uniform distribution on $[0,A_{\rm max}]$ secondly (for the orange curves)}, and following the planning of experiments method thirdly (for the red curves).} %; the trustworthy of the credibi credibility zone is considered trustworthy if }
    \label{lowdoe:fig:credibility}
\end{figure}

\section{Discussion}\label{lowdoe:sec:discussion}

The results presented in the previous section clearly illustrate the superiority of the PE-based approach over the standard {and uniform} approaches. Similar results, not presented here for the sake of brevity, and obtained with the PGA as IM, as well as with other types of structures, also confirm these results.

The strength of the approach proposed in this work is its completely analytical nature, which avoids the use of MCMC methods for the \emph{a posteriori} estimation of fragility curves. To do this, however, it is necessary to assume that the logarithm of the EDP evolves linearly as a function of the logarithm of the IM.

So, to solve this problem, we are faced with a contradiction. In order to satisfy the linearity assumption, it is necessary, on the one hand, that the learning zone is local, that is to say restricted to the vicinity of the IMs for which the fragility curve evolves significantly from 0 to 1. On the other hand, \cref{lowdoe:eq:apostcf} shows that an empirical variance of the IMs that is too small is penalizing from the point of view of the variance of the estimation of the fragility curves. The variance of the latter is in fact inversely proportional to that of the IMs considered for learning.

As the numerical results show, the proposed learning method localizes the learning in the area of interest and significantly reduces the model bias. As a result, this is accompanied by a slight reduction in the size of the credibility interval with the number of training data.

This therefore suggests that the proposed method is effective for samples
with limited size (for instance, given the results presented in \cref{lowdoe:sec:results}, an appropriate limit could be a sample size smaller than 100 for the case study treated in this appendix). Beyond that, given the cost associated with each training data, it seems preferable to move towards less constrained and more sophisticated methods.

\section{Conclusion}\label{lowdoe:sec:conclusion}

Assessing the seismic fragility of structures and components when few data are available is a challenging task and the Bayesian framework is known to be effective for these types of problems.

In this work we proposed an efficient Bayesian methodology whose strength lies in its fully analytical nature, which avoids the use of MCMC methods for the \emph{a posteriori} estimation of fragility curves.
The effectiveness of the method comes from the assumption of linearity between the logarithm of the EDP and that of the IM of interest. As this hypothesis implies a model bias in most practical cases, we proposed a strategy in order to minimize this bias, by concentrating the learning in the vicinity of the IMs for which the fragility curve evolves significantly from 0 to 1.

The numerical results clearly illustrate the superiority of the proposed approach over an approach without a learning strategy. 
They emphasize the robustness of a design of experiments which is based on a sensitivity analysis of the posterior distribution.
 Such construction is not limited to the modeling we derive in this work in particular, and could still be adapted to another to increase its learning abilities. 
They also suggest that the proposed method is effective for a limited sample size (about 100 in our settings). Beyond that, given the cost associated with each training data, it seems preferable to move towards less constrained and more sophisticated methods, in order to more effectively minimize both the biases and the variance of the estimates.

For practitioners, this method therefore constitutes a rapid and robust tool for first estimates of fragility curves in a context where the datasets are of limited size.

%As the design of experiment 



% \appendix

\section{Details regarding the construction of $\mbf z$}\label{lowdoe:app:sec:DiagUa}

%\subsection{Definition}
%\paragraph{Diagonalization of $U_{\mbf a}$} 

%Now consider the pair $\xi=(\mu,\sigma)$, for it can be build a likelihood as
Conditionally to $(\mbf a, \theta)$, we derive the distribution of $\hatmbf y-\rho_k\hatmbf a$: 
    \begin{equation}\label{lowdoe:eq:likelihood2}
        \hatmbf y - \rho_k\hatmbf a|\mbf a, \theta \sim\cN(\mu \mbf 1, \sigma^2 U_{\mbf a})  , \quad\text{with}\quad U_{\mbf a} = I - \frac{\hatmbf a(\frac{1}{2}\hatmbf a - \overline{\hatmbf a})^{\top}  }{k\Var_k\hatmbf a} - \frac{(\frac{1}{2}\hatmbf a - \overline{\hatmbf a}) \hatmbf a^{\top}  }{k\Var_k\hatmbf a} ,\quad \overline{\hatmbf a} = \frac{1}{k}\sum_{i=1}^k \log a_i,
        %\\
        % &\ell_k(\mbf z |\mbf a,\xi) = \prod_{i=1}^k\frac{1}{\sqrt{2\pi}\sigma}\exp\left(-\frac{(z_i - \mu U_{\mbf a,i}^{-1/2})^2}{2\sigma^2}  \right)
    \end{equation}
where $\mbf 1$ denotes the vector of $\RR^k$ which contains only ones. %, and the matrix $U_{\mbf a}$ is given by
% \begin{equation}
%     U_{\mbf a} = I - \frac{\hatmbf a(\frac{1}{2}\hatmbf a - \overline{\hatmbf a})^{\top}  }{k\Var_k\hatmbf a} - \frac{(\frac{1}{2}\hatmbf a - \overline{\hatmbf a}) \hatmbf a^{\top}  }{k\Var_k\hatmbf a}\ ;\quad \overline{\hatmbf a} = \frac{1}{k}\sum_{i=1}^k \log a_i.
% \end{equation}
Below is suggested a diagonalization of the matrix $U_{\mbf a}$: we define $P_{\mbf a}$ such that $P_{\mbf a}^{\top}P_{\mbf a}=I$ and $P_{\mbf a}^{\top}U_{\mbf a}P_{\mbf a}$ is a diagonal matrix. To define $\mbf z$, we denote by $\tilde P_{\mbf a}$ the matrix in $\RR^{k\times(k-1)}$ composed by the $k-1$ first columns of $P_{\mbf a}$. {In what follows, we assume that $k>2$ and that the coordinates of $\mbf a$ are not all identical.} \\


The matrix $U_{\mbf a}$ takes the form of $I-uv^{\top}-vu^{\top}$ for some vectors $u$ and $v$ of $\RR^k$, which are linearly independent. It is clear that the diagonalization of $U_{\mbf a}$ is linked with the one of $V=uv^{\top}+vu^{\top}$.

First of all notice that $v^\perp$ and $u^\perp$ are two different hyperplanes because of the linear independence of $u$ and $v$. That makes $u^\perp\cap v^\perp$ a subspace of dimension $k-2$. %; and $u^\perp\prive\RR v$, $v^\perp\prive\RR u$ two non empty sets.
Therefore, as one would notice that $u^\perp\cap v^\perp\subset\Kernel V$ and $\Image V\subset\Span(u,v)$, the converse inclusions stand.

This way, while $0$ is the first eigenvalue of $V$ with rank $k-2$, an other eigenvalue $r$ must admit eigenvectors in $\Span(u,v)$, which should make the system
    \begin{equation}\label{lowdoe:app:eq:systemr}
        \left\{\begin{array}{rl}
             r\gamma &= \gamma v^{\top}u+\delta v^{\top}v  \\
             r\delta &= \gamma u^{\top}u + \delta u^{\top}v 
        \end{array}  \right.
    \end{equation}
admitting an infinity of solutions w.r.t.{ }$(\gamma,\delta)$. Equivalently, its determinant must be null which lead to the two solutions 
    \begin{equation}
        r = v^{\top}u \pm \|v\|\|u\|.
    \end{equation}
As $u$ and $v$ are linearly independent, the equation above defines two different eigenvalues $r_+$ and $r_-$, both of rank $1$.
Let $r$ be one of those, a resolution of the equation system~(\ref{lowdoe:app:eq:systemr}) gives that the eigenspace associated with $r$ is $\Span(v^{\top}vu+(r-v^{\top}u)v)$.\\

Coming back to $U_{\mbf a}$, the arguments above show that the eigenvalues of $U_{\mbf a}$ are $1$, $1-r_-$ and $1-r_+$ with respective ranks $k-2$, $1$ and $1$, and with:
\begin{equation}
    r_+ = 1 \ ,\quad r_- = \frac{- k\overline{\hatmbf a}^2}{\|\hatmbf a-\overline{\hatmbf a} \|^2},
\end{equation}
because 
\begin{equation}
    \|\hatmbf a\|^2\|\frac{1}{2}\hatmbf a - \overline{\hatmbf a} \|^2  = \sum_{i=1}^k\hat a_i^2
    \left(\frac{1}{4}\sum_{i=1}^k\hat a_i^2 + \left(\sum_{i=1}^k\hat a_i^2\right)\frac{1}{k}\left(\sum_{i=1}^k\hat a_i\right)^2 - \left(\sum_{i=1}^k\hat a_i^2\right)\frac{1}{k}\left(\sum_{i=1}^k\hat a_i \right)^2 \right)\\
        = \frac{1}{4}\left(\sum_{i=1}^k\hat a_i^2\right)^2. % \\
   % \mbox{and}\quad 
\end{equation}

Now, let us choose $w_1,\dots,w_{k-2}$ an orthonormal basis of $\hatmbf a^\perp\cap(\frac{1}{2}\hatmbf a-\overline{\hatmbf a})^\perp$. Let us define 
\begin{equation}\label{lowdoe:eq:wk-1,wk}
    w_{k-1} = \frac{1}{\sqrt k}\mbf 1 ,\quad w_{k} = \frac{\hatmbf a - \overline{\hatmbf a}}{\|\hatmbf a-\overline{\hatmbf a}\|}.
\end{equation}
% \begin{align}
%     w_{k-1} &= \frac{1}{\sqrt k}\mbf 1 \\ %\frac{\overline{\hatmbf a}\mbf 1}{\frac{1}{2}\hatmbf a^{\top}\hatmbf a} \\
%     w_{k} &= \frac{\hatmbf a - \overline{\hatmbf a}}{\|\hatmbf a-\overline{\hatmbf a}\|}.
% \end{align}
We remind $\mbf 1$ is the vector whose coordinates are ones. Therefore, denoting $P_{\mbf a}$ the matrix whose columns are the $w_i$, $i=1,\dots,k$; it comes $P_{\mbf a}^{\top}P_{\mbf a}=I$ and
    \begin{equation}
        P^{\top}_{\mbf a}U_{\mbf a}P_{\mbf a} = \diag\left(1,\dots,1,\frac{\hatmbf a^{\top}\hatmbf a}{\|\hatmbf a-\overline{\hatmbf a}\|^2}, 0\right).
    \end{equation}

% {The complete definition of $P_{\mbf a}$ depends on the chosen orthonormal basis $w_1,\dots,w_{k-2}$ of $\hatmbf a^\perp\cap(\frac{1}{2}\hatmbf a^\perp-\overline{\hatmbf a})$.
% In practice, we proceed as follows to construct them: from $w_{k-1}=(w_{k-1}^{(j)})_{j=1}^k$ and $w_k=(w_k^{(j)})_{j=1}^k$ as defined in equation (\cref{{lowdoe:eq:wk-1,wk}), we denote by $\tilde w_i=(\tilde w_i^{(j)})_{j=1}^k$ the vector of $\RR^k$ such that
% \begin{align}
%     (\tilde w^{(0)}_i,\,\tilde w_i^{(i)},\,\tilde w^{(i+1)}_i) &= (w^{(0)}_{k-1},\, w_{k-1}^{(i)},\,w^{(i+1)}_{k-1})\wedge (w^{(0)}_k,\,w_k^{(i)},\,w^{(i+1)}_k)\\ &= (w^{(i)}_{k-1}w^{(i+1)}_{k}-w^{(i)}_{k}w^{(i+1)}_{k-1},\, w^{(i+1)}_{k}w^{(0)}_{k-1}-w^{(i+1)}_{k-1}w^{(0)}_{k},\, w^{(0)}_{k-1}w_{k}^{(i)}-w^{(0)}_{k}w_{k-1}^{(i)}),\nonumber
% \end{align}
% and $\tilde w_i^{(j)}=0$ for any $j\not\in\{0,i,i+1\}$. That ensures the $\tilde w_i$, $i=1,\dots,k-2$ to form a basis of $\Span(w_{k-1},w_k)^\perp=\hatmbf a^\perp\cap(\frac{1}{2}\hatmbf a^\perp-\overline{\hatmbf a})$. Then, the orthonormal basis $w_1,\dots,w_{k-2}$ results from the Graam-Schmit process applied on $\tilde w_1,\dots\tilde w_{k-2}$.
% }


{The complete definition of $P_{\mbf a}$ depends on the chosen orthonormal basis $w_1,\dots,w_{k-2}$ of $\hatmbf a^\perp\cap(\frac{1}{2}\hatmbf a-\overline{\hatmbf a})^\perp$.
In practice, we proceed as follows to construct it, starting from $w_{k-1}=(w_{k-1}^{(j)})_{j=1}^k$ and $w_k=(w_k^{(j)})_{j=1}^k$ as defined in \cref{lowdoe:eq:wk-1,wk}. 
We denote by $e_i$ the canonical vectors of $\RR^k$ (the $j$th coordinate of $e_i$ is equal to $0$ iff $j\ne i$). As the coordinates of $\mbf a$ are not all the sames, there exist $j,p$ such that $w_{k}^{(j)}\ne w_{k}^{(p)}$ (in practice, we select the minimal $j$ and the minimal $p$ such that this property is verified). Thus, we can show that the vectors $w_k,w_{k-1},(e_i)_{i\ne j,p}$ form a basis of $\RR^k$ by computing their determinant:
    \begin{equation}
        \det(w_k,w_{k-1},(e_i)_{i\ne j,p}) = w_k^{(j)}w_{k-1}^{(p)}(-1)^{j+p+1} - w_{k-1}^{(j)}w_k^{(p)}(-1)^{j+p+1}\ne 0,
    \end{equation}
%with $\varepsilon$ being the determinent of the family $(e_i)_{i\ne j,p}$ after removing their $i$th and their $p$th row, so that $\varepsilon = 1$.
as $w_{k-1}^{(j)}=w_{k-1}^{(p)}$.
Eventually, the family of the $w_i,\,i=1,\dots k$ is the result of the Gram-Schmidt process applied to $u_k,\dots,u_1=w_k,w_{k-1},(e_i)_{i\ne j,p}$: 
    \begin{equation}
        w_{k-i} = \frac{\tilde w_{k-i}}{\|\tilde w_{k-i}\|},\qquad \tilde w_{k-i} = u_{k-i} - \sum_{j<i} w_{k-j}^{\top}u_{k-i}\cdot w_{k-j}, 
    \end{equation}
%with $\langle\cdot,\cdot\rangle$ denoting the usual scalar product in $\RR^k$. 
Note that this process leaves the expressions of $w_k$ and $w_{k-1}$ unchanged. The family $w_1,\dots,w_{k-2}$ thus forms a basis of $\Span(w_k,w_{k-1})^\perp=\hatmbf a^\perp\cap(\frac{1}{2}\hatmbf a-\overline{\hatmbf a})^\perp$.

We invite the reader to notice that in any way, the construction of those $k-2$ first columns of $P_{\mbf a}$ has no influence on the resulting posterior distribution of interest (given by \cref{lowdoe:eq:posterior}). Indeed, that latter only involves the expressions of $z_{k-1}$ and of $\sum_{i=1}^{k-2}z_i^2$. Concerning the first one, it is equal to $w_{k-1}^{\top}(\hatmbf y-\rho_k\hatmbf a)$, and concerning the second one, it is equal to:
    \begin{align*}
            \sum_{i=1}^{k-2}z_i^2=\sum_{i=1}^{k-2}|w_i^{\top}(\hatmbf y-\rho_k\hatmbf a)|^2&=\sum_{i=1}^{k}|w_i^{\top}(\hatmbf y-\rho_k\hatmbf a)|^2-|w_{k-1}^{\top}(\hatmbf y-\rho_k\hatmbf a )|^2-|w_k^{\top}(\hatmbf y-\rho_k\hatmbf a )|^2\\
                &=\|\hatmbf y-\rho_k\hatmbf a\|^2-|w_{k-1}^{\top}(\hatmbf y-\rho_k\hatmbf a )|^2-|w_k^{\top}(\hatmbf y-\rho_k\hatmbf a )|^2.
        \end{align*}
    }







\chapter{Non-asymptotic reference priors, a brief study}



\begin{abstract}[\hspace*{-10pt}]
    This appendix compiles some theoretical developments that were conducted during my end-of-study internship that took place at CEA Saclay in 2021. %one year before this thesis at CEA Saclay.
%  precedes the thesis: at CEA Saclay 
\end{abstract}


\begin{abstract}
    abstract
\end{abstract}


\minitoc

\section{Motivations and context}

Reference prior theory has been built on the idea to construct  priors whose influence onto the prior would be minimized, in order to let the latter being informed by the data in priority.
This aim of that idea is to qualify the resulting prior as ``objective''.

To achieve that goal, 
instead of seeking to maximize the expected divergence between the prior and the posterior (i.e., the mutual information), 
the authors who developed the theory \citep{bernardo_reference_1979,bernardo_bayesian_1994} suggested maximizing the asymptotic value of that quantity as the number of observation tends to infinity (see \cref{chap:intro-ref} for a complete review of the theory).
Indeed, it is seen as an issue that the mutual information ---and its maximal argument--- depends on the number of observations $k$, since the ``objective'' prior should, by nature, be adequate for any data samples, no matter their size.
According to the \citet{bernardo_bayesian_1994}, when the number of observations $k$ is fixed, the distribution of the vector of the data is not fully described. They advance that, denoting $\sI^k$ the mutual information when $k$ data items are observed, the limit $\sI^\infty$ (if it exists) measures the knowledge missing from the prior.



%the theory seeks to maximize the mutual information, that is defined as an expected divergence between the prior and the posterior.

However, there exist few elements that support formally this asymptotic definition of the reference prior in the literature.
While there is consensus on the definition of \citet{bernardo_reference_1979} that consists in asymptotically maximizing the mutual information, we are interested in the expression of the prior that maximizes it non-asymptotically.
This study is motivated by the two following thoughts: (i)~in a case where the number of observations represents a fundamental element of the problem, a prior that takes it into account could be valorized; (ii)~expressing the non-asymptotic reference prior can help to understand or to express the asymptotic reference priors.

We suggest in this short appendix a main result that expresses implicitly the non-asymptotic reference priors. First, we recall the framework of the reference prior theory in the next section, then our results is developed in ??.
In ?? we suggest a discussion of our result and elucidate its link with classical asymptotic reference priors.




\section{Non-asymptotic reference priors}

We remind that the original framework of the reference prior theory is comprehensively detailed in \cref{chap:intro-ref}. We consider a Bayesian framework: observations $\mbf Y_k\in\cY^k$ follows conditionally to $T=\theta\in\Theta$ the distribution $\PP_{\mbf Y_k|\theta}=\PP_{Y|\theta}^{\otimes k}$. The marginal distribution is denoted  $\PP_{\mbf Y_k}$ and the prior distribution $\varPi$.
We suppose that the model admits a likelihood denoted $\ell_k$, and that all the distributions admits densities. The marginal, posterior, and prior densities are respectively denoted $p_{\mbf Y_k}$, $p$, and $\pi$.

Under those settings, the mutual information given $k$ observations is defined as
    \begin{equation}
        \sI^k(\varPi) =
    \end{equation}

In this appendix, we define a non-asympotic reference prior as a maximal argument of the mutual information.
\begin{defi}[Non-asymptotic reference prior]
    p
\end{defi}


The result below gives an implicit expression of the non-asymptotic reference priors among a non-restrictive set of priors

\begin{thm}
    f
\end{thm}


Following the idea provided in \cref{chap:constrained-prior}, we suggest also the study of non-asymptotic reference prior under constraints. The result below considers constraints that take the form of linear constraints. 


\begin{thm}
    g
\end{thm}







\section{Link with asymptotic reference priors}




% \section{Proofs}



\section{Discussion}









\chapter{Reminder of useful mathematical notions}

 
 
\bookmarksetup{startatroot}% this is it
%\addtocontents{toc}{\bigskip}

\printbibliography[heading=chapter,title=Bibliography]


\end{document}
