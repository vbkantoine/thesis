
\begin{mtchideinmaintoc}[-1]
\chapter{Conclusion \& perspectives}
\end{mtchideinmaintoc}
% \refstepcounter{chapter}
% \addcontentsline{toc}{chapter}{}%
% \markboth{Conclusion \& perspectives}{}
% \addtocounter{chapter}{1}
% \addstarredchapt{}
% \mtcaddchapter
% \adjustmtc
%\decrementmtc


\begin{abstract}[Foreword]
    In this part, we provide a general conclusion to this thesis and this manuscript. Following the structure below, we express the conclusive thoughts that terminate this work.
\end{abstract}


\minitoc

% \listofconcSec
%\parttoc
% \adjustmtc
% \renewcommand{\thesection}{}

% \mtcaddchapter
% \setcounter{section}{0}
% \adjuststc
% \section*{test}
% \addtocontents{lof}{test}

% \concSec{Test}






% \section{Summary}

% % Dans le chapitre 1, nous avons apporté un a priori au contenu de ce manuscrit.
% In this last part, we propose an \emph{a posteriori} analysis of the research conducted during this thesis, and developed in  \cref{part:ref-theory} and \cref{part:spra}.
% The \emph{a priori} of this content was elucidated in \cref{chap:intro-english} through an analysis of the problem and an analysis of the approaches that it invited, in our opinion, to explore.
% This way, we remind that 6 open questions where elucidated?, we recapitualte those below:\\[-5pt]





% \ques{i}{: How can one define and support the objectivity of a prior?}


% \ques{ii}{: What are the limits of the implementation of such non-informative priors, and how to reconcile their use with practical needs?}

% \ques{iii}{: How to build and derive such priors in practice?}

% \ques{iv}{: in the context of SPRA, what are the forms of these objective priors within a model for seismic fragility estimation}

% \ques{v}{: What are the consequences entailed by the lack of information in this model and how to solve them?}

% \ques{vi}{: How to finally leverage the most the different sources of information in the whole Bayesian workflow of the studied model?}





% %Naturally, these 6 questions address two different domains that are distinc from a first sight?




% %Ici, nous conduisons l'analyse a posteriori, au regard des résultats ui font suite au deux parties qui précèdent.


% % Pour la construction de l'a priori à ce contenu, nous avons conduit au chapitre 1 une analyse de la problématique, et une  sur les approches qu'elle semblait inviter à mener.
% % De cette sorte, on rappelle que cette analyse donnait lieu à l'expression de 6 questions ouverts, que nous récapitulons ci-dessous.









% Naturallement, ces 6 questions adressent 2 mondes a première vue distincts : la théorie des priors de référence ainsi que l'étude de courbes de fragilité sismique.
% Ainsi, les questions, l'analyse de sujet et de la problématique a d'abord été enrichie d'une analyse profonde de l'etat de l'art de chacun de ces deux domaines.
% Cette distinction a permis de structurer les travaux conduit pendant cette thèse ainsi que le manuscrit selon s'il se ratachait plutôt à l'un ou à l'autre des deux domaines. 
% C'est pourquoi dans les deux sections qui suivent (\cref{sec:concl:refpr,sec:concl:frags}) nous faerrons le point sur les aspects contributifs de ce travail 
% à chacun des deux sujets.
% Cependant, nous appuyons aussi l'aspect contributif propre de l'interaction entre les deux sujet, novateur et qui est la finalité de la réponse à la problématique du sujet, nous en parlons en \cref{sec:concl:general}.
% Ces travaux sont vaste etpuisqu'ils appréhendent un large pannel de choses, ils ouvrent par la même grand nombre de pistes. Nous discontons de celles-ci en \cref{sec:concl:perspectives}.







\section{Summary}




In this final chapter, we present an \emph{a posteriori} analysis of the research conducted throughout this thesis, that we developed in \cref{part:ref-theory} and \cref{part:spra}. The \emph{a priori} takes the form here of the 
motivations and preliminary considerations that were outlined in \cref{chap:intro-english}, through a detailed examination of the problem and a discussion of the methodological avenues that, in our view, invited for further exploration.
To structure this investigation, we recall that we identified six guiding questions that shaped the architecture of the thesis. These are recapitulated below:\\[-5pt]



\ques{i}{How can one define and support the objectivity of a prior?}


\ques{ii}{What are the limitations of implementing such non-informative priors, and how can their use be reconciled  with practical needs?}

\ques{iii}{How can such priors be constructed and derived in practice?}

\ques{iv}{In the context of SPRA, what forms do these objective priors take within a seismic fragility curve model?}

\ques{v}{What are the implications of information scarcity in such models, and how can they be addressed?}

\ques{vi}{How can the different sources of information be best integrated across the full Bayesian workflow of the model under study?}

These six questions naturally span two domains that may appear distinct at first glance: the reference prior theory and the study of seismic fragility curves. Accordingly, the initial formulation of the research questions and the analysis of the core problem were complemented by an in-depth review of the state of the art in both fields.

This distinction helped to structure the work carried out in this thesis, as well as the organization of the manuscript itself, with each part aligned more closely with one of the two domains. Consequently, in the following sections (\cref{sec:concl:refpr,sec:concl:frags}), we reflect separately on the contributions made to each area. However, particular emphasis is also placed on the original contribution outlined from the interaction between the two domains. This interaction represents an innovative aspect that ultimately address the main problem addressed by the thesis, we discuss this point in \cref{sec:concl:general}.
These works are various and given the wide range of issues they engage with, several open directions for future research emerge. These are discussed in \cref{sec:concl:perspectives}.




\section{On the contributions to the reference prior theory}\label{sec:concl:refpr}

% L'objectivisme
% % La qualification d'objectivité 
% reste un idéal pas très bien défini dans le workflow Bayésien. La litérature s'accorde plutôt en règle général sur le fait qu'on s'en approche lorsque l'on favorise les données et les informations qu'elles apportent en ce sens devant tout autre incorporation subjective.
% Il est difficile de conclure si les priors de référence sont des priors objectifs, néanmoins, leur construction cherche à minimiser l'introduction de subjectivité par le prior.

% Au final, on peut donc voir un prior objectif comme un idéal, et le cadre des priors de référence apporte un indice de quantification d'une proximité de cet idéal.
% On a tenté dans le \cref{chap:ref-generalized} d'étendre la portée de cet indice, en cherchant à le rendre plus général, nos résultats ont prouvé que la solution du problème définissant les priors de référence %(i.e., quel est le prior de ref) 
% est robuste à notre généralisation. Dans la pluspart des cas, sous un minimum d'hypothèses, le prior de référence est le prior de Jeffreys.
% On répond alors ici à la \textbf{question i}. 


% Pour contrer les limitations connues et nombreuses du prior de Jeffreys on doit alors se replonger dans le questionnement de l'objectivité voulue. Finalement, nombreuses sont les études ou les priors ``les plus obectifs possible'' ne sont pas tant ceux qui sont attendus.
% Trop complexes, trop peu simplement implémentable, ou trop peu informatifs donnant lieu à des posteriors impropres... Il faut dans le cas ou ces problèmes sont limitant à l'étude altérer l'objectivisme et accepter de la conditionner à la réalisation de son but.
% C'est sous cette conslusion que nous sommes venus à la réponse à la \textbf{question ii} telle qu'elle est apportée par le \cref{chap:constrained-prior}.
% Les contraintes que nous y proposons à incorporer au cadre des problèmes definissant les priors de référence sont pensées pour peu déformer l'objectivité, tout en amélirant la praticabilité.
% Il s'agit dans ce chapitre de résultats théoriques nouveaux qui viennent enrichir la définition de priors qui s'appuient sur le canvas des priors de référence.


% Enfin, pour parvenir aux implementations couteuses numériquement nous proposons dans une dernière contribution une méthode d'approximation du prior de référence. 
% Cette méthode apporte une réponse à la \textbf{question iii}, et est développée dans le \cref{chap:varp}.
% Comme dans le chapitre qui le précède, elle vient avec un nouveau coùt à l'objectivisme puisque (i) elle impose la parametrisation implicite du prior au travers d'un réseau de neurones, et (ii) elle ne promet pas et ne garantie pas une approxiation parfaite du meilleur prior (i.e., le ``plus objectif'') parmis ces priors implicitement paramétrés.
% Cependant, cette implémentation nouvelle dans sa forme et son scope, permet un nouveau pas dans le domaine de l'inférence bayésienne objective (ou en tout cas proche d'être objective) en pratique.
% Il s'agit en effet du dernier chainons manquant pour allier ensemble les réponses des précédentes qustions dans un cadre tractable.



% Nous concluont ainsi, en ayant observé que l'objectivisme a un coût sur l'usage et vice versa.
% Cependant, nous souhaitons insister sur la capacité des travaux conduits dans la partie I de ce manuscrit à allier au maximum ces deux idéaux, le plus souvent en cherchant à atteindre la praticité minimale en perdant le moins d'objectivité possible.




% \section{2 2}




Objectivism in the Bayesian workflow refers to the quest for an elusive ideal. % within the Bayesian workflow. 
In general, the literature agrees that objectivity is best approached when data and the information they provide are prioritized over any subjective incorporation. It is difficult to assert definitively whether reference priors are truly objective priors; nevertheless, their construction is explicitly designed to minimize the influence of subjective thoughts.

Thus, one can view an objective prior as an ideal, and the framework of reference priors offers a clue on a quantification about how closely a given prior approaches that ideal. In \cref{chap:ref-generalized}, we sought to extend the scope of this measure by proposing a generalization of its fundamental definition. %the foundational problem that defines reference priors. 
Our results demonstrated that the solution to this generalized problem remains robust. In most cases, under minimal assumptions, the reference prior remains equivalent to Jeffreys prior. This investigation provided our answer to \textbf{question i}.

Actually, to address the many known limitations of Jeffreys prior, it is necessary to question the amount of objectivity that is truly expected for the problem of interest. %revisit the notion of objectivity itself. 
In practice, it is often found that the ``most objective'' priors are not always the most desirable. They may be too complex, impractical to implement, or insufficiently informative leading, for example, to improper posteriors. When these issues become limiting, objectivism must be revisited and conditioned by the practical goals of the analysis. This is the conclusion we arrived at in \cref{chap:constrained-prior}, which responds to \textbf{question ii}.
In that chapter, we propose incorporating constraints into the reference prior framework. These constraints are carefully designed to preserve objectivity as much as possible, while improving practical usability. They represent novel theoretical contributions that enrich the definition of priors built upon the foundation of reference priors, but tailored for real-world applicability.

Finally, in response to \textbf{question iii}, we address the challenge of computational cost associated with reference prior implementation. In this contribution, developed in \cref{chap:varp}, we propose a method for approximating reference priors. As for the preceding chapter, this method introduces a new cost to the objectivity, namely (i) it imposes the implicit parameterization of the prior through a neural network, and (ii) it does not guarantee that this parameterization will perfectly approximate the optimal (i.e., ``most objective'') prior. %—it offers a promising practical advancement.
This approach, novel in both form and scope, represents a significant step toward enabling objective (or near-objective) Bayesian inference in practice. It also serves as the last link that connects the answers to the preceding questions within a tractable framework.

We therefore conclude that objectivity and usability are often in tension: quest for objectivity comes at a cost to practical application, and vice versa. However, we emphasize the capacity of the work presented in the \cref{part:ref-theory} of this manuscript to reconcile these two ideals as much as possible —--generally by aiming for minimal practicality with the least possible sacrifice of objectivity.








% Bien que nous n'ayons pas de déféition d'un prior parfaitement objetif, les soltions apportées 



% Cette solution théorique qui prend la forme d'un prior bien fixe directement les limites de l'emploi de la théorie. 











% Le théorie des priors de références




\section{On the contributions to the estimation of seismic fragility curves}\label{sec:concl:frags}

% L'estimation de courbes de fragilité sismique est un sujet largement étudié puisque c'est une problématique concrète et qui a un impact réel.
% Cela la rend tout auant critique. Dans cette thèse, nous nous sommes limités au cas d'étude du modèle probi-lognormal omniprésent lorsque les données observées sont binaire.
% Ce modèle est en effet suffisemant riche pour que la conduite de travaux à son sujet ait suffit à l'obtention de contributions riches à la problématique.

% Tout d'abord, la construction de prior compréhensivement tailorés pour l'estimation bayésienne de ce modèle était une piste inexplorée, que nous avons décidé d'investiguer sous le scope des prios de références. C'est ainsi que nous avons construit, calculé, et pleinement étudié le prior de Jeffreys pour ce modèle. Plus qu'apporter ainsi une reponse à la \textbf{question iv}, nous avons dans le \cref{chap:prem} d'autant plus démontré que ce prior présentait des résultat bien plus performant en terme d'efficacité et d'efficience que deux approches competitives sérieuses de la littérature. %Bien qu'il n'ait pas été designé par la performance 

% Malgré tout, nous avons étudié par la même occasion la vraisemblance de ce modèle, nous soulignons que les taux asymptotiques de cette derniere n'avaient pas été étudiés comme ceci auparavant.
% Cette étude nous a permis de définir le phénomène de dégénrescence, observable dans les estimations issues de nombreuses méthodes. Il s'agit d'un phénomène qui survient lorsque les données informent peu le modèle, et cela met en péril l'emploi du prior objectif.
% On a alors proposé une nouvelle construction de prior, en s'appuyant toujours sur la théorie des priors de référence, et en particulier sur certaines des contributions que nous avions menées dans la première partie du manuscrit. En effet, la limite imposée par la dégénérescence se rapporte à une des limitations envisagées en \textbf{question ii}. C'est alors dans le \cref{chap:constrained-frags} que nous mettons en oeuvre l'application de cette méthodologie au modèle probit-lognormal des courbes de fragilité, apportant une réponse à la \textbf{question v}.


% Enfin, nous avons conclu notre étude en cherchant à optimiser au mieux toutes les sources d'informations qui viennent à disposition de l'estimation des courbes de fragilité dans le modèle probit-lognormal.
% C'est ainsi qu'à l'information a priori nous avons ajouté dans le \cref{chap:doe} une optimisation de l'information issue des données via une planification d'experience.
% Cette méthodologie peut-être vue comme une méthodologie ''ultime'' d'estimation des courbes de fragilité en ce sens, d'autant que, appuyée par un prior toujours de référence, on sait en assurer une certine auditabilité chère au milieu d'application. Elle apporte alors une réponse qui va même au delà de la \textbf{question vi}.
% Cette méthode répond en effet à beaucoup de problématiques du sujet à savoir: la dégénérescence et les estimations difficiliement exploitable qu'elle induit sont rapidement effacées, l'estimation est rapidement robuste et le prior est oublié, le biais induit par la consideration du modèle probit-lognormal est étudié et rapidement atteint.
% En notre sens, nous apportons la meilleur des réponse à l'estimation de courbe de fragilité à peu de données en employant ce modèle, et proprosons une limite à partir de laquelle un modèle différent est à favoriser.



% \section{3 2}


The estimation of seismic fragility curves is a widely studied topic due to its concrete and impactful applications, making it both relevant and critically important. In this thesis, we focused on the well-established probit-lognormal model, commonly used when observed data about structural responses are binary. The study of this model %is a problem that is sufficiently rich for the conduction of wor
is a sufficiently rich topic to lead to significant contributions to our problem

% to serve as a meaningful case study, and our investigations within its scope have led to significant contributions.

First, the construction of comprehensively tailored priors for Bayesian estimation under this model had remained largely unexplored. We chose to address this gap through the scope of reference prior theory. Specifically, we derived, computed, and thoroughly analyzed the Jeffreys prior for the probit-lognormal model. This work, developed in \cref{chap:prem} does not only answer \textbf{quesion iv}, but it also demonstrated that this prior performs significantly better in terms of both efficiency and accuracy compared to two serious competing approaches from the literature.

In parallel, we conducted a detailed study of the likelihood function of the model. Notably, the asymptotic behavior of this likelihood had not been analyzed in this form before. This analysis led us to identify and formalize the phenomenon of degeneracy, a critical issue that arises when the data provide insufficient information to inform the model. 
Degeneracy is frequently observed in estimates from many methods and
undermines the reliability of objective priors, among others. In response, we proposed a new construction of the prior, still supported by reference prior theory, and based on some of the theoretical contributions made in the \cref{part:ref-theory} of this manuscript. The limitation introduced by degeneracy is directly related to the concerns raised in \textbf{question ii}. The application of this new methodology to the probit-lognormal model is developed in \cref{chap:constrained-frags}, thereby providing a concrete response to \textbf{question v}.

Finally, we concluded our study by aiming to optimally integrate all sources of information available for estimating fragility curves within the probit-lognormal framework. This way, in \cref{chap:doe}, we augmented prior information with an optimization of data-based information through a formal experimental design strategy. This methodology can be seen as a kind of “ultimate” approach to fragility curve estimation. Moreover, since it is used alongside a reference prior, it guarantees a level of auditability that is valued in the SPRA. This work provides an answer that extends beyond \textbf{question vi}. % not only addresses **Question vi**, but extends beyond it.
Indeed,
the proposed methodology tackles many of the central issues identified throughout this research: (i) degeneracy and the resulting unreliable estimates are quickly resolved; (ii) robustness is achieved early in the inference process; (iii) the prior information becomes negligible in front of  the data-based one; and (iv) the bias inherent to the probit-lognormal model is characterized and rapidly quantified. We believe this method offers the most effective approach to fragility curve estimation under conditions of limited data when using this model. Furthermore, we provide a practical threshold beyond which it becomes advisable to consider an alternative model.















% Bien qu'omniprésent, 












\section{General outlook}\label{sec:concl:general}


% Finalement, plus que contribuer d'une part à la théorie des priors de référence et d'autre part à l'estimation de courbes de fragilité sismique, nous avons dans cette thèse mis en lumière le lien entre ces deux mondes. Au final, les recherches sur chacuns de ces deux axes ce sont alimentées l'une et l'autre.
% Cette  est essentielle puisqu'elle démontre %à la fois l'intérêt d'avancer dans une voie  
% un intérêt à la motivation technique pour une  théorique, ainsi qu'un intérêt à l'attrait abstrait pour la conduction d'une étude pratique.
% % Les travaux ercutant dans leur individualité sont loins d'être isolé et s'inscrivent dans un tout motivé, réel et startégique
% % En ce sens,


% Dans notre cas, l'application des priors de références à l'estimation de courbes de fragilité sismiques est un problème qui n'a pas d'antécédent dans la litérature. Il en est de même du problème du développement de la théorie des priors de référence motivé par les modèles de courbes de fragilité. 
% Il s'agit des deux problèmes qui caractérisent les liens entre les chapitres de cette thèse, et qui ont été traités en parallèles.
% %
% Explicitement, 
% l'étude de l'application des priors de référence au cadre des courbes de fragilité a commencé avec le calcul du prior de Jeffreys pour le modèle probit-lognormal, tel que développé au chapitre 8. En effet, le prior de Jeffreys était déjà admis dans la littérature comme la solution au problème définissant les priors de référence en général. 
% Avec l'idée de s'assurer de l'objectivité de ce prior au delà du cadre historique de la théorie, nous avons motivé l'étude de la généralisation de ce dernier, ce qui nous a mené au travail développé au chapitre 4.
% Aussi, les calculs conduit au chapitre 8 pour l'application du prior de Jeffreys au modèle probit-lognormal ont mis en evidence certaines de ses limitations qui limitait alors son usage. Ces limitations, qui relèvent de la propreté a posteriori et de la complexité numérique, ont motivé le questionnement de la théorie sur des méthodes qui éviteraient aux priors de référence de tomber dans ces travers. C'est ainsi que les travaux du chapitre 7, qui sont directement appliqués au chapitre 9, sont nés et solutionnent pleinement ces limitations.
% Le chapitre 10 représente une combinaison alors ``ultime'' de ces éléments, combinés aussi à l'optimization de l'information issue des données en plus de celle a priori, donnant lieu à une estimation très bonne des courbes de fragilité sous le modèle probit-lognormal.
% Enfin, le chapitre 6 présente un attribut plus ouvert, en proposant une méthode d'approximation du prior qui se veut plus efficace encore mais qui reste imparfaite en terme de fiabilité. C'est pourquoi au delà démonstration de ses capacités faite au chapitre 9, elle n'a pas été employée pour l'étude ``finale'' menée au chapitre 10.


% \section{t}


% Finally, t
This thesis contributes not only to the theory of reference priors on one hand, and to the estimation of seismic fragility curves on the other, but more importantly, it highlights the connection between these two domains. The research carried out along each axis has fed into and enriched the other. This reflection is essential, as it underscores the value of technical motivations in guiding theoretical development, and conversely, the practical relevance of abstract considerations. % for conducting practictheoretical insights.

In our case, the application of reference priors to the estimation of seismic fragility curves is a problem with no precedent in the existing literature. Similarly, the development of reference prior theory driven by the specific context of fragility curve models is equally novel. These two parallel problems define the core linkages between the chapters of this thesis, and have been addressed in tandem throughout the work.

Specifically, the study of reference prior application began with the derivation of the Jeffreys prior for the probit-lognormal model, as presented in \cref{chap:prem}. Indeed, Jeffreys prior was already recognized in the literature as the solution to reference prior problem under general settings. With the idea to support even more
%  Seeking to ensure the 
 objectivity of this prior beyond the historical framework of the theory, we led ourselves to revisit and generalize the foundational reference prior problem. This motivated the work carried out in \cref{chap:ref-generalized}, which validates the use of Jeffreys prior for conducting an objective Bayesian study. %objective aspect of .  %Since we found out that the Jeffreys prior is robust 

Furthermore, the derivations performed in \cref{chap:prem} revealed the limitations of Jeffreys prior when applied to the probit-lognormal model. These limitations being related to posterior improper aspect and computational complexity. 
These issues prompted a theoretical inquiry into how the reference prior framework might be adapted to avoid such drawbacks. This led to the development of the methods introduced in \cref{chap:constrained-prior}, which were directly applied in \cref{chap:constrained-frags} to address and resolve those limitations.


\Cref{chap:varp} offers a more open-ended contribution by proposing an approximation method for reference priors aimed at improving computational efficiency. While this method shows promise, it remains imperfect in terms of reliability. Therefore, although its potential was demonstrated in \cref{chap:constrained-frags}, it was not used in the ``final'' practical study conducted in \cref{chap:doe}.


Finally, \cref{chap:doe} represents a sort of ``ultimate'' synthesis, combining all these elements, and including the optimization of data-based information in addition to prior information. This work resulted in a highly effective estimation methodology for fragility curves under the probit-lognormal model.



% In summary, the interplay between theoretical development and practical application has been a driving force throughout this thesis, reinforcing the depth and relevance of the contributions made in both fields.






% Ce sont ainsi deux problèmes qui ressortent des travaux mené dans cette thèse et 


% Pour y parv


% était un problème



% Ici, les problèmes observés lors de


% En vue d'un estimation objective des courbes de fragilité sismiques, le travail prime alors comme un tout.
% Et nous dirions que nous avons proposé un cadre général pour à la fois construire des priors objectifs, à la fois les rendre 
% Le tout







\section{Perspectives and connections with other fields}\label{sec:concl:perspectives}

% De nombreuses perspectives se voient ouvertes pas les travaux menés dans cette thèse et les résultats obtenus.
% % On rappelle 


% \paragraph{Sur le dévelopement de la théorie des priors de référence}
% D'une part il est possible d'approfondir la théorie des priors de références en analyse Bayésienne objective. 
% % Nous dirions que son cadre saurait encore être enrichi. notamment, on pourrait chercher à résoudre le problème définissant le prior de référence lorsuqe la mesure de dissimilarité qui déféinit l'information mutuelle ne se limite pas à des $f$-divergences. Aussi, bien que le cadre bayésien en lui même a étét quelque peu requestionné dans la partie I de cette thèse, quelques limites restent
% Les priors de référence ne sont en général pas uniques, qui plus est lorsque ceux-ci peuvent être définit conditionellement à des contraintes comme nous le proposons dans la partie I de ce manuscrit.
% %Nous avancerions que sa principale limite est qu'elle manque d'outil permettant la comparaison et l'évaluation de différents priors
% Alors, la théorie reste limitée par son manque d'outil permettant la comparaison et l'évaluation de différents priors.
% %
% On se demande plus précisement si (i) est-it possible de définir une mesure comparant la nature objective de priors au sens de la théorie des priors de référence ? (ii) could such measure be extended to the common cases where the prior
% --—and sometimes even the posterior—-- is improper? (iii)
% Could we define ``the most'' appropriate way to
% constrain a prior regarding specific practical needs?

% An answer to the first question necessitates further developments of the mutual information
% expression. It might be possible, for instance,
% to push
% further the current asymptotic form that it takes   to determine and
%  recover a ``quantity of objectivity'' that it measures. %Additionally, leveraging the connection
% % we unveiled in \cref{chap:ref-generalized} between this criterion and the field of Global Sensitivity Analysis (GSA), other
% % extensions of the mutual information expression can be invented, such as by considering dis-
% % similarity measures based on RKHS. This path would allow us to benefit from implementation
% % techniques developed for GSA.

% The second question could be approached with a further investigation of an appropriate framework that formalizes the use of improper priors in Bayesian modeling. 
% This point was already questioned in the part I of this manuscript with a construction that is compatible with the one of \citet{bioche_approximation_2015}, for instance. Yet topological considerations and implications on the reference prior theory have been poorly addressed in this thesis.

% %the topology that the author suggests makes the proper priors too ``close'' from the improper ones in our sense. As an example, it does 


% % is not fully compatible with the reference prior theory tools.
% The third question, finally, has a very open nature. It can be seen as an application of the eventual answers of the last two questions.

%. An aim would be to enrich open paths of the literature
% with the help of the developments of the two previous questions





% \paragraph{Sur l'explicabilité des incertitutdes des modèles numériques}


% Aussi, notre intrinsection conduite entre priors de référence et courbes de fragilté dans des cas concrêts a démontré comment cette théorie et ses méthodes sait constituer un point d'encrage à l'explicabilité d'estimations dans un modèle numérique. 


% Indeed, ``black-box'’ models —--such as ML or numerical simulations
% algorithms--— embed both uncertainty that comes from the data (aleatoric), and uncertainty that
% comes from the model (epistemic), see \cite{hullermeier_aleatoric_2019}, or \cref{chap:intro-english}.
% In the same way as for the probit-lognormal model of seismic fragility curves, the explainability of the `pipeline' necessitates quantifying those, and especially handling the second. 
% Some works in the literature address the explainability of the uncertainties embedded in these models within a frequentist paradigm. We can cite \cite{il_idrissi_quantile-constrained_2024,wimmer_quantifying_2023}, which rely on sensitivity analysis tools to tackle the problem.

% It is also possible to treat the problem under the Bayesian paradigm.
% This may be done by defining an appropriate prior on the model or its parameters and measuring the sensitivity a posteriori of the quantities of interest to the model itself. 
% This path would raise the following questions that would  leverage the definition of objective priors, as priors that are designed to let the data guide the posterior distribution:
% (i) what would be the appropriate
% prior to define in such context and how to approximate it efficiently? (ii) is the result sensitive in
% the prior? (iii) how to define and derive the aforementioned a posteriori sensitivity?

% Naturally,
%   research on these questions would also
% leverage works on the perspectives mentioned earlier on the development of the reference prior theory. % suggested  in sections 1 and
% % 2 of this document, as it raises several fundamental questions: 








% \paragraph{Sur l'estimation de courbes de fragilité sismique}

% Enfin, concernant l'estimation de courbe de fragilité sismique, peu d'appronfissement du modèle probit-lognormal sont envisagé. Cependant, l'estimation selon d'autres cadres, d'autres datasets et/ou d'autres modèles reste ouverte et reste un cadre dans lequel une étude objective peut-être conduite. %Il semblerait que notre étude démontre l'aspect limitant de la considération de données seulement binaires de la réponse de la structure.
% %Dans notre étude
% Au sujet des dataset, une première voie ouverte est la considération d'IM multiples en entrée du modèle, donnant lieu à plus d'information sur le signal auquel la structure est soumise. Une seconde voie est la considération de sorties non binaires, telles que des EDPs lorsque'elles sont disponibles comme information sur la réponse de la strucutre.
% Différents travaux ont déjà étudié ces perspectives, en usant de nombreux modèles. A notre conaissance, aucun n'abordent le sujet selon une approche d'allocation ``optimale'' de l'information en analyse bayésienne (i.e., en couplant planification d'expérience et prior spécifiquement construit).

% Finalement, différents modèles apporteront chacun un niveau de fidélité différents au véritable comportement de la structure.
% A ce titre nous mentionnons l'usage d'un modèle différent du modèle probit-lognormal en annexe B, considéré comme de plus basse fidélité. L'emploi de différentes analyses, de différentes fidélités a priori pourrait enrichir l'information donnant lieu à l'estimation finale de la courbe de fragilité.





% % Dans le cadre de la considération de données seulement binaires de la réponses de la structure, il est envisageable d'acquérir de l'information supplémentaire en considérant 



% % On peut citer...







% \section{t}






The work conducted in this thesis opens several avenues for future research and application. Three possible directions are proposed in what follows %These perspectives span three main directions: advancing the theory of reference priors, improving the explainability of numerical models, and enhancing the estimation of seismic fragility curves.



\paragraph{Further development of the reference prior theory}

First, it is possible to deepen the reference prior theory for objective Bayesian analysis. Reference priors are generally not unique, and this non-uniqueness becomes even more prominent when these priors are authorized to be defined conditionally on some constraints as proposed in the \cref{part:ref-theory} of this manuscript. In our opinion, the current theory remains limited by the lack of tools to compare and evaluate different priors systematically. This leads to several open questions:
(i) Can we define a measure to compare the objectivity of priors in the context of reference prior theory? (ii)
Could such a measure be extended to settings where the prior —--and possibly even the posterior—-- is improper? (iii) Can we define the ``most appropriate'' way to constrain a prior, tailored to specific practical needs?

An answer to the first question necessitates further developments of the mutual information expression.
It might be possible, for instance, to push further the current asymptotic form that it takes to determine and
recover a ``quantity of objectivity'' that it measures.

The second question calls for a robust theoretical framework to formalize the use of improper priors in Bayesian modeling. This issue was partially addressed in the \cref{part:ref-theory} of this manuscript with a construction that aligns with that of \citet{bioche_approximation_2015}, for instance. However, a deeper treatment of the topological implications and their effect on reference prior theory remains to be explored.

The third question is broader in scope and can be seen as a natural application of answers to the first two. %It opens the door to a theory of constrained objective priors that remains flexible while retaining interpretability.



\paragraph{Application to the explainability of uncertainty in numerical models}

Second, our coupling of reference priors with the estimation of seismic fragility curves in real-world scenarios has shown how this theoretical framework can anchor the explainability of outputs from numerical models. Indeed, ``black-box'' models —--such as machine learning algorithms or numerical simulations—-- embed both aleatoric (data-related) and epistemic (model-related) uncertainties, as discussed in \cite{hullermeier_aleatoric_2019} and in Chapter 1.
As in the case of the probit-lognormal model used for seismic fragility curves estimation, explaining the behavior of the ``pipeline'' requires quantifying those, and especially handling the second. %with particular attention to epistemic sources. 
Some works in the literature address the explainability of the uncertainties embedded in these
models within a frequentist paradigm. We can cite
\cite{il_idrissi_quantile-constrained_2024,wimmer_quantifying_2023}, which rely on
sensitivity analysis tools to tackle the problem.

It is also possible to treat the problem under the Bayesian paradigm.
This Bayesian path would involve defining appropriate priors over the model or its parameters and evaluating the posterior sensitivity of key quantities of interest to those prior assumptions. This path would raise the following questions that would leverage the definition
of objective priors, as priors that are designed to let the data guide the posterior distribution:
(i) What would
be the appropriate prior to define in such context and how to approximate it efficiently? (ii) Would the result be 
sensitive in the prior? (iii) How to define and derive the aforementioned \emph{a posteriori} sensitivity?

Naturally, research on these questions would also benefit from works on the perspectives mentioned earlier
on the development of the reference prior theory.



\paragraph{Towards better estimations of seismic fragility curves}


Regarding the estimation of seismic fragility curves, the probit-lognormal model has been the main focus of this thesis,
% , and few refinements of the probit-lognormal model are envisaged.
and its further refinement it not envisages.
However, the estimation according to
of other frameworks, datasets and/or models remains open and remains a framework in which an objective study can be conducted.
% However, estimations under 
% However, applying the methodology to different modeling frameworks, datasets, and structural behaviors remains a rich area for objective Bayesian analysis.

First, one potential direction is to consider multiple IMs as inputs to the statistical model, allowing for a more comprehensive information on the seismic excitation on a structure. Second, instead of binary outcomes, continuous structural responses --—such as EDPs—-- could be incorporated when available, since they embed richer information describing the impact of the input onto the structure. These directions have been explored using various models in existing literature. However, to our knowledge, none of these studies adopt a Bayesian approach that ``optimally'' allocates the information within the workflow (e.g., through the combination of an experimental design with a specifically tailored prior).
%both prior and data-driven information—i.e., one that couples **experimental design** with a **tailored reference prior**.

Lastly, we conclude by observing that different models naturally offer different fidelity levels in approximating the real fragility of the studied structural. As an example, we mention that we discuss in \cref{app:chap:ESAIM} the use of 
a lower-fidelity model compared to the probit-lognormal model for fragility curves estimation. Combining multi-fidelity analyses, each bringing complementary information, could enhance the robustness and informativeness of the final fragility curve estimation.








\section{Final words}\label{sec:concl:final}


% Nous n'avons pas dans cette thèse contruit le prior objecctif par essence, ni même estimé la courbe de fragilité la plus exacte des structures mécaniques étudiées.
% Cependant, en s'adressant à la fois séparément mais aussi très poreuseent à ces deux questions nous nous sommes approché au plus près possible de ces deux idéaux.
% %Nous croyons que ce raprochement s'est fait en contribuant à la fois indépendament au domaine des courbes de fragilité 
% % Nous croyons que
% Nous concluons en appuyant que les méthodologies nouvelles apportées par cette thèse, ainsi que les résultats théoriques, formels, et expérimentaux qui les accompagnent démontre de leur capacité à intelligement balancer les raprochements aux deux idéaux de sorte à répondre pour le mieux à la problématique définie par le sujet de thèse.
% % Ceci étant d'autant plus appuyé par le fait qu'une part substantielle de ces résultats a été publiée après peer reviewing.

% % Nous insistons que tous nos résultats ont été longuement analysés et relus, et tous 
% %De cette manière, ces résultats, dont la portée dans leur domaine a déjà ou est en cours d'être jugée par le peer-reviewing, ont une portée qui va au delà.Chacun de ces résultats
% Pour chacun des résultats qui fait parti de ceux qui articulent cette conclusion, nous insistons sur la rigueur apporté à notre approche, de sorte à appuyer au mieux la percutance du travail à la problématique générale.
% Chacune des contributions individuelle a été soumise (et est dans certain cas déjà validées) à la communauté scientifique.
% Le sérieux de notre démarche et de ces derniers mots est ainsi supporté.


% \section{t}



In this thesis, we did not construct the definitive objective prior, nor did we estimate the most exact fragility curve for the mechanical structures studied. However, by addressing both of these questions individually and in close interaction, we have moved as close as possible to these two ideals.

We conclude by emphasizing that the new methodologies introduced in this work, along with their accompanying theoretical, formal, and experimental results, demonstrate a thoughtful balance between these two objectives. This balance has been key in providing the most effective response to answer the problem defined by this thesis.

For each result that forms the foundation of this conclusion, we insist that we have maintained a high level of rigor in our approach, in order to support the impact of the work to the problem context. Each individual contribution has been submitted (and in some cases already published) within the scientific community. These final words are thus backed by the seriousness and integrity with which this research has been conducted.







% \section{Outlook}


% \adjusmptc

\newpage\thispagestyle{plain}

