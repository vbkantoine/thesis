
\begin{mtchideinmaintoc}[-1]
\chapter{Conclusion \& perspectives}
\end{mtchideinmaintoc}
% \refstepcounter{chapter}
% \addcontentsline{toc}{chapter}{}%
% \markboth{Conclusion \& perspectives}{}
% \addtocounter{chapter}{1}
% \addstarredchapt{}
% \mtcaddchapter
% \adjustmtc
%\decrementmtc


\begin{abstract}[Foreword]
    In this part, we provide a general conclusion to this thesis and this manuscript. Following the structure below, we express the conclusive thoughts that terminate this work.
\end{abstract}


\minitoc

% \listofconcSec
%\parttoc
% \adjustmtc
% \renewcommand{\thesection}{}

% \mtcaddchapter
% \setcounter{section}{0}
% \adjuststc
% \section*{test}
% \addtocontents{lof}{test}

% \concSec{Test}






% \section{Summary}

% % Dans le chapitre 1, nous avons apporté un a priori au contenu de ce manuscrit.
% In this last part, we propose an \emph{a posteriori} analysis of the research conducted during this thesis, and developed in  \cref{part:ref-theory} and \cref{part:spra}.
% The \emph{a priori} of this content was elucidated in \cref{chap:intro-english} through an analysis of the problem and an analysis of the approaches that it invited, in our opinion, to explore.
% This way, we remind that 6 open questions where elucidated?, we recapitualte those below:\\[-5pt]





% \ques{i}{: How can one define and support the objectivity of a prior?}


% \ques{ii}{: What are the limits of the implementation of such non-informative priors, and how to reconcile their use with practical needs?}

% \ques{iii}{: How to build and derive such priors in practice?}

% \ques{iv}{: in the context of SPRA, what are the forms of these objective priors within a model for seismic fragility estimation}

% \ques{v}{: What are the consequences entailed by the lack of information in this model and how to solve them?}

% \ques{vi}{: How to finally leverage the most the different sources of information in the whole Bayesian workflow of the studied model?}





% %Naturally, these 6 questions address two different domains that are distinc from a first sight?




% %Ici, nous conduisons l'analyse a posteriori, au regard des résultats ui font suite au deux parties qui précèdent.


% % Pour la construction de l'a priori à ce contenu, nous avons conduit au chapitre 1 une analyse de la problématique, et une reflexion sur les approches qu'elle semblait inviter à mener.
% % De cette sorte, on rappelle que cette analyse donnait lieu à l'expression de 6 questions ouverts, que nous récapitulons ci-dessous.









% Naturallement, ces 6 questions adressent 2 mondes a première vue distincts : la théorie des priors de référence ainsi que l'étude de courbes de fragilité sismique.
% Ainsi, les questions, l'analyse de sujet et de la problématique a d'abord été enrichie d'une analyse profonde de l'etat de l'art de chacun de ces deux domaines.
% Cette distinction a permis de structurer les travaux conduit pendant cette thèse ainsi que le manuscrit selon s'il se ratachait plutôt à l'un ou à l'autre des deux domaines. 
% C'est pourquoi dans les deux sections qui suivent (\cref{sec:concl:refpr,sec:concl:frags}) nous faerrons le point sur les aspects contributifs de ce travail 
% à chacun des deux sujets.
% Cependant, nous appuyons aussi l'aspect contributif propre de l'interaction entre les deux sujet, novateur et qui est la finalité de la réponse à la problématique du sujet, nous en parlons en \cref{sec:concl:general}.
% Ces travaux sont vaste etpuisqu'ils appréhendent un large pannel de choses, ils ouvrent par la même grand nombre de pistes. Nous discontons de celles-ci en \cref{sec:concl:perspectives}.







\section{Summary}




In this final chapter, we present an \emph{a posteriori} analysis of the research conducted throughout this thesis, that we developed in \cref{part:ref-theory} and \cref{part:spra}. The \emph{a priori} takes the form here of the 
motivations and preliminary considerations that were outlined in \cref{chap:intro-english}, through a detailed examination of the problem and a discussion of the methodological avenues that, in our view, invited for further exploration.
To structure this investigation, we recall that we identified six guiding questions that shaped the architecture of the thesis. These are recapitulated below:\\[-5pt]


\ques{i}{: How can one define and support the objectivity of a prior?}


\ques{ii}{: What are the limits of the implementation of such non-informative priors, and how to reconcile their use with practical needs?}

\ques{iii}{: How to build and derive such priors in practice?}

\ques{iv}{: in the context of SPRA, what are the forms of these objective priors within a model for seismic fragility estimation}

\ques{v}{: What are the consequences entailed by the lack of information in this model and how to solve them?}

\ques{vi}{: How to finally leverage the most the different sources of information in the whole Bayesian workflow of the studied model?}

These six questions naturally span two domains that may appear distinct at first glance: the reference prior theory and the study of seismic fragility curves. Accordingly, the initial formulation of the research questions and the analysis of the core problem were complemented by an in-depth review of the state of the art in both fields.

This distinction helped to structure the work carried out in this thesis, as well as the organization of the manuscript itself, with each part aligned more closely with one of the two domains. Consequently, in the following sections (\cref{sec:concl:refpr,sec:concl:frags}), we reflect separately on the contributions made to each area. However, particular emphasis is also placed on the original contribution outlined from the interaction between the two domains. This interaction represents an innovative aspect that ultimately address the main problem addressed by the thesis, we discuss this point in \cref{sec:concl:general}.
These works are various and given the wide range of issues they engage with, several open directions for future research emerge. These are discussed in \cref{sec:concl:perspectives}.




\section{On the contributions to the reference prior theory}\label{sec:concl:refpr}

% L'objectivisme
% % La qualification d'objectivité 
% reste un idéal pas très bien défini dans le workflow Bayésien. La litérature s'accorde plutôt en règle général sur le fait qu'on s'en approche lorsque l'on favorise les données et les informations qu'elles apportent en ce sens devant tout autre incorporation subjective.
% Il est difficile de conclure si les priors de référence sont des priors objectifs, néanmoins, leur construction cherche à minimiser l'introduction de subjectivité par le prior.

% Au final, on peut donc voir un prior objectif comme un idéal, et le cadre des priors de référence apporte un indice de quantification d'une proximité de cet idéal.
% On a tenté dans le \cref{chap:ref-generalized} d'étendre la portée de cet indice, en cherchant à le rendre plus général, nos résultats ont prouvé que la solution du problème définissant les priors de référence %(i.e., quel est le prior de ref) 
% est robuste à notre généralisation. Dans la pluspart des cas, sous un minimum d'hypothèses, le prior de référence est le prior de Jeffreys.
% On répond alors ici à la \textbf{question i}. 


% Pour contrer les limitations connues et nombreuses du prior de Jeffreys on doit alors se replonger dans le questionnement de l'objectivité voulue. Finalement, nombreuses sont les études ou les priors ``les plus obectifs possible'' ne sont pas tant ceux qui sont attendus.
% Trop complexes, trop peu simplement implémentable, ou trop peu informatifs donnant lieu à des posteriors impropres... Il faut dans le cas ou ces problèmes sont limitant à l'étude altérer l'objectivisme et accepter de la conditionner à la réalisation de son but.
% C'est sous cette conslusion que nous sommes venus à la réponse à la \textbf{question ii} telle qu'elle est apportée par le \cref{chap:constrained-prior}.
% Les contraintes que nous y proposons à incorporer au cadre des problèmes definissant les priors de référence sont pensées pour peu déformer l'objectivité, tout en amélirant la praticabilité.
% Il s'agit dans ce chapitre de résultats théoriques nouveaux qui viennent enrichir la définition de priors qui s'appuient sur le canvas des priors de référence.


% Enfin, pour parvenir aux implementations couteuses numériquement nous proposons dans une dernière contribution une méthode d'approximation du prior de référence. 
% Cette méthode apporte une réponse à la \textbf{question iii}, et est développée dans le \cref{chap:varp}.
% Comme dans le chapitre qui le précède, elle vient avec un nouveau coùt à l'objectivisme puisque (i) elle impose la parametrisation implicite du prior au travers d'un réseau de neurones, et (ii) elle ne promet pas et ne garantie pas une approxiation parfaite du meilleur prior (i.e., le ``plus objectif'') parmis ces priors implicitement paramétrés.
% Cependant, cette implémentation nouvelle dans sa forme et son scope, permet un nouveau pas dans le domaine de l'inférence bayésienne objective (ou en tout cas proche d'être objective) en pratique.
% Il s'agit en effet du dernier chainons manquant pour allier ensemble les réponses des précédentes qustions dans un cadre tractable.



% Nous concluont ainsi, en ayant observé que l'objectivisme a un coût sur l'usage et vice versa.
% Cependant, nous souhaitons insister sur la capacité des travaux conduits dans la partie I de ce manuscrit à allier au maximum ces deux idéaux, le plus souvent en cherchant à atteindre la praticité minimale en perdant le moins d'objectivité possible.




% \section{2 2}




Objectivism in the Bayesian workflow refers to the quest for an elusive ideal. % within the Bayesian workflow. 
In general, the literature agrees that objectivity is best approached when data and the information they provide are prioritized over any subjective incorporation. It is difficult to assert definitively whether reference priors are truly objective priors; nevertheless, their construction is explicitly designed to minimize the influence of subjective thoughts.

Thus, one can view an objective prior as an ideal, and the framework of reference priors offers a clue on a quantification about how closely a given prior approaches that ideal. In \cref{chap:ref-generalized}, we sought to extend the scope of this measure by proposing a generalization of its fundamental definition. %the foundational problem that defines reference priors. 
Our results demonstrated that the solution to this generalized problem remains robust. In most cases, under minimal assumptions, the reference prior remains equivalent to Jeffreys prior. This investigation provided our answer to \textbf{question i}.

Actually, to address the many known limitations of Jeffreys prior, it is necessary to question the amount of objectivity that is truly expected for the problem of interest. %revisit the notion of objectivity itself. 
In practice, it is often found that the ``most objective'' priors are not always the most desirable. They may be too complex, impractical to implement, or insufficiently informative leading, for example, to improper posteriors. When these issues become limiting, objectivism must be revisited and conditioned by the practical goals of the analysis. This is the conclusion we arrived at in \cref{chap:constrained-prior}, which responds to \textbf{question ii}.
In that chapter, we propose incorporating constraints into the reference prior framework. These constraints are carefully designed to preserve objectivity as much as possible, while improving practical usability. They represent novel theoretical contributions that enrich the definition of priors built upon the foundation of reference priors, but tailored for real-world applicability.

Finally, in response to \textbf{question iii}, we address the challenge of computational cost associated with reference prior implementation. In this contribution, developed in \cref{chap:varp}, we propose a method for approximating reference priors. As for the preceding chapter, this method introduces a new cost to the objectivity, namely (i) it imposes the implicit parameterization of the prior through a neural network, and (ii) it does not guarantee that this parameterization will perfectly approximate the optimal (i.e., ``most objective'') prior. %—it offers a promising practical advancement.
This approach, novel in both form and scope, represents a significant step toward enabling objective (or near-objective) Bayesian inference in practice. It also serves as the last link that connects the answers to the preceding questions within a tractable framework.

We therefore conclude that objectivity and usability are often in tension: quest for objectivity comes at a cost to practical application, and vice versa. However, we emphasize the capacity of the work presented in the \cref{part:ref-theory} of this manuscript to reconcile these two ideals as much as possible —--generally by aiming for minimal practicality with the least possible sacrifice of objectivity.








% Bien que nous n'ayons pas de déféition d'un prior parfaitement objetif, les soltions apportées 



% Cette solution théorique qui prend la forme d'un prior bien fixe directement les limites de l'emploi de la théorie. 











% Le théorie des priors de références




\section{On the contributions to the estimation of seismic fragility curves}\label{sec:concl:frags}

% L'estimation de courbes de fragilité sismique est un sujet largement étudié puisque c'est une problématique concrète et qui a un impact réel.
% Cela la rend tout auant critique. Dans cette thèse, nous nous sommes limités au cas d'étude du modèle probi-lognormal omniprésent lorsque les données observées sont binaire.
% Ce modèle est en effet suffisemant riche pour que la conduite de travaux à son sujet ait suffit à l'obtention de contributions riches à la problématique.

% Tout d'abord, la construction de prior compréhensivement tailorés pour l'estimation bayésienne de ce modèle était une piste inexplorée, que nous avons décidé d'investiguer sous le scope des prios de références. C'est ainsi que nous avons construit, calculé, et pleinement étudié le prior de Jeffreys pour ce modèle. Plus qu'apporter ainsi une reponse à la \textbf{question iv}, nous avons dans le \cref{chap:prem} d'autant plus démontré que ce prior présentait des résultat bien plus performant en terme d'efficacité et d'efficience que deux approches competitives sérieuses de la littérature. %Bien qu'il n'ait pas été designé par la performance 

% Malgré tout, nous avons étudié par la même occasion la vraisemblance de ce modèle, nous soulignons que les taux asymptotiques de cette derniere n'avaient pas été étudiés comme ceci auparavant.
% Cette étude nous a permis de définir le phénomène de dégénrescence, observable dans les estimations issues de nombreuses méthodes. Il s'agit d'un phénomène qui survient lorsque les données informent peu le modèle, et cela met en péril l'emploi du prior objectif.
% On a alors proposé une nouvelle construction de prior, en s'appuyant toujours sur la théorie des priors de référence, et en particulier sur certaines des contributions que nous avions menées dans la première partie du manuscrit. En effet, la limite imposée par la dégénérescence se rapporte à une des limitations envisagées en \textbf{question ii}. C'est alors dans le \cref{chap:constrained-frags} que nous mettons en oeuvre l'application de cette méthodologie au modèle probit-lognormal des courbes de fragilité, apportant une réponse à la \textbf{question v}.


% Enfin, nous avons conclu notre étude en cherchant à optimiser au mieux toutes les sources d'informations qui viennent à disposition de l'estimation des courbes de fragilité dans le modèle probit-lognormal.
% C'est ainsi qu'à l'information a priori nous avons ajouté dans le \cref{chap:doe} une optimisation de l'information issue des données via une planification d'experience.
% Cette méthodologie peut-être vue comme une méthodologie ''ultime'' d'estimation des courbes de fragilité en ce sens, d'autant que, appuyée par un prior toujours de référence, on sait en assurer une certine auditabilité chère au milieu d'application. Elle apporte alors une réponse qui va même au delà de la \textbf{question vi}.
% Cette méthode répond en effet à beaucoup de problématiques du sujet à savoir: la dégénérescence et les estimations difficiliement exploitable qu'elle induit sont rapidement effacées, l'estimation est rapidement robuste et le prior est oublié, le biais induit par la consideration du modèle probit-lognormal est étudié et rapidement atteint.
% En notre sens, nous apportons la meilleur des réponse à l'estimation de courbe de fragilité à peu de données en employant ce modèle, et proprosons une limite à partir de laquelle un modèle différent est à favoriser.



% \section{3 2}


The estimation of seismic fragility curves is a widely studied topic due to its concrete and impactful applications, making it both relevant and critically important. In this thesis, we focused on the well-established probit-lognormal model, commonly used when observed data about structural responses are binary. The study of this model %is a problem that is sufficiently rich for the conduction of wor
is a sufficiently rich topic to lead to significant contributions to our problem

% to serve as a meaningful case study, and our investigations within its scope have led to significant contributions.

First, the construction of comprehensively tailored priors for Bayesian estimation under this model had remained largely unexplored. We chose to address this gap through the scope of reference prior theory. Specifically, we derived, computed, and thoroughly analyzed the Jeffreys prior for the probit-lognormal model. This work, developed in \cref{chap:prem} does not only answer \textbf{quesion iv}, but it also demonstrated that this prior performs significantly better in terms of both efficiency and accuracy compared to two serious competing approaches from the literature.

In parallel, we conducted a detailed study of the likelihood function of the model. Notably, the asymptotic behavior of this likelihood had not been analyzed in this form before. This analysis led us to identify and formalize the phenomenon of degeneracy, a critical issue that arises when the data provide insufficient information to inform the model. 
Degeneracy is frequently observed in estimates from many methods and
undermines the reliability of objective priors, among others. In response, we proposed a new construction of the prior, still supported by reference prior theory, and based on some of the theoretical contributions made in the \cref{part:ref-theory} of this manuscript. The limitation introduced by degeneracy is directly related to the concerns raised in \textbf{question ii}. The application of this new methodology to the probit-lognormal model is developed in \cref{chap:constrained-frags}, thereby providing a concrete response to \textbf{question v}.

Finally, we concluded our study by aiming to optimally integrate all sources of information available for estimating fragility curves within the probit-lognormal framework. This way, in \cref{chap:doe}, we augmented prior information with an optimization of data-based information through a formal experimental design strategy. This methodology can be seen as a kind of “ultimate” approach to fragility curve estimation. Moreover, since it is used alongside a reference prior, it guarantees a level of auditability that is valued in the SPRA. This work provides an answer that extends beyond \textbf{question vi}. % not only addresses **Question vi**, but extends beyond it.
Indeed,
the proposed methodology tackles many of the central issues identified throughout this research: (i) degeneracy and the resulting unreliable estimates are quickly resolved; (ii) robustness is achieved early in the inference process; (iii) the prior information becomes negligible in front of  the data-based one; and (iv) the bias inherent to the probit-lognormal model is characterized and rapidly quantified. We believe this method offers the most effective approach to fragility curve estimation under conditions of limited data when using this model. Furthermore, we provide a practical threshold beyond which it becomes advisable to consider an alternative model.















% Bien qu'omniprésent, 












\section{General outlook}\label{sec:concl:general}


Finalement, plus que contribuer d'une part la théorie des priors de référence et d'autre part à l'estimation de courbes de fragilité sismique, nous avons dans cette thèse mis en lumière le lien entre ces deux mondes. Au final, les recherches sur chacuns de ces deux axes ce sont alimantées l'un l'autre.


Cette reflexion est essentielle puisqu'elle démontre à la fois l'intérêt d'avancer dans une voie  % Les travaux ercutant dans leur individualité sont loins d'être isolé et s'inscrivent dans un tout motivé, réel et startégique



Ici, les problèmes observés lors de


En vue d'un estimation objective des courbes de fragilité sismiques, le travail prime alors comme un tout.
Et nous dirions que nous avons proposé un cadre général pour à la fois construire des priors objectifs, à la fois les rendre 
Le tout







\section{Perspectives and connections with other fields}\label{sec:concl:perspectives}

De nombreuses perspectives se voient ouvertes pas les travaux menés dans cette thèese et les résultats obtenus.

D'une part sur l'approfondisssmeent théorique des priors de réféerences en analyse Bayésienne objective. 
Nous dirions que son cadre saurait encore être enrichi



Aussi, notre intrinsection conduite entre priors de réfférence et courbes de fragilté dans des cas concrêt a démontré comment cette théorie et ses méthodes sait constituer un point d'enchrage à l'explicabilité d'estimation dans un modèle. 



Enfin, concernant l'estimation de courbe de fragilité sismique, peu d'appronfissement du modèle probit-lognormal sont envisagé. Cependant, l'estimation selon d'autres cadres, d'autres datasets et/ou d'autres modèles reste ouverte et reste un cadre dans lequel une étude objective peut-être conduite. On peut citer...














\section{Final words}\label{sec:concl:final}


Nous n'avons pas dans cette thèse contruit le prior objecctif par essence, ni même estimé la corube de fragilité exacte de fait.
Cependant, en s'adressant à la fois séparément mais aussi très poreuseent à ces deux questions nous nous sommes approché au plus près possible de ces deux idéaux.
%Nous croyons que ce raprochement s'est fait en contribuant à la fois indépendament au domaine des courbes de fragilité 
% Nous croyons que
Nous concluons en appuyant que les méthodologies nouvelles apportées par cette thèse, ainsi que les résultats théoriques, formels, et expérimentaux qui les accompagnent démontre de leur capacité à intelligement balancer les raprochements aux deux idéaux de sorte à répondre pour le mieux à la problématique définie par le sujet de thèse.
% Ceci étant d'autant plus appuyé par le fait qu'une part substantielle de ces résultats a été publiée après peer reviewing.

% Nous insistons que tous nos résultats ont été longuement analysés et relus, et tous 
%De cette manière, ces résultats, dont la portée dans leur domaine a déjà ou est en cours d'être jugée par le peer-reviewing, ont une portée qui va au delà.Chacun de ces résultats
Pour chacun des résultats qui fait parti de ceux qui articulent cette conclusion, nous insistons sur la rigueur apporté à notre approche, de sorte à appuyer au mieux la percutance du travail à la problématique.
Chacune des contributions individuelle a été soumise, et est dans certain cas déjà validées, à la communauté.
Le sérieux de notre démarche et de ces derniers mots est ainsi supporté.









% \section{Outlook}


% \adjusmptc

\newpage

