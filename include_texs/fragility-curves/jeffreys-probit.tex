


\begin{abstract}[\hspace*{-10pt}]
    This chapter draws mainly on the published work: \fullcite{van_biesbroeck_reference_2024}  % Ce chapitre reprend principalement les travaux publiés dans: 
\end{abstract}

\begin{abstract}
    abstract
\end{abstract}


\minitoc

\section{Introduction}


\section{Probit-lognormal model and Bayesian framework}

We remind that the probit-lognormal model of the fragility curve has been described in the \cref{chap:frags-intro}.
It defines the fragility curve of the mechanical system of interest as 
\begin{equation}\label{eq:PEM:probitfrag}
    P_f(a)=\PP({``failure''}|IM=a) = \Phi\left(\frac{\log a-\log\alpha}{\beta}\right),
\end{equation}
where $\Phi$ is the c.d.f. of a standard Gaussian, and $\alpha$, $\beta$ are parameters that we seek to estimate. To be precise, $\alpha\in(0,\infty)$ is the median and $\beta\in(0,\infty)$ is the log standard deviation of the curve. We denote $\theta=(\alpha,\beta)\in\Theta=(0,\infty)^2$.

In statistical terms, we consider that the failure of the equipment is the realization of a random variable $Z$, which takes values in $\{0,1\}$ ($1$ for failure, $0$ for non-failure). We also denote by $A$ the random variable of the IM. It takes value in a set $\cA=(0,\infty)$ and is supposed to follow a distribution $H$. Conditionally to $\theta$, the tuple $Y=(Z,A)$ follows a distribution defined by $A\sim H$ and $Z|A,\theta\sim\cB(P_f(a))$, where $\cB(p)$ denotes the Bernoulli distribution of parameter $p$, and $P_f(a)$ is defined in \cref{eq:PEM:probitfrag}.

We recall that given realizations $(\mbf z_k,\mbf a_k)$, where $\mbf z_k=(z_i)_{i=1}^k$, $\mbf a_k=(a_i)_{i=1}^k$, of the r.v. $Y$, this model admits the following likelihood:
\begin{equation}
    \ell_k(\mbf z^k|\mbf a^k,\theta) = \prod_{i=1}^k\ell(z_i|a_i,\theta) = \prod_{i=1}^k\Phi\left(\frac{\log a_i-\log\alpha}{\beta}\right)^{z_i}\left(1-\Phi\left(\frac{\log a_i-\log\alpha}{\beta}\right)\right)^{1-z_i}.
\end{equation}




We also recall that the reference prior theory is comprehensively introduced in the \cref{chap:intro-ref}. 





\section{Jeffreys prior construction in the probit-lognormal model}


    \subsection{Derivation of the Jeffreys prior}


    \subsection{Practical implementation}



    \subsection{Thorough study of the prior's decay rates}




\section{Competing approaches and estimation tools}



    \subsection{Bayesian estimates of the seismic fragility curve}

    \subsection{Competing prior taken from the literature}

    \subsection{Maximum likelihood estimation with bootstrapping}




\section{Limits of the estimates given by the three approaches: the curse of degeneracy}


\section{Performance evaluation metrics}



\section{Numerical applications}


\subsection{Case study 1: the elasto-plastic oscillator}

\subsection{Case study 2: the piping system from a pressurized reactor}

\subsection{Case study 3: the stacked structure for storage of packages}



\section{Conclusion}

