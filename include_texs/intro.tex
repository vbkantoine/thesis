

\chapter{Introduction}




\section{Motivation and positioning of the thesis}

\subsection{Uncertainty quantification in seismic probabilistic risk assessment studies}

\begin{itemize}
    \item The concept
    \item The criticality of nuclear industry, risen of methods, PBEE, fragility curves in this context
    \item The SPRA in France, and in the CEA \cite{roger_seisme_2020}
    \item Tools of UQ, Sensitivity analysis
\end{itemize}


\subsection{The choice of the prior in Bayesian studies}

\begin{itemize}
    \item Bayesian studies
    \item Their role in real world studies
    \item The impact of the prior and the research of objectivity
\end{itemize}


\subsection{Motivating prior elicitation research for SPRA studies}

\begin{itemize}
    \item Bayes as a practical tool for prediction in UQ in SPRA
    \item The critical role of objectivity, in contexts with little datasets
\end{itemize}



\section{Outline of the manuscript and contributions}

\begin{itemize}
    \item The questions that this manuscript seeks to answer
    \item Some objective of the thesis
    \item Contributions in two areas: theory and practice. They actually are subtly intricate: practical needs motivated theoretical developments and theoretical results provided practical solutions to enhance the state-of-the-art
    \item Exhaustive list of contributions
\end{itemize}






\chapter{Introduction en français}

\section{Motivation et positionnement de la thèse}

\section{Esquisse du manuscrit et constributions}