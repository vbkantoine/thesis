\usepackage[utf8]{inputenc}

\usepackage{geometry}
\usepackage{bookmark,hyperref}
\usepackage{xcolor}
\usepackage[english]{babel}
\usepackage{graphicx}


\usepackage{amsmath}
\usepackage{amsfonts}
\usepackage{amsthm}

\usepackage{mathrsfs}
\usepackage{bbm}
\usepackage{enumitem}
\usepackage{cleveref}
\usepackage{titlesec}
\usepackage{minitoc}
\usepackage{fancyhdr}
\usepackage{ragged2e}
\usepackage{algorithm}
\usepackage{algorithmic}

\usepackage{tikz}
\usetikzlibrary{arrows.meta}

\hypersetup{colorlinks, linkcolor=red, citecolor=blue, urlcolor=blue, bookmarksopen=true, bookmarksopenlevel=0}



% mathbb
\newcommand{\CC}{\mathbb{C}}
\newcommand{\EE}{\mathbb{E}}
\newcommand{\NN}{\mathbb{N}}
\newcommand{\PP}{\mathbb{P}}
\newcommand{\QQ}{\mathbb{Q}}
\newcommand{\RR}{\mathbb{R}}
\newcommand{\VV}{\mathbb{V}}

% mathcal
\newcommand{\cA}{\mathcal{A}}
\newcommand{\cB}{\mathcal{B}}
\newcommand{\cC}{\mathcal{C}}
\newcommand{\cD}{\mathcal{D}}
\newcommand{\cE}{\mathcal{E}}
\newcommand{\cF}{\mathcal{F}}
\newcommand{\cG}{\mathcal{G}}
\newcommand{\cH}{\mathcal{H}}
\newcommand{\cI}{\mathcal{I}}
\newcommand{\cJ}{\mathcal{J}}
\newcommand{\cK}{\mathcal{K}}
\newcommand{\cL}{\mathcal{L}}
\newcommand{\cM}{\mathcal{M}}
\newcommand{\cN}{\mathcal{N}}
\newcommand{\cO}{\mathcal{O}}
\newcommand{\cP}{\mathcal{P}}
\newcommand{\cQ}{\mathcal{Q}}
\newcommand{\cR}{\mathcal{R}}
\newcommand{\cS}{\mathcal{S}}
\newcommand{\cT}{\mathcal{T}}
\newcommand{\cV}{\mathcal{V}}
\newcommand{\cW}{\mathcal{W}}
\newcommand{\cX}{\mathcal{X}}
\newcommand{\cY}{\mathcal{Y}}
\newcommand{\cZ}{\mathcal{Z}}

% mathscr
\newcommand{\sA}{\mathscr{A}}
\newcommand{\sB}{\mathscr{B}}
\newcommand{\sI}{\mathscr{I}}
\newcommand{\sM}{\mathscr{M}}
\newcommand{\sO}{\mathscr{O}}
\newcommand{\sP}{\mathscr{P}}
\newcommand{\sR}{\mathscr{R}}
\newcommand{\sT}{\mathscr{T}}
\newcommand{\sX}{\mathscr{X}}
\newcommand{\sY}{\mathscr{Y}}
\newcommand{\sZ}{\mathscr{Z}}

% \theoremstyle{plain}
% \newtheorem{thm}{Theorem}[chapter]
% \crefname{thm}{theorem}{theorems}
% \newtheorem{prop}{Proposition}[chapter]
% \crefname{prop}{proposition}{propositions}
% \newtheorem{lem}{Lemma}[chapter]
% \crefname{lem}{lemma}{lemmas}
% \newtheorem{cor}{Corollary}[chapter]
% \crefname{cor}{corollary}{corollaries}

% \theoremstyle{definition}
% \newtheorem{defi}{Definition}[chapter]
% \crefname{defi}{definition}{definitions}
% \newtheorem{assu}{Assumption}[chapter]
% \crefname{assu}{assumption}{assumptions}
% \newtheorem{rem}{Remark}[chapter]
% \crefname{rem}{remark}{remarks}
% \newtheorem{ex}{Example}[chapter]
% \crefname{ex}{example}{examples}

% operateurs
\DeclareMathOperator{\Tr}{Tr}
\DeclareMathOperator{\Supp}{Supp}
\DeclareMathOperator{\Vect}{Vect}
\DeclareMathOperator*{\argmin}{arg\,min}
\DeclareMathOperator*{\argmax}{arg\,max}
\DeclareMathOperator{\Cov}{Cov}
\DeclareMathOperator{\Var}{Var}
\DeclareMathOperator{\erf}{erf}
\DeclareMathOperator*{\tend}{\to}
\newcommand{\rD}{\mathrm{D}}
\def\bX{\mathbf{X}}
\DeclareMathOperator{\diag}{diag}
\DeclareMathOperator{\Span}{Span}
\DeclareMathOperator{\Image}{im}
\DeclareMathOperator{\Kernel}{ker}


\newcommand{\conv}[2][\ ]{\overset{#1}{\underset{#2}{\to}}}
\newcommand{\aseq}[2][]{\overset{#1}{\underset{#2}{=}}}
\newcommand{\equi}[1]{\underset{#1}{\sim}}
\newcommand{\mbf}[1]{\mathbf{#1}}
\newcommand{\indic}{\mathbbm{1}}
\newcommand{\hatmbf}[1]{\hat{\mathbf{#1}}}
\let\eps\varepsilon
\let\to\longrightarrow


\newenvironment{abstract}[1][Abstract]{\paragraph{#1}}{}





% \defbibheading{none}{%
% }
% \DeclareSourcemap{
%   \maps[datatype=bibtex]{
%     \map[overwrite]{
%       \perdatasource{biblio.bib}
%       \step[fieldset=keywords, fieldvalue={,conf}, append]
%     }
%   }
% }




\newcommand{\logoEd}{EDMH}																		%% Logo de l'école doctorale. Indiquer le sigle (EDIPP, EDMH) / Doctoral school logo. Indicate the acronym : EDMH, EDIPP
\newcommand{\PhDTitleFR}{Priors de référence: inférence objective et pratique, appliqués pour des estimateurs auditable de courbes de fragilité sismique}													%% Titre de la thèse en français / Thesis title in french
\newcommand{\keywordsFR}{Prior objectif, Analyse bayésienne, Aléa sismique, Courbe de fragilité, Quantification des incertitudes}														%% Mots clés en français, séprarés par des , / Keywords in french, separated by ,
\newcommand{\abstractFR}{Dans le cadre de l’évaluation de la sûreté sismique des structures et des équipements d’installations industrielles, les courbes de fragilité des structures de génie civil et des équipements représentent un outil d’aide à la décision. Issu des études probabilistes de sûreté menées sur ces installations, elles expriment la probabilité d’une défaillance de l’entité conditionnée à une intensité de mesure de l’aléa sismique, ou bien même en pratique d’autre phénomènes tels que le vent ou la houle. L’évaluation de ces courbes par méthodes de Monte-Carlo et l’emploi de simulations numériques de mécanique et de signaux sismiques générés est la méthode la plus souvent plébiscitée mais dont la fiabilité est questionnée lorsque le jeu de données à disposition est faible. A ce titre et en connaissance de la nécessité de compréhension de comportement « ultime » de la structure pour permettre sa simulation de réponse à l’événement sismique considéré, alors que celles-ci présentent dans la majorité des cas un comportement non linéaire, l’enrichissement d’une base de données présente une complexité qui impliquent des temps de calculs prohibitifs. L’apport du cadre bayésien s’inscrit donc dans cette démarche d’un apprentissage plus efficace sous connaissance a priori de la répartition probabiliste des paramètres du modèle que constitue la courbe de fragilité. Leur indicateur de confiance se dirige alors dans le choix de cette loi de probabilité a priori, pour lequel toute subjectivité serait critiquable. Les méthodes de choix et de calcul de prior objectifs répondent à ce problème en s’appuyant sur des critères définis et objectifs, enrichissant la théorie des prior de références en l’appliquant à la problématique des courbes de fragilité sismiques.}															%% Résumé en français / abstract in french

\newcommand{\PhDTitleEN}{\PhDTitle}													%% Titre de la thèse en anglais / Thesis title in english
\newcommand{\keywordsEN}{Objective prior, Bayesian analysis, Seismic hazard, Fragility curve, Uncertainties quantification}														%% Mots clés en anglais, séprarés par des , / Keywords in english, separated by ,
\newcommand{\abstractEN}{%
Reference prior theory provides a principled framework for objective Bayesian inference, aiming to minimize subjective input and allow data-based information to drive the estimates distribution. %This is particularly valuable in high-stakes applications, where transparency, auditability, and robustness are essential. 
% This theory becomes particularly relevant for
For this reason, the application of this theory to the estimation of seismic fragility curves becomes particularly relevent.
Indeed, seismic fragility curves are essential elements of seismic probabilistic risk assemssment studies, they define the probability of failure of a mechanical structure as a function of seismic scenarios. Since they inform critcal decisions in infrastrucure safety, a complete auditability of the pipeline leading to estimates of thess curves is required.
% they require a total auditability and robustness of the pipeline that produce estimates of them.

%One such application is the estimation of seismic fragility curves—probabilistic tools that characterize structural failure likelihood as a function of earthquake intensity. These curves play a central role in seismic probabilistic risk assessment, where they inform critical decisions in infrastructure safety and resilience.

This thesis investigates the interplay between reference prior theory and seismic fragility estimation, yielding original contributions in three directions. First, we advance the theoretical foundations of reference priors by developing novel constructions of them. 
Our goal is to support their objectivity, while improving their practical appicability. Our results take the form of three theoretical contributions in this domain that propose (i)~an extension of the thoery that supports the objectivity of the Jeffreys prior, (ii)~constraints that can be incorporated to the prior to enhance.
%Our goal is to question the balance between obejctivity
%under constraints, improving their practical applicability, and proposing ways to evaluate and compare their objectivity. 
Second, we revisit the estimation of seismic fragility curves under the prominent probit-lognormal model, in a context where the data are particularly sparse.
Our goal is to conduct a Bayesian estimation of seimsic fragility curves that leverage the optimmization of every sorts of information, including the \emph{a priori} one, in order to provides estimates that are robust and auditable
Our resutls highlight the limitations and irregularities of the model and propose methods that provide accurate and efficient estimates of the curves. The evaluation of our approaches are carried out on different case studies taken from the nuclear industry.

%a common yet delicate setting due to data sparsity and binary observations. We formally compute the Jeffreys prior for this model, highlight its limitations, and introduce alternative priors that address issues of degeneracy and inefficiency. We also incorporate optimal experimental design to improve estimation quality.

This thesis finally builds a strong link between these two domains.
The application to seismic fragility curves not only motivates theoretical developments but also benefits directly from them, yielding estimation pipelines that are more robust, interpretable, and auditable. Through this bi-directional dialogue, we demonstrate how abstract statistical theory can be meaningfully applied to real-world engineering problems.
We illustrate how reference prior theory can be extended and applied to meet the needs of modern reliability analysis.}

\renewcommand{\abstractFR}{\abstractEN}



% The reference prior theory provides the means to define a prior that can be qualified as ``objective''.
% stands as a cornerstone of objective Bayesian analysis. By formally defining priors that aim to minimize the influence of subjective information, it offers a principled framework for drawing inferences that are driven primarily by the observed data. 
% For this reason, the framework of reference prior theory becomes significantly relevant for the estimation of seimsic fragilty curves. Indeed, these curves represent a central component of seismic probabilistic risk assessment studies as they quantify the probability of failure of a mechanical component under a specific seimic scenario. 

% The work developed in this thesis is threefold. First, we make original contributions to the theory of reference priors itself. While the theory has matured significantly since its inception, limitations remain—particularly in its capacity to yield practical and well-behaved priors in complex or constrained settings. We propose new ways to construct and interpret reference priors under conditional constraints, investigate their asymptotic properties, and develop diagnostic tools to compare competing priors in terms of objectivity and robustness. These developments both clarify and extend the mathematical foundations of the theory, paving the way for its application in real-world settings where priors must meet practical requirements.

% Second, we focus on the estimation of seismic fragility curves, specifically under the widely used probit-lognormal model. In this context, observations are typically binary (e.g., structure failed or survived), and data scarcity often undermines the stability of classical estimates. We revisit the estimation problem from a Bayesian perspective, beginning with the formal computation of the Jeffreys prior for this model. We show that, while theoretically objective, this prior exhibits limitations in practice—such as posterior degeneracy in low-information regimes. We respond with methodological innovations, including a new class of constrained reference priors that explicitly address these issues. Furthermore, we explore how experimental design can be integrated into the inferential process to optimally extract information from limited data, significantly improving the robustness and interpretability of fragility estimates.

% Third, and most critically, we bridge the gap between these two domains—using the application to seismic fragility curves as both a proving ground and a source of inspiration for the development of reference prior theory. This bi-directional interplay is a central theme of the thesis. On one hand, the challenges encountered in fragility modeling motivate theoretical advances in the construction and evaluation of priors. On the other, these advances enable new and more reliable methods for estimating fragility curves, especially in data-poor settings. For instance, we demonstrate how sensitivity to prior assumptions can be controlled and quantified within the reference prior framework, enhancing the explainability and auditability of the final risk estimates.

% Taken together, the results of this work not only advance the frontier of objective Bayesian analysis but also deliver tangible improvements to the practice of seismic risk assessment. The integration of these two threads—reference prior theory and fragility curve estimation—offers a powerful example of how abstract statistical ideas can be harnessed to meet concrete engineering needs.
% }															
%% Résumé en anglais / abstract in english
