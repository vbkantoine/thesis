\usepackage[utf8]{inputenc}

\usepackage{geometry}
\usepackage{bookmark,hyperref}
\usepackage{xcolor}
\usepackage[english]{babel}
\usepackage{graphicx}


\usepackage{amsmath}
\usepackage{amsfonts}
\usepackage{amsthm}

\usepackage{mathrsfs}
\usepackage{bbm}
\usepackage{enumitem}
\usepackage{cleveref}
\usepackage{titlesec}
\usepackage{minitoc}
\usepackage{fancyhdr}
\usepackage{ragged2e}
\usepackage{algorithm}
\usepackage{algorithmic}

\usepackage{tikz}
\usetikzlibrary{arrows.meta}

\hypersetup{colorlinks, linkcolor=red, citecolor=blue, urlcolor=blue, bookmarksopen=true, bookmarksopenlevel=0}



% mathbb
\newcommand{\CC}{\mathbb{C}}
\newcommand{\EE}{\mathbb{E}}
\newcommand{\NN}{\mathbb{N}}
\newcommand{\PP}{\mathbb{P}}
\newcommand{\QQ}{\mathbb{Q}}
\newcommand{\RR}{\mathbb{R}}
\newcommand{\VV}{\mathbb{V}}

% mathcal
\newcommand{\cA}{\mathcal{A}}
\newcommand{\cB}{\mathcal{B}}
\newcommand{\cC}{\mathcal{C}}
\newcommand{\cD}{\mathcal{D}}
\newcommand{\cE}{\mathcal{E}}
\newcommand{\cF}{\mathcal{F}}
\newcommand{\cG}{\mathcal{G}}
\newcommand{\cH}{\mathcal{H}}
\newcommand{\cI}{\mathcal{I}}
\newcommand{\cJ}{\mathcal{J}}
\newcommand{\cK}{\mathcal{K}}
\newcommand{\cL}{\mathcal{L}}
\newcommand{\cM}{\mathcal{M}}
\newcommand{\cN}{\mathcal{N}}
\newcommand{\cO}{\mathcal{O}}
\newcommand{\cP}{\mathcal{P}}
\newcommand{\cQ}{\mathcal{Q}}
\newcommand{\cR}{\mathcal{R}}
\newcommand{\cS}{\mathcal{S}}
\newcommand{\cT}{\mathcal{T}}
\newcommand{\cV}{\mathcal{V}}
\newcommand{\cW}{\mathcal{W}}
\newcommand{\cX}{\mathcal{X}}
\newcommand{\cY}{\mathcal{Y}}
\newcommand{\cZ}{\mathcal{Z}}

% mathscr
\newcommand{\sA}{\mathscr{A}}
\newcommand{\sB}{\mathscr{B}}
\newcommand{\sI}{\mathscr{I}}
\newcommand{\sM}{\mathscr{M}}
\newcommand{\sO}{\mathscr{O}}
\newcommand{\sP}{\mathscr{P}}
\newcommand{\sR}{\mathscr{R}}
\newcommand{\sT}{\mathscr{T}}
\newcommand{\sX}{\mathscr{X}}
\newcommand{\sY}{\mathscr{Y}}
\newcommand{\sZ}{\mathscr{Z}}

% \theoremstyle{plain}
% \newtheorem{thm}{Theorem}[chapter]
% \crefname{thm}{theorem}{theorems}
% \newtheorem{prop}{Proposition}[chapter]
% \crefname{prop}{proposition}{propositions}
% \newtheorem{lem}{Lemma}[chapter]
% \crefname{lem}{lemma}{lemmas}
% \newtheorem{cor}{Corollary}[chapter]
% \crefname{cor}{corollary}{corollaries}

% \theoremstyle{definition}
% \newtheorem{defi}{Definition}[chapter]
% \crefname{defi}{definition}{definitions}
% \newtheorem{assu}{Assumption}[chapter]
% \crefname{assu}{assumption}{assumptions}
% \newtheorem{rem}{Remark}[chapter]
% \crefname{rem}{remark}{remarks}
% \newtheorem{ex}{Example}[chapter]
% \crefname{ex}{example}{examples}

% operateurs
\DeclareMathOperator{\Tr}{Tr}
\DeclareMathOperator{\Supp}{Supp}
\DeclareMathOperator{\Vect}{Vect}
\DeclareMathOperator*{\argmin}{arg\,min}
\DeclareMathOperator*{\argmax}{arg\,max}
\DeclareMathOperator{\Cov}{Cov}
\DeclareMathOperator{\Var}{Var}
\DeclareMathOperator{\erf}{erf}
\DeclareMathOperator*{\tend}{\to}
\newcommand{\rD}{\mathrm{D}}
\def\bX{\mathbf{X}}
\DeclareMathOperator{\diag}{diag}
\DeclareMathOperator{\Span}{Span}
\DeclareMathOperator{\Image}{im}
\DeclareMathOperator{\Kernel}{ker}


\newcommand{\conv}[2][\ ]{\overset{#1}{\underset{#2}{\to}}}
\newcommand{\aseq}[2][]{\overset{#1}{\underset{#2}{=}}}
\newcommand{\equi}[1]{\underset{#1}{\sim}}
\newcommand{\mbf}[1]{\mathbf{#1}}
\newcommand{\indic}{\mathbbm{1}}
\newcommand{\hatmbf}[1]{\hat{\mathbf{#1}}}
\let\eps\varepsilon
\let\to\longrightarrow


\newenvironment{abstract}[1][Abstract]{\paragraph{#1}}{}





% \defbibheading{none}{%
% }
% \DeclareSourcemap{
%   \maps[datatype=bibtex]{
%     \map[overwrite]{
%       \perdatasource{biblio.bib}
%       \step[fieldset=keywords, fieldvalue={,conf}, append]
%     }
%   }
% }




\newcommand{\PhDTitle}{%
	%Reference priors: objective and practical inference,applied for auditable estimations of seismic fragility curves\\%
    %
    Extended reference prior theory for objective and practical inference, application to %\makebox[4em][r]
    {robust and auditable seismic fragility curves%}
}}

\newcommand{\logoEd}{EDMH}																		%% Logo de l'école doctorale. Indiquer le sigle (EDIPP, EDMH) / Doctoral school logo. Indicate the acronym : EDMH, EDIPP
\newcommand{\PhDTitleFR}{Théory des priors de référence étendue pour une inférence objective et pratique, application pour des estimateurs auditable de courbes de fragilité sismique}													%% Titre de la thèse en français / Thesis title in french
\newcommand{\keywordsFR}{Prior objectif, Analyse bayésienne, Aléa sismique, Courbe de fragilité, Quantification des incertitudes}														%% Mots clés en français, séprarés par des , / Keywords in french, separated by ,
\newcommand{\abstractFR}{%
La théorie des priors de référence fournit un cadre approprié à une inférence bayésienne objective, puisqu'elle %visant à minimiser
vise à minimiser la subjectivité introduite et à permettre aux informations issues des données d'orienter la distribution des estimations. Pour cette raison, l'application de cette théorie à l'estimation des courbes de fragilité sismique est particulièrement pertinente. En effet, ces courbes
sont des éléments essentiels
des études simsmiques probabilistes de sûreté~; elles expriment la probabilité de défaillance d'une structure mécanique en fonction d'indateurs définissant des scénarios sismiques. Puisqu'elles informent des décisions critiques en matière de sécurité des infrastructures, une auditabilité complète de l'approche qui conduit aux estimations de ces courbes est nécessaire.

Cette thèse étudie l'interaction entre la théorie des priors de référence et l'estimation des courbes de fragilité sismique, apportant des contributions originales dans ces deux domaines. Tout d'abord, nous complétons les fondements théoriques des priors de référence en développant de nouvelles constructions de ceux-ci. Notre objectif est de soutenir leur objectivité tout en améliorant leur applicabilité pratique. Nos résultats prennent la forme de contributions théoriques dans ce domaine qui sont basées sur une définition généralisée de l'information mutuelle. Nos approches abordent les principaux problèmes des priors de référence, à savoir le caractère impropre de leur distribution ou de leu distribution \emph{a posteriori}, et leur formulation complexe pour une utilisation pratique.

Esnuite, nous revisitons l'estimation des courbes de fragilité sismique basée sur le modèle probit-lognormal dans un contexte où les données sont particulièrement rares. Notre objectif est de réaliser une estimation bayésienne des courbes de fragilité qui tire parti de l'optimisation de toutes sortes d'informations, y compris l'information \emph{a priori}, afin de fournir des estimations robustes et auditables. Nos résultats mettent en évidence les limites et les irrégularités du modèle et proposent des méthodes qui fournissent des estimations précises et efficaces des courbes. Les évaluations de nos approches sont réalisées sur différents cas d'étude issus de l'industrie nucléaire.

Cette thèse établit un lien fort entre ces deux domaines. L'application aux courbes de fragilité sismique a non seulement motivé les développements théoriques mais leur a aussi directement profité, produisant finalement un cadre d'estimation plus robuste, interprétable et vérifiable.
}															%% Résumé en français / abstract in french

\newcommand{\PhDTitleEN}{\PhDTitle}													%% Titre de la thèse en anglais / Thesis title in english
\newcommand{\keywordsEN}{Objective prior, Bayesian analysis, Seismic hazard, Fragility curve, Uncertainties quantification}														%% Mots clés en anglais, séprarés par des , / Keywords in english, separated by ,
\newcommand{\abstractEN}{%
Reference prior theory provides a principled framework for objective Bayesian inference, aiming to minimize subjective input and allow data-based information to drive the estimates distribution. %This is particularly valuable in high-stakes applications, where transparency, auditability, and robustness are essential. 
% This theory becomes particularly relevant for
For this reason, the application of this theory to the estimation of seismic fragility curves is particularly relevant.
Indeed, these curves are essential elements of seismic probabilistic risk assessment studies; they express the probability of failure of a mechanical structure as a function of indicators that define 
seismic scenarios. Since they inform critical decisions in infrastructure safety, a complete auditability of the pipeline that leads to the estimates of these curves is required.
% they require a total auditability and robustness of the pipeline that produce estimates of them.

%One such application is the estimation of seismic fragility curves—probabilistic tools that characterize structural failure likelihood as a function of earthquake intensity. These curves play a central role in seismic probabilistic risk assessment, where they inform critical decisions in infrastructure safety and resilience.

This thesis investigates the interplay between reference prior theory and seismic fragility curves estimation, yielding original contributions in these two domains. First, we complement the theoretical foundations of reference priors by developing novel constructions of them. 
Our goal is to support their objectivity while improving their practical applicability. Our results take the form of
theoretical contributions in this domain that are based on a generalized definition of the mutual information.
Our approaches tackle the principal issues of reference priors, namely their improper characteristic or that of their posterior, and their complex formulation for practical use.

%  definition of reference priors
% We tackle the pincipal issues encountered by using the referebce prios in practice by proposing methods that address 
% We provide adaptations of the reference priors definiton
% three theoretical contributions in this domain that propose (i)~an extension of the thoery that supports the objectivity of the Jeffreys prior, (ii)~constraints that can be incorporated to the prior to enhance.
%Our goal is to question the balance between obejctivity
%under constraints, improving their practical applicability, and proposing ways to evaluate and compare their objectivity. 
Second, we revisit the estimation of seismic fragility curves based on the prominent probit-lognormal model in a context where the data are particularly sparse.
Our goal is to conduct a Bayesian estimation of seismic fragility curves that leverages the optimization of every sort of information, including the \emph{a priori} one, in order to provide estimates that are robust and auditable.
Our results highlight the limitations and irregularities of the model and propose methods that provide accurate and efficient estimates of the curves. The evaluations of our approaches are carried out on different case studies taken from the nuclear industry.

%a common yet delicate setting due to data sparsity and binary observations. We formally compute the Jeffreys prior for this model, highlight its limitations, and introduce alternative priors that address issues of degeneracy and inefficiency. We also incorporate optimal experimental design to improve estimation quality.

This thesis builds a strong link between these two domains.
The application to seismic fragility curves not only motivated theoretical developments but also directly benefited them, ultimately producing a more robust, interpretable, and verifiable estimation framework.
%The application to seismic fragility curves not only motivates theoretical developments but also benefits directly from them, yielding estimation pipelines that are more robust, interpretable, and auditable. %Through this bi-directional dialogue, %, we demonstrate how abstract statistical theory can be meaningfully applied to real-world engineering problems.
%we illustrate how reference prior theory can be extended and applied to meet the needs of modern reliability analysis.
}

% \renewcommand{\abstractFR}{\abstractEN}



% The reference prior theory provides the means to define a prior that can be qualified as ``objective''.
% stands as a cornerstone of objective Bayesian analysis. By formally defining priors that aim to minimize the influence of subjective information, it offers a principled framework for drawing inferences that are driven primarily by the observed data. 
% For this reason, the framework of reference prior theory becomes significantly relevant for the estimation of seimsic fragilty curves. Indeed, these curves represent a central component of seismic probabilistic risk assessment studies as they quantify the probability of failure of a mechanical component under a specific seimic scenario. 

% The work developed in this thesis is threefold. First, we make original contributions to the theory of reference priors itself. While the theory has matured significantly since its inception, limitations remain—particularly in its capacity to yield practical and well-behaved priors in complex or constrained settings. We propose new ways to construct and interpret reference priors under conditional constraints, investigate their asymptotic properties, and develop diagnostic tools to compare competing priors in terms of objectivity and robustness. These developments both clarify and extend the mathematical foundations of the theory, paving the way for its application in real-world settings where priors must meet practical requirements.

% Second, we focus on the estimation of seismic fragility curves, specifically under the widely used probit-lognormal model. In this context, observations are typically binary (e.g., structure failed or survived), and data scarcity often undermines the stability of classical estimates. We revisit the estimation problem from a Bayesian perspective, beginning with the formal computation of the Jeffreys prior for this model. We show that, while theoretically objective, this prior exhibits limitations in practice—such as posterior degeneracy in low-information regimes. We respond with methodological innovations, including a new class of constrained reference priors that explicitly address these issues. Furthermore, we explore how experimental design can be integrated into the inferential process to optimally extract information from limited data, significantly improving the robustness and interpretability of fragility estimates.

% Third, and most critically, we bridge the gap between these two domains—using the application to seismic fragility curves as both a proving ground and a source of inspiration for the development of reference prior theory. This bi-directional interplay is a central theme of the thesis. On one hand, the challenges encountered in fragility modeling motivate theoretical advances in the construction and evaluation of priors. On the other, these advances enable new and more reliable methods for estimating fragility curves, especially in data-poor settings. For instance, we demonstrate how sensitivity to prior assumptions can be controlled and quantified within the reference prior framework, enhancing the explainability and auditability of the final risk estimates.

% Taken together, the results of this work not only advance the frontier of objective Bayesian analysis but also deliver tangible improvements to the practice of seismic risk assessment. The integration of these two threads—reference prior theory and fragility curve estimation—offers a powerful example of how abstract statistical ideas can be harnessed to meet concrete engineering needs.
% }															
%% Résumé en anglais / abstract in english
